\documentclass[10p,a4paper,BCOR = 12mm, DIV=15]{scrbook}
\usepackage[utf8x]{inputenc}
\usepackage{ngerman}
\usepackage{amssymb}
\usepackage{amsthm}
\usepackage{amsmath}
\usepackage{mathtools} % for shortintertext
\usepackage{paralist}
\usepackage[numbers,square]{natbib}
\usepackage{setspace}
\author{Wolfgang Keller}
\title{Erweiterte Formulierungen des Spannbaum-Polytops}

\setcounter{secnumdepth}{3}
\setcounter{tocdepth}{3}

\begin{document}
\newtheorem{Def}{Definition}
\newtheorem{Le}[Def]{Lemma}
\newtheorem{Sa}[Def]{Satz}
\newtheorem{Kor}[Def]{Korollar}
\newtheorem{Prop}[Def]{Proposition}
\newtheorem{Pro}[Def]{Problem}
\newtheorem{Bem}[Def]{Bemerkung}
\newtheorem{Bsp}[Def]{Beispiel}
\newtheorem{BspDef}[Def]{Beispiel/Definition}
\newtheorem{Ver}[Def]{Vermutung}

\newcommand{\rec}{\operatorname{rec}}
\newcommand{\cone}{\operatorname{cone}}
\newcommand{\conv}{\operatorname{conv}}
\newcommand{\bild}{\operatorname{bild}}
\newcommand{\rg}{\operatorname{rg}}
\newcommand{\proj}{\operatorname{proj}}
\newcommand{\vertices}{\operatorname{vert}}
\newcommand{\aff}{\operatorname{aff}}
\newcommand{\sgn}{\operatorname{sgn}}
\newcommand{\trace}{\operatorname{tr}}

\newenvironment{bew}{\begin{proof}[Beweis]}{\end{proof}}

\bibliographystyle{alphadin}

\maketitle

\tableofcontents

\chapter*{Eingangszitate}

\begin{quote}
"`Bäume sind Gedichte, die die Erde in den Himmel schreibt. Wir fällen sie und verwandeln sie in Papier, um unsere Leere darauf auszudrücken."' \\
\rightline{\scriptsize{Khalil Gibran (1883 - 1931), Sämtliche Werke}}
\end{quote}

\begin{quote}
"`Ein Narr sieht nicht denselben Baum, den ein Weiser sieht."' \\
\rightline{\scriptsize{William Blake (1757 - 1827), Die Hochzeit von Himmel und Hölle}}
\end{quote}

\begin{quote}
"`A tree is an incomprehensible mystery."' \\
\rightline{\scriptsize{Jim Woodring (*1952)}}
\end{quote}

\begin{quote}
"`A tree's a tree. How many more do you need to look at?"' \\
\rightline{\scriptsize{Ronald Reagan (1911-2004)}}
\end{quote}

\begin{quote}
"`Behind every tree there's a new monster."' \\
\rightline{\scriptsize{Todd Rundgren (*1948)}}
\end{quote}

\part{Definitionen und Grundlagen}

\label{part:def_grundlagen}

\chapter{Grundlegende Definitionen und Hintergründe}

\section{Allgemeine Definitionen}

\begin{Def}
Sei $S$ eine Menge und $n\in\mathbb{Z}_{\geq 0}$. Dann definieren wir
\begin{itemize}
\item $\left[n\right] := \left\{1, \ldots, n\right\}$
\item $S \choose n$ als die Menge aller n-elementigen Teilmengen von S
\item $S^{\underline{n}}$ als die Menge aller n-Tupel mit paarweise verschiedenen Elementen aus S
\item $S^{\underline{n}^<}$ als die Menge aller n-Tupel $\left(s_1, \ldots, s_n\right)$ mit paarweise verschiedenen Elementen aus S, so dass $s_1 < s_2 < \ldots s_n$ erfüllt ist
\item $0^X$ ($X$ beliebige Menge) als den Nullvektor des $\mathbb{R}^X$. Wenn offensichtlich ist, auf welchen Vektorraum sich bezogen wird, so verwenden wir auch $0$ als Bezeichnung für den Nullvektor
\item Sei $T \subseteq S$. Dann bezeichnen wir mit $\chi\left(T\right) \in \mathbb{R}^{S}$ den Vektor aus dem $\mathbb{R}^{S}$ mit $\left(\chi\left(T\right)\right)_s = 1 \Leftrightarrow s \in T$ und ${\chi\left(T\right)}_s = 0 \Leftrightarrow s \notin T$ (charakteristischer Vektor von $T$). Da in dieser Arbeit stets klar ist, auf welches $S$ bezogen wird, wird $\chi\left(T\right)$ nicht zusätzlich durch $S$ indiziert
\item $EV_n$ als Abkürzung für $\left({[n]\choose 2} \times [n]\right) \backslash \bigcup\limits_{\left\{v_1, v_2\right\} \in {[n]\choose 2}} \left(\left\{v_1, v_2\right\}, v_1\right) \cup \left(\left\{v_1, v_2\right\}, v_2\right)$
\item Für $n \geq 1$ definieren wir $a +_{\left[n\right]} b$ bzw. $a  -_{\left[n\right]} b$ als Abkürzung für $a + b \mod n$ bzw. $a - b \mod n$, wobei das Ergebnis in $\left[n\right]$ liegt (im Gegensatz zur "`normalen"' Addition bzw. Subtraktion modulo $n$)
\item $E \left(S\right) := {S \choose 2} \cap E$ und $A\left(S\right) := S^{\underline{2}} \cap A$
\item Für $T$ mit $S \cap T = \emptyset$ wollen wir $\delta\left(S, T\right) := \bigcup_{s \in S, t \in T} \left\{\left\{s, t\right\}\right\}$ definieren
\end{itemize}
\end{Def}

\section{Graphentheoretische Grundlagen}

\begin{Def}
Sei $V$ (Knotenmenge -- "`vertices"') endlich und nichtleer und sei $E \subseteq {V \choose 2}$ (Kantenmenge -- "`edges"'). Dann bezeichnen wir $\left(V, E\right)$ als \emph{Graph}.
%Für einen vorgegebenen Graph $G$ wollen wir seine Knotenmenge mit $V\left(G\right)$ und seine Kantenmenge mit $E\left(G\right)$ bezeichnen.
\end{Def}

\begin{Bem}
Manchmal werden Graphen allgemeiner definiert -- so lässt man in der Praxis je nach Anwendung beispielsweise auch
\begin{itemize}
\item Schleifen (loops) (Kanten von einem Knoten zu sich selbst)
\item Mehrfachkanten (parallele Kanten)
\item gerichtete Kanten (Bögen)
\end{itemize}
zu. Hierfür sind in der Literatur Begriffe wie "`nichtschlichte Graphen"', "`Multigraphen"', "`Digraphen"', "`gemischte Graphen"' etc. verbreitet. Für die Zwecke dieser Arbeit wollen wir diese allgemeineren Graphendefinitionen ignorieren (lediglich in Kapitel \ref{sec:branch_arb} werden wir auf Digraphen zurückgreifen -- die hierzu nötigen Definitionen werden dort vorgenommen).
\end{Bem}

\begin{BspDef}
Sei $n \geq 1$. Dann definieren wir $K_n := \left(\left[n\right], {\left[n\right] \choose 2}\right)$ als den \emph{vollständigen Graphen} auf der Knotenmenge $\left[n\right]$.

Für $n \geq 1$ definieren wir $W_n := \left(\left[n\right], \left\{\left\{1, 2\right\}, \left\{2, 3\right\}, \ldots, \left\{n-1, n\right\}\right\}\right)$ als den \emph{Weg-Graphen} auf der Knotenmenge $\left[n\right]$.

Für $n \geq 3$ definieren wir $C_n := \left(\left[n\right], \left\{\left\{1, 2\right\}, \left\{2, 3\right\}, \ldots, \left\{n-1, n\right\}, \left\{n, 1\right\}\right\}\right)$ als den \emph{Kreis-Graphen} auf der Knotenmenge $\left[n\right]$.
\end{BspDef}

\begin{Def}
Sei $G=(V, E)$ ein Graph und $v\in V$. Dann setzen wir 
\begin{displaymath}
\deg_G\left(v\right) := \left|\left\{w: \left\{v, w\right\} \in E\right\}\right|.
\end{displaymath}
\end{Def}

Nun zu den Konstrukten des \emph{Untergraphs} und der \emph{Isomorphie} von Graphen:

\begin{Def}
Sei $G=(V, E)$ ein Graph und $V' \subseteq E$ und $E' \subseteq {V' \choose 2} \cap E$. Dann heißt $\left(V', E'\right)$ ein \emph{Untergraph} oder \emph{Subgraph} von $G$.
\end{Def}

\begin{Def}
Seien $G_1=(V_1, E_1)$ und $G_2=(V_2, E_2)$ Graphen. Wir nennen $G_1$ und $G_2$ \emph{isomorph}, wenn es eine bijektive Abbildung $\Psi : V_1 \rightarrow V_2$ gibt, so dass für alle $\left\{v, v'\right\} \in {V_1 \choose 2}$ gilt: $\left\{\Psi\left(v\right), \Psi\left(v'\right)\right\} \in E_2 \Leftrightarrow \left\{v, v'\right\} \in E_1$ .
\end{Def}

\begin{Def}
\label{def:weg}
Sei $G=(V, E)$ ein Graph.

Eine Abfolge von paarweise verschiedenen Knoten und Kanten
\begin{displaymath}
\left(v_0, e_1, v_1 \ldots, v_{k-1}, e_k, v_k\right)
\end{displaymath}
mit $k \geq 0$ und $e_i = \left\{v_{i-1}, v_i\right\} \in E$ für alle $i \in \left[k\right]$ heißt ($v_0$-$v_k$-)Weg in $G$.

Für einen vorgegebenen Weg $W$ wollen wir die Menge seiner Knoten mit $V\left(W\right)$  und die Menge seiner Kanten mit $E\left(W\right)$ bezeichnen.
\end{Def}

\begin{Bem}
Nun können wir uns fragen, warum wir die Kanten $e_i$ in die Definition eines Weges aufnehmen. Diese sind in der Tat verzichtbar -- wir könnten also einen Weg einfach als eine Abfolge von paarweise verschiedenen Knoten $\left(v_0, v_1 \ldots, v_k\right)$ mit $k \geq 0$ und $\left\{v_{i-1}, v_i\right\} \in E$ für alle $i \in \left[k\right]$ definieren.

Der Grund, warum wir dennoch einen Weg gemäß Definition \ref{def:weg} definieren, liegt darin begründet, dass hierdurch jene Definition einfach auf Multi-Graphen verallgemeinert werden kann und deswegen in der Literatur üblich ist.
\end{Bem}

\begin{Def}
Durch "`es existiert zwischen $v_1, v_2 \in V$ in $G$ ein $v_1$-$v_2$-Weg"' wird auf der Menge der Knoten des Graphen $G=\left(V, E\right)$ eine Äquivalenzrelation definiert (Reflexivität und Symmetrie sind offensichtlich; Transitivität überprüft man leicht).

Die Äquivalenzklassen dieser Äquivalenzrelation bezeichnen wir als \emph{Zusammenhangskomponenten von G}.

Ein Graph heißt \emph{zusammenhängend}, wenn er nur aus einer Zusammenhangskomponente besteht.
\end{Def}

\begin{Def}
\label{def:kreis}
Sei $G=(V, E)$ ein Graph.

Einen zu $C_n$ ($n \geq 3$) isomorphen Untergraph $C$ von $G$ nennen wir einen \emph{Kreis} in $G$.

Ein Graph $G$ heißt \emph{kreisfrei}, wenn er keine Kreise als Teilgraphen enthält.
\end{Def}

\begin{Bem}
Warum definieren wir einen Kreis nicht als einen geschlossenen Weg? Abgesehen vom technischen Problem, dass wir bei einer derartigen Definition eine Ausnahme von der Regel einführen müssten, dass die Knoten eines Wegs disjunkt sein müssen (da der erste und letzte Knoten eines geschlossenen Wegs identisch sind), gibt es noch ein anderes Problem: durch eine derartige Definition werden zwei Kreise als verschieden betrachtet, wenn man sie bei unterschiedlichen Knoten beginnen (und damit enden) lässt.

Für die meisten Anwendungen ist dies eine unerwünschte Eigenschaft, weswegen wir dann nicht umhin kämen, einen (offensichtlichen) Äquivalenz-Begriff für Kreise einzuführen. Dies hätte jedoch die Konsequenz zur Folge, dass wir stets überprüfen müssten, ob alle Definitionen bezüglich einer Vertreter-Auswahl von äquivalenten Kreise wohldefiniert sind.

Alle diese technischen Probleme werden durch Definition \ref{def:kreis} vermieden.
\end{Bem}

\begin{Bem}
Es scheint naheliegend, analog zu Definition \ref{def:kreis} einen Weg in $G=(V, E)$ als einen Teilgraphen zu definieren, welcher isomorph zu einem Weg-Graphen $W_n$ ($n \geq 1$) ist.

Es ist jedoch für viele Anwendungen erwünscht, dass, wenn wir die Orientierung eines Wegs ändern (also die Operation
\begin{displaymath}
\left(v_0, e_1, v_1, \ldots, v_{n-1}, e_n, v_n\right) \mapsto \left(v_n, e_n, v_{n-1}, \ldots, v_1, e_1, v_0\right)
\end{displaymath}
anwenden), die resultierenden Wege als zueinander \emph{nicht} äquivalent betrachtet werden; deshalb haben wir davon abgesehen.
\end{Bem}

\begin{Def}
Sei G ein Graph. G heißt \emph{Wald}, wenn G kreisfrei ist. Einen zusammenhängenden Wald bezeichnen wir als \emph{Baum}.
\end{Def}

\begin{Bem}
Äquivalente Kriterien dafür, dass ein Graph $G=(V, E)$ ein Wald ist, sind
\begin{itemize}
\item $G$ besteht aus genau $\left|V\right| - \left|E\right|$ Zusammenhangskomponenten
\item Für alle $e \in E$ gilt: $(V, E \backslash \left\{e\right\})$ hat mehr Zusammenhangskomponenten als $G$
\end{itemize}
Äquivalente Definitionen dafür, dass ein Graph $G=(V, E)$ ein Baum ist, sind
\begin{itemize}
\item $G$ ist minimal zusammenhängend (d. h. zusammenhängend und für alle $e\in E$ gilt: \linebreak $(V, E \backslash \left\{e\right\})$ ist nicht zusammenhängend)
\item $G$ ist maximal kreisfrei (d. h. kreisfrei und für alle $e\in {V \choose 2} \backslash E$ gilt: $(V, E \mathbin{\dot{\cup}} \left\{e\right\})$ enthält einen Kreis)
\item $G$ ist zusammenhängend und enthält $\left|V\right|-1$ Kanten
\item $G$ ist kreisfrei (Wald) und enthält $\left|V\right|-1$ Kanten
\end{itemize}
\end{Bem}

\begin{Def}
Sei $G=\left(V, E\right)$ ein Graph und sei $G'=\left(V, E'\right)$ mit $E' \subseteq E$.

Wenn $G'$ ein Wald ist, so bezeichnen wir $G'$ als \emph{Wald} in $G$.

Falls $G'$ ein Baum ist, so bezeichnen wir $G'$ als \emph{aufspannenden Baum} von $G$.

Falls $G'$ zusammenhängend ist, so bezeichnen wir $G'$ als \emph{Konnektor} von $G$.
\end{Def}

\begin{Def}
\label{def:erreichbar_weg}
Sei $G=(V, E)$ ein Graph und $v\in V$. Dann bezeichnen wir mit $R_G(v)$ die Menge der von $v$ erreichbaren Knoten aus G.
\end{Def}

\begin{Def}
\label{def:kanten_weg}
Sei $F=(V, E)$ ein Wald und $v_1, v_2 \in V$. Dann wollen wir mit $w^E\left(v_1, v_2\right)$ die Kanten des eindeutig bestimmten $v_1$-$v_2$-Weges in F (sofern er existiert) bezeichnen.
\end{Def}

\section{Geometrische Definitionen und Grundlagen}

% Kommentar entfernen, wenn affiner Unterraum verwendet wird
% Affine Unterräume, 
\subsection{Konvexe Mengen und Kegel}

% Kommentar entfernen, wenn affiner Unterraum verwendet wird
%\begin{Def}
%Sei $V$ ein $\mathbb{R}$-Vektorraum und $A \subseteq V$. Wir definieren
%\begin{displaymath}
%A\textnormal{ \emph{affiner Unterraum}} :\Leftrightarrow \forall \lambda \in \mathbb{R}, a_1, a_2 \in A: \lambda a_1 + \left(1-\lambda\right) a_2 \in A.
%\end{displaymath}
%\end{Def}

\begin{Def}
Sei $V$ ein $\mathbb{R}$-Vektorraum und $C \subseteq V$. Wir definieren
\begin{displaymath}
C\textnormal{ \emph{konvex}} :\Leftrightarrow \forall \lambda \in \left[0, 1\right], c_1, c_2 \in C: \lambda c_1 + \left(1-\lambda\right) c_2 \in C.
\end{displaymath}
\end{Def}

\begin{Def}
Sei $V$ ein $\mathbb{R}$-Vektorraum und $C \subseteq V$ mit $C \neq \emptyset$. Dann definieren wir
\begin{displaymath}
C\textnormal{ \emph{Kegel}} :\Leftrightarrow \forall \lambda_1, \lambda_2 \in \mathbb{R}_{\geq 0}, c_1, c_2 \in C: \lambda_1 c_1 + \lambda_2 c_2 \in C.
\end{displaymath}
\end{Def}

\subsection{Hüllen}

\begin{Def}
Sei $V$ ein $\mathbb{R}$-Vektorraum und $M \subseteq V$. Wir definieren
%{
%\allowdisplaybreaks
\begin{eqnarray*}
% Kommentar entfernen, wenn affiner Unterraum verwendet wird
% \aff M & := & \bigcap_{\substack{
%\widetilde{M}: M \subseteq \widetilde{M} \subseteq V: \\
%M\textnormal{ affiner Unterraum von }V}} \widetilde{M} \\
%& = & \bigcup_{n \in \mathbb{Z}_{\geq 1}} \left\{\sum_{i=1}^n \lambda_i x_i: \lambda_i \in \mathbb{R}^n \wedge \sum_{i=1}^n \lambda_i = 1 \wedge x_i\in M\textnormal{ für alle }i \in \left[n\right] \right\} \\
\conv M & := & \bigcap_{\substack{
\widetilde{M}: M \subseteq \widetilde{M} \subseteq V: \\
M\textnormal{ konvex}}} \widetilde{M} \\
& = & \bigcup_{n \in \mathbb{Z}_{\geq 1}} \left\{\sum_{i=1}^n \lambda_i x_i: \lambda_i \in \mathbb{R}_{\geq 0}^n \wedge \sum_{i=1}^n \lambda_i = 1 \wedge x_i\in M\textnormal{ für alle }i \in \left[n\right] \right\} \\
\cone M & := & \bigcap_{\substack{
\widetilde{M}: M \subseteq \widetilde{M} \subseteq V: \\
M\textnormal{ Kegel}}} \widetilde{M} \\
& = & \bigcup_{n \in \mathbb{Z}_{\geq 0}} \left\{\sum_{i=1}^n \lambda_i x_i: \lambda_i \in \mathbb{R}_{\geq 0}^n \wedge x_i\in M\textnormal{ für alle }i \in \left[n\right] \right\}
\end{eqnarray*}
%}
als 
% Kommentar entfernen, wenn affiner Unterraum verwendet wird
% affine, 
konvexe bzw. konische Hülle von $M$. Man beachte, dass in der Definition der konischen Hülle die leere Summe von Vektoren als Nullvektor definiert wird, so dass die Definition auch für $M=\emptyset$ korrekt ist.
\end{Def}

\subsection{Polyeder und Polytope}

\begin{Def}
Die Menge aller abgeschlossenen Halbräume im $\mathbb{R}^d$ wollen wir mit $\mathbb{H}_{\leq}^d$ bezeichnen und die Menge aller Hyperebenen im $\mathbb{R}^d$ mit $\mathbb{H}_=^d$.

Sei $\left\langle\cdot, \cdot\right\rangle$ ein konkretes Skalarprodukt im $\mathbb{R}^d$.

Für ein vorgegebenes $a \in \mathbb{R}^d \backslash 0^d$ und $b \in \mathbb{R}$ sei $H_{\leq}^{\left\langle\cdot, \cdot\right\rangle, a, b} := \left\{x \in \mathbb{R}^d: \left\langle a, x\right\rangle \leq b\right\} \in \mathbb{H}_{\leq}^d$ und $H_=^{\left\langle\cdot, \cdot\right\rangle, a, b} := \left\{x \in \mathbb{R}^d: \left\langle a, x\right\rangle = b\right\} \in \mathbb{H}_=^d$.
\end{Def}

\begin{Sa} (Minkowski)
Es gilt für $P \subseteq \mathbb{R}^d$:
\begin{displaymath}
\exists \underbrace{v_1, \ldots, v_{k_1}}_{=: V}, \underbrace{r_1, \ldots, r_{k_2}}_{=: R} \in \mathbb{R}^d: P = \conv V + \cone R \Leftrightarrow \exists H_1, \ldots, H_{k_3} \in \mathbb{H}_{\leq}^d: P = \bigcap_{i=1}^{k_3} H_i
\end{displaymath}
mit $k_1, k_2, k_3 \in \mathbb{Z}_{\geq 0}$. Wir wollen hierbei den leeren Schnitt auf der rechten Seite (also die Situation im Fall $k_3 = 0$) als den $\mathbb{R}^d$ definieren.

Eine Menge $P$, welche diese Voraussetzungen erfüllt, wollen wir als \emph{Polyeder} bezeichnen. Falls $\cone R = \left\{0^d\right\}$ ist (was äquivalent zu "`$P$ beschränkt"' ist), so bezeichnen wir $P$ als \emph{Polytop}.

Die linksseitige Definition bezeichnen wir als \emph{innere Darstellung} von P, die rechtsseitige als \emph{äußere Darstellung} von P.
\end{Sa}

\chapter{Polyedrische Kombinatorik}

\section{Definition des Spannbaum- und Wald- und Konnektor-Polytops}

\begin{Def}
Sei $G=\left(V, E\right)$ ein Graph. Dann definieren wir das \emph{Spannbaum-Polytop} (spanning tree polytope) $P^{G, spt} \subset \mathbb{R}^E$ über dem Graphen $G$ als
\begin{displaymath}
P^{G, spt} := \conv \left\{\chi_T: \left(V, T\right) \textnormal{ aufspannender Baum in } G\right\},
\end{displaymath}
das \emph{Wald-Polytop} (forest polytope) $P^{G, for} \subset \mathbb{R}^E$ über $G$ als
\begin{displaymath}
P^{G, for} := \conv \left\{\chi_F: \left(V, F\right) \textnormal{ Wald in } G\right\}
\end{displaymath}
und das \emph{Konnektor-Polytop} (connector polytope) $P^{G, con} \subset \mathbb{R}^E$ über G als
\begin{displaymath}
P^{G, con} := \conv \left\{\chi_C: \left(V, C\right) \textnormal{ Konnektor von } G\right\}.
\end{displaymath}
\end{Def}

Für diese Arbeit spielt insbesondere das Spannbaum-Polytop $P^{K_n, spt}$ über dem voll\-ständigen Graphen $K_n$ eine zentrale Rolle.

\section{Rechtfertigung der polyedrischen Kombinatorik}

Warum und in welchem Kontext ist es sinnvoll, ein Objekt wie das Spannbaum-, Wald- oder Konnektor-Polytop zu definieren?

Die Antwort liefert folgendes Modell. Seien $\left\{p^1, \ldots, p^k\right\} \in \mathbb{R}^d$ (beispielsweise charakteristische Vektoren von Bäumen, Wäldern, Konnektoren etc.) vorgegeben.

Wir suchen für ein beliebiges vorgegebenes $c \in \mathbb{R}^d$
\begin{displaymath}
\max_{i \in \left[k\right]}\ \left\langle c, p^i\right\rangle.
\end{displaymath}

Dazu folgender Satz:
\begin{Sa}
\label{sa:rechtfertigung}
Seien $\underbrace{\left\{p^1, \ldots, p^k\right\}}_{=: P} \in \mathbb{R}^d$ mit $k \geq 1$ vorgegeben und $\left\langle\cdot, \cdot\right\rangle$ sei ein Skalarprodukt im $\mathbb{R}^d$.

Dann gilt:
\begin{displaymath}
\conv P = \bigcup \left\{Q: P \subseteq Q \subset \mathbb{R}^d \wedge Q \textnormal{ beschränkt} \wedge \forall c \in \mathbb{R}^d\textnormal{ gilt: } \sup_{x \in Q} \left\langle c, x\right\rangle = \max_{x \in P} \left\langle c, x\right\rangle\right\}.
\end{displaymath}

Mit anderen Worten: $\conv P$ ist die inklusionsmaximale Menge unter allen Mengen $Q$, so dass für alle $c \in \mathbb{R}^d$ gilt: $\max_{x \in Q} \left\langle c, x\right\rangle = \max_{x \in P} \left\langle c, x\right\rangle$.
\end{Sa}

\begin{Bem}
Wenn wir für eine unbeschränkte Menge $M \subseteq \mathbb{R}$ definieren
\begin{displaymath}
\sup M := \infty
\end{displaymath}
und für $\emptyset \subset \mathbb{R}$ setzen:
\begin{displaymath}
\sup \emptyset := -\infty
\end{displaymath}
(also Ausweitung der Definition des Supremums auf jede beliebige Teilmenge von $\mathbb{R}$ statt auf nach oben beschränkte, nichtleere Teilmengen), so können wir Satz \ref{sa:rechtfertigung} durch Abschwächen von $k \geq 1$ zu $k\geq 0$ und Fallenlassen der Bedingung der Beschränktheit von $Q$ leicht verschärfen.
\end{Bem}

\begin{bew}
Bevor wir die Inklusionen beweisen, wollen wir noch anmerken, dass aus der Konstruktion folgt, dass wegen $P \subseteq Q$ der Wert $\sup_{x \in Q} \left\langle c, x\right\rangle = \max_{x \in P} \left\langle c, x\right\rangle$ sicherlich in einem Punkt aus $P$ angenommen wird (auch wenn es selbstverständlich weitere Punkte in $Q$ mit dieser Eigenschaft geben kann).

Zu $\subseteq$: Sei $c\in \mathbb{R}^d$ fest vorgegeben. Für jedes $x\in \conv P$ gilt (mit $\lambda \in \mathbb{R}_{\geq 0}^k$ und $\sum_{l=1}^k \lambda_l =1$):
\begin{eqnarray*}
\left\langle c, x \right\rangle & = & \left\langle c, \sum_{l=1}^k \lambda_l p^l \right\rangle \\
& = & \sum_{l=1}^k \lambda_l \left\langle c, p^l \right\rangle \\
& \leq & \sum_{l=1}^k \lambda_l \max_{m=1, \ldots, k} \left\langle c, p^m \right\rangle \\
& = & \max_{m=1, \ldots, k} \left\langle c, p^m \right\rangle.
\end{eqnarray*}

Zu $\supseteq$: Angenommen, es gäbe ein beschränktes $Q \subset \mathbb{R}^d$ mit $Q \backslash \conv P \neq \emptyset$ und für alle $c \in \mathbb{R}^d\textnormal{ gälte: } \max_{x \in Q} \left\langle c, x\right\rangle = \max_{x \in P} \left\langle c, x\right\rangle$. Sei $x^* \in Q \backslash \conv P$.

Dann existiert nach dem Trennungssatz für abgeschlossene konvexe Mengen ein  $H_=^{\left\langle\cdot, \cdot\right\rangle, a, b} \in \mathbb{H}_{=}^d$ mit
\begin{displaymath}
\forall x \in \conv P: \left\langle a, x\right\rangle < b < \left\langle a, x^*\right\rangle.
\end{displaymath}

Somit haben wir einen Widerspruch zu $\forall c \in \mathbb{R}^d\textnormal{ gilt: } \sup_{x \in Q} \left\langle c, x\right\rangle = \max_{x \in P} \left\langle c, x\right\rangle$.
\end{bew}

\section{Eine äußere Darstellung des Spannbaum-, Wald- und Konnektor-Polytops}

In \cite{schrijver2003combinatorial} und \cite{magnanti_wolsey1995} findet man Beweise für die Aussagen des folgenden Satzes, der eine äußere Darstellung des Spannbaum-, Wald- und Konnektor-Polytops liefert:
\begin{Sa}
\label{sa:darstellung_spannbaum_waldpolytop}
Sei $G=\left(V, E\right)$ ein Graph. Dann gilt:
\begin{eqnarray}
P^{G, for} =  \left\{ x \in \mathbb{R}^E: \right. & & \nonumber \\
x\left(E\left(S\right)\right) & \leq & \left|S\right|-1\ \forall S \subseteq V: \left|S\right| \geq 2 \label{eq:cycle_elim1} \\
x_e & \geq & \left. 0\ \forall e \in E \right\} \nonumber
\shortintertext{und}
P^{G, spt} =  \left\{ x \in \mathbb{R}^E: \right. & & \nonumber \\
x\left(E\left(S\right)\right) & \leq & \left|S\right|-1\ \forall S \subset V: \left|S\right| \geq 2 \label{eq:cycle_elim2} \\
x\left(E\right) & = & \left|V\right|-1 \label{eq:spt1_sum_edges} \\
x_e & \geq & \left. 0\ \forall e \in E \right\}. \nonumber
\end{eqnarray}

Sei $\Pi\left(V\right)$ die Menge aller Partitionen von $V$, d. h. Zerlegungen von $V$ in (disjunkte) nichtleere Teilmengen, und sei $\mathcal{P} \in \Pi\left(V\right)$. Dann definieren wir
\begin{displaymath}
\delta_E \left(\mathcal{P}\right) := \left\{ \left\{v_1, v_2\right\}: v_1 \in P_1, v_2 \in P_2, \left\{P_1, P_2\right\} \in {\mathcal{P} \choose 2} \right\} \cap E.
\end{displaymath}

Dann gilt:
\begin{eqnarray}
P^{G, con} =  \left\{ x \in \mathbb{R}^E: \right. & & \nonumber \\
x\left(\delta_E\left(\mathcal{P}\right)\right) & \geq & \left|\mathcal{P}\right|-1\ \forall \mathcal{P} \in \Pi\left(V\right) \label{eq:multicut1} \\
x_e & \leq & 1\ \forall e \in E \nonumber \\
x_e & \geq & \left. 0\ \forall e \in E \right\} \nonumber
\shortintertext{und}
P^{G, spt} =  \left\{ x \in \mathbb{R}^E: \right. & & \nonumber \\
x\left(\delta_E\left(\mathcal{P}\right)\right) & \geq & \left|\mathcal{P}\right|-1\ \forall \mathcal{P} \in \Pi\left(V\right) \backslash \left\{\left\{v\right\}: v \in V\right\} \label{eq:multicut2} \\
x\left(E\right) & = & \left|V\right|-1  \label{eq:spt2_sum_edges} \\
x_e & \geq & \left. 0\ \forall e \in E \right\}. \nonumber
\end{eqnarray}
\end{Sa}

\begin{Kor}
\begin{eqnarray}
{P^{G, con}}^+ =  \left\{ x \in \mathbb{R}^E: \right. & & \nonumber \\
x\left(\delta_E\left(\mathcal{P}\right)\right) & \geq & \left|\mathcal{P}\right|-1\ \forall \mathcal{P} \in \Pi\left(V\right) \\
x_e & \geq & \left. 0\ \forall e \in E \right\}. \nonumber
\end{eqnarray}
\end{Kor}

\begin{Bem}
Wir bezeichnen die Ungleichungen \eqref{eq:cycle_elim1} und \eqref{eq:cycle_elim2} als "`Cycle-Elimination-Ungleichungen"' (cycle elimination constraints).

Die Ungleichungen \eqref{eq:multicut1} und \eqref{eq:multicut2} heißen "`Multicut-Ungleichungen"'.
\end{Bem}

Der Vollständigkeit halber schreiben wir einen an \cite{magnanti_wolsey1995} angelehnten Beweis für die Äquivalenz beider Darstellungen des Spannbaum-Polytops explizit auf:

\begin{bew}
Sei $P^{G, spt, cyc}$ die Darstellung des Spannbaum-Polytops mittels der Cycle-Elimination-Ungleichungen und $P^{G, spt, mcut}$ die Darstellung unter Nutzung der Multi\-cut-Ungleichungen.

Zu $P^{G, spt, cyc} \subseteq P^{G, spt, mcut}$:

Wir müssen die Gültigkeit der Multicut-Ungleichungen zeigen.

Sei hierzu $\mathcal{P} \in \Pi\left(V\right)$. Wenn wir von \eqref{eq:spt1_sum_edges} die Summe der Cycle-Elimination-Ungleichungen
\begin{displaymath}
x\left(E\left(P\right)\right) \leq \left|P\right|-1
\end{displaymath}
über alle $P \in \mathcal{P}$ abziehen, erhalten wir
\begin{displaymath}
x\left(\delta\left(\mathcal{P}\right)\right) \geq \left|\mathcal{P}\right|-1.
\end{displaymath}

Zu $P^{G, spt, mcut} \subseteq P^{G, spt, cyc}$:

Sei $S \subset \left[n\right]: \left|S\right| \geq 2$ vorgegeben.

Setze
\begin{displaymath}
\mathcal{P} := \left\{S\right\} \mathbin{\dot{\cup}} \mathop{\dot{\bigcup}}_{v \in V \backslash S} \left\{\left\{v\right\}\right\}
\end{displaymath}

Somit gilt:
\begin{displaymath}
x\left(\delta_E\left(\mathcal{P}\right)\right) \geq n-\left|S\right|.
\end{displaymath}

Wenn wir diese Ungleichung von \eqref{eq:spt2_sum_edges} abziehen, erhalten wir
\begin{equation}
x\left(E\right) - x\left(\delta_E\left(\mathcal{P}\right)\right) \leq \left|S\right| - 1. \label{eq:pre_mcut}
\end{equation}

Durch Anwendung der Identität
\begin{eqnarray*}
x\left(E\right) - x\left(\delta_E\left(\mathcal{P}\right)\right) & = & \sum_{\left\{v_1, v_2\right\} \in E} x_{\left\{v_1, v_2\right\}} - \left(\sum_{\substack{
v_1 \in S, \\
v_2 \in \left[n\right] \backslash S: \\
\left\{v_1, v_2\right\} \in E}} x_{\left\{v_1, v_2\right\}} + \sum_{\left\{v_1, v_2\right\} \in {\left[n\right] \backslash S \choose 2} \cap E} x_{\left\{v_1, v_2\right\}}\right) \\
& = & x\left(E\left(S\right)\right)
\end{eqnarray*}
auf die linke Seite von \eqref{eq:pre_mcut} ist alles gezeigt.
\end{bew}

\begin{Bem}
\label{bem:spt_multicut_reduziert}
Häufig (vgl. \cite{magnanti_wolsey1995} oder \cite{korte2012combinatorial}) findet man in Beschreibungen des Spann\-baum-Polytops mittels Multicut-Ungleichungen noch die Ungleichung
\begin{displaymath}
x_e \leq 1 \ \forall e \in E.
\end{displaymath}

Wie wir im Beweis gesehen haben ist diese in der Beschreibung von $P^{G, spt}$ mittels Multicut-Ungleichungen redundant -- ebenso wie alle Multicut-Ungleichungen, welche sich \emph{nicht} in der Form
\begin{displaymath}
x\left(\delta_E\left(\mathcal{P}\right)\right) \geq \left|\mathcal{P}\right|-1\ \forall \mathcal{P} \textnormal{ mit } \mathcal{P} = \left\{S\right\} \mathbin{\dot{\cup}} \mathop{\dot{\bigcup}}_{v \in V \backslash S} \left\{\left\{v\right\}\right\} \textnormal{ mit } S \subset \left[n\right] \textnormal{ und } \left|S\right| \geq 2
\end{displaymath}
darstellen lassen.
\end{Bem}

\begin{Bem}
\label{bem:spt_multicut_austausch}
Die Cycle-Elimination-Ungleichungen
\begin{displaymath}
x\left(E\left(S\right)\right) \leq \left|S\right|-1\ \forall S \subset \left[n\right]: \left|S\right| \geq 2
\end{displaymath}
stehen in 1-zu-1-Korrespondenz zu den Multicut-Ungleichungen 
\begin{displaymath}
x\left(\delta_E\left(\mathcal{P}\right)\right) \geq \left|\mathcal{P}\right|-1\ \forall \mathcal{P} \textnormal{ mit } \mathcal{P} = \left\{S\right\} \mathbin{\dot{\cup}} \mathop{\dot{\bigcup}}_{v \in V \backslash S} \left\{\left\{v\right\}\right\} \textnormal{ mit } S \subset \left[n\right] \textnormal{ und } \left|S\right| \geq 2,
\end{displaymath}
d. h. jede Cycle-Elimination-Ungleichung in $P^{G, spt}$ kann gegen die analoge Multicut-Ungleichung ausgetauscht werden, ohne dass sich das Polytop ändert.

Mit der reduzierten Beschreibung von $P^{G, spt}$, welche sich aus Bemerkung \ref{bem:spt_multicut_reduziert} ergibt, gilt des Weiteren, dass wenn wir aus dieser einige der Multicut-Ungleichungen entfernen, das Ergebnis dem entspricht, wie wenn wir in der Beschreibung von $P^{G, spt}$ mittels Cycle-Elimination-Ungleichungen die analogen Cycle-Elimination-Unglei\-ch\-un\-gen entfernen.
\end{Bem}

\section{Ecken des Spannbaum-Polytops}

In \cite{matousek2007diskrete} und \cite{lovasz_combinatorics} wird folgender Satz bewiesen:
\begin{Sa}
(Cayley-Formel)
\begin{displaymath}
\left|\left\{T \subseteq {\left[n\right] \choose 2}: T\textnormal{ aufspannender Baum in }K_n\right\}\right| = n^{n-2}.
\end{displaymath}
\end{Sa}

Zwar werden wir erst in Abschnitt \ref{sec:ecken} eine ausführlichere Darstellung von Ecken von Polyedern bringen, dennoch sei schon hier die Definition einer Ecke einer konvexen Menge gegeben:

\begin{Def}
\label{def:ecke}
Sei $C$ konvex und sei $c \in C$. Wir definieren
\begin{displaymath}
c \textnormal{ \emph{Ecke} von }C :\Leftrightarrow \not\exists c_1, c_2 \in C: c_1, c_2 \neq c \wedge c \in \conv \left(\left\{c_1, c_2\right\}\right).
\end{displaymath}

Die Menge aller Ecken von $C$ wollen wir mit $\vertices \left(C\right)$ bezeichnen.
\end{Def}

$P^{K_n, spt}$ ist die konvexe Hülle charakteristischer Vektoren aufspannender Bäume (und damit können höchstens diese potentielle Ecken bilden). Umgekehrt ist jeder dieser charakteristischen Vektoren eines Baums 0/1-wertig. Somit kann ein solcher nicht als Konvexkombination charakteristischer Vektoren anderer aufspannender Bäume dargestellt werden. Also ist jeder charakteristische Vektor eines aufspannenden Baums eine Ecke von $P^{K_n, spt}$. Daraus folgt:
\begin{Kor}
\begin{displaymath}
\left|\vertices \left(P^{K_n, spt}\right)\right| = n^{n-2}.
\end{displaymath}
\end{Kor}

Es ist anzumerken, dass die Cayley-Formel einen Spezialfall von folgendem allgemeineren Resultat darstellt, für welches man in \citep{matousek2007diskrete} und \cite{lovasz_combinatorics} einen Beweis finden kann:
\begin{Sa}
\label{sa:allg_cayley_formel}
Sei $G=\left(V, E\right)$ mit $V = \left[n\right]$ und $n \geq 2$. Setze $Q\in \mathbb{R}^{n \times n}$ (die sogenannte Laplace-Matrix von G) mittels
\begin{eqnarray*}
q_{i i} & := & \deg_G\left(i\right)\ \forall i \in \left[n\right] \\
q_{i j} & := & \begin{cases}
-1 & \left\{i, j\right\} \in E \\
0 & \left\{i, j\right\} \notin E
\end{cases}\ \forall \left(i, j\right) \in \left[n\right]^{\underline{2}}.
\end{eqnarray*}
Sei außerdem $Q_{i j}$ die Determinante der Matrix, welche man aus $Q$ erhält, indem man Zeile $i$ und Spalte $j$ streicht.

Dann gilt für $i, j \in \left[n\right]$:
\begin{displaymath}
\left|\left\{T \subseteq E: \left(V, T\right)\textnormal{ aufspannender Baum in }G\right\}\right| = \left(-1\right)^{i+j} Q_{i j}.
\end{displaymath}
\end{Sa}

Mit dem selben Argument wie für den vollständigen Graph erhält man somit
\begin{Kor}
Seien $G$, $Q$, $Q_{i j}$ wie in Satz \ref{sa:allg_cayley_formel}. Dann gilt für $i, j \in \left[n\right]$:
\begin{displaymath}
\left|\vertices \left(P^{G, spt}\right)\right| = \left(-1\right)^{i+j} Q_{i j}.
\end{displaymath}
\end{Kor}

\begin{Bem}
Man kann Satz \ref{sa:allg_cayley_formel} auch auf Multi-Graphen verallgemeinern (vgl. \citep{matousek2007diskrete}). In diesem Fall setzt man für die Matrix $Q$:
\begin{eqnarray*}
q_{i i} & := & \deg_G\left(i\right)\ \forall i \in \left[n\right] \\
q_{i j} & := & \begin{cases}
-m & \left\{i, j\right\} \in E \\
0 & \left\{i, j\right\} \notin E
\end{cases}\ \forall \left(i, j\right) \in \left[n\right]^{\underline{2}},
\end{eqnarray*}
wobei bei $\deg_G$ keinerlei Schleifen mitgezählt werden, allerdings Mehrfach-Kanten entsprechend ihrer Vielfachheit mehrfach. $m$ stehe für die Anzahl Kanten zwischen $i$ und $j$.
\end{Bem}

\section{Erweiterte Formulierungen}

Wie wir gesehen haben, wird das Spannbaum-, Wald- bzw. Konnektor-Polytop durch exponentiell in $n$ viele Ungleichungen beschrieben. Leider sind im Allgemeinen $2^{\Theta\left(n\right)}$ viele Ungleichungen aus der Darstellung in Satz \ref{sa:darstellung_spannbaum_waldpolytop} irredundant.

Um das einzusehen, betrachten wir folgenden Satz:

\begin{Sa}
Eine Ungleichung \eqref{eq:cycle_elim2} des Spannbaum-Polytops $P^{G, spt}$ ist genau dann facetten-induzierend, wenn $S$ einen zweifach zusammenhängenden Subgraph in $G$ induziert.
\end{Sa}

Dieser wird in \citep[S:~862]{schrijver2003combinatorial} beschrieben. Seine Gültigkeit folgt unmittelbar aus Satz 40.5 desselben Buchs.

Demnach sind $\Theta\left(2^n\right)$ Ungleichungen aus \eqref{eq:cycle_elim2} in der Beschreibung von $P^{K_n, spt}$ irredundant.

Da es Algorithmen gibt, die für ein vorgegebenes $c \in \mathbb{R}^{n \choose 2}$ in polynomiell beschränkter Zeit einen aufspannenden Baum, Wald bzw. Konnektor von $K_n$ finden, dessen charakteristischer Vektor $x$ bezüglich $\left\langle c, x\right\rangle$ minimiert/maximiert wird (beispielsweise Algorithmen von Kruskal oder Prim; vgl. \cite{korte2012combinatorial} und \cite{schrijver2003combinatorial}) und somit die Ungleichungen in polynomiell beschränkter Zeit separierbar sind, ist dies für die meisten praktischen Anwendungen kein Problem.\footnote{Ein -- zugegebenermaßen eher theoretisches -- Beispiel, wo dies jedoch problematisch wird, ist das Problem "`Gegeben sei ein Punkt im $\mathbb{R}^{\left[n\right] \choose 2}$; liste alle Ungleichungen des Spannbaum-, Wald- bzw. Konnektor-Polytops auf, die von diesem Punkt \emph{erfüllt} werden."', da hier die Output-Länge exponentiell in n groß werden kann.}

Dennoch wäre es sinnvoll, eine Beschreibung des Spannbaum-, Wald- bzw. Konnek\-tor-Polytops zu besitzen, welche mit polynomiell in $n$ beschränkt vielen Ungleichungen auskäme. Um dies zu erreichen, werden wir im Folgenden mit sogenannten \emph{erweiterten Formulierungen} arbeiten.

Eine in \cite{springerlink:ext_form_comb_opt} gegebene Beschreibung führt zu folgender Definition:
\begin{Def}
Sei
\begin{displaymath}
X = \left\{x \in \mathbb{R}^N: A' x \leq b', x_i \textnormal{ ganzzahlig für } i \in I \subseteq \left[n\right]\right\}
\end{displaymath}
und sei
\begin{displaymath}
Q=\left\{\left(\begin{array}{c}
x \\
z
\end{array}\right) \in \mathbb{R}^n \times \mathbb{R}^p: A x + B z \leq b\right\}.
\end{displaymath}
Dann heißt Q eine \emph{erweiterte Formulierung von $X$}, wenn gilt:
\begin{displaymath}
\proj_x\left(Q\right) = \conv \left(X\right).
\end{displaymath}

Diese heißt \emph{kompakt}, wenn die Größe der Matrix $\left(A\,|\,B\,|\,b\right)$ polynomiell beschränkt in der Größe der Matrix $\left(A'\,|\,b'\right)$ ist.
\end{Def}

Die Formulierung in \cite{optima:ext_form_comb_opt} führt auf folgende Definition:

\begin{Def}
Eine \emph{Erweiterung} eines Polytops $P \subset \mathbb{R}^n$ ist ein Polyeder $Q \subseteq \mathbb{R}^d$ zusammen mit einer linearen Projektion $p: \mathbb{R}^d \rightarrow \mathbb{R}^n$mit $p\left(Q\right) = P$.

Ein Paar, bestehend aus einer Beschreibung von $Q$ mittels Gleichungen und Ungleichungen und $p$ bezeichnen wir als \emph{erweiterte Formulierung} von $P$. Ihre \emph{Größe} ist die Anzahl der Ungleichungen in der Beschreibung.
\end{Def}

Es ist jedoch anzumerken, dass es für die Zwecke dieser Arbeit keinen Unterschied macht, auf welche Definition wir setzen, da das Spannbaum-, Wald- und Konnektor-Polytop (wie der Name schon sagt) jeweils Polytope sind und als Abbildung $p$ stets die Projektion in den Raum der $x$-Variablen verwendet wird.

\part{Erweiterte Formulierungen des Spannbaum-Polytops und verwandter Polytope}

\label{part:spannbaum_polytop}

\chapter{Die erweiterte Formulierung $P^{K_n, spt, symm, 0}$ des Spannbaum-Polytops}

\section{Vereinfachung einer erweiterten Formulierung des Spannbaum-Polytops}

In \cite{optima:ext_form_comb_opt} wird für $P^{K_n, spt}$ folgende erweiterte Formulierung angegeben, welche die Grundlage für die vorliegende Arbeit liefert:
\begin{eqnarray}
P^{K_n, spt, symm, 0} := \left\{(x, z)\in\mathbb{R}^{[n]\choose 2}\times \mathbb{R}^{[n]^{\underline{3}}}:\right. & & \nonumber \\
x\left({\left[n\right] \choose 2}\right) & = & n-1 \label{eq:spt1_redundant_1} \\
x_{\left\{v, w\right\}} - z_{\left(v, w\right), u} - z_{\left(w, v\right), u} & = & 0\ \forall \left(v, w, u\right)\in [n]^{\underline{3}} \label{eq:spt1_redundant_2} \\
x_{\left\{v, w\right\}} + \sum_{u\in[n]\backslash\left\{v, w\right\}} z_{\left(v, u\right), w} & = & 1\ \forall \left(v, w\right)\in [n]^{\underline{2}} \label{eq:spt1_3} \\
x_{\left\{v, w\right\}} & \geq & 0\ \forall \left\{v, w\right\}\in {[n] \choose 2} \nonumber \\
z_{\left(v, w\right), u} & \geq & \left. 0\ \forall \left(v, w, u\right)\in [n]^{\underline{3}} \nonumber \right\}.
\end{eqnarray}

Diese erweiterte Formulierung geht ursprünglich auf Martin zurück (\cite{Martin_extended_formulations}). Dass es sich hierbei tatsächlich um eine erweiterte Formulierung des Spannbaum-Polytops handelt, werden wir weiter unten beweisen.

Bevor wir diese erweiterte Formulierung näher betrachten, wollen wir noch anmerken, dass man in \cite{springerlink:ext_form_comb_opt} eine analoge erweiterte Formulierung für $P^{G, spt}$, wobei $G$ ein beliebiger Graph sei, finden kann.

Betrachten wir nun $P^{K_n, spt, symm, 0}$ näher: Offensichtlich ist jede Gleichung in \eqref{eq:spt1_redundant_2} symmetrisch in $v$ und $w$. Damit kann die Bedingung "`$\forall \left(v, w, u\right)\in [n]^{\underline{3}}$"' in der Gleichungs-Serie \eqref{eq:spt1_redundant_2} zu "`$\forall \left(\left\{v, w\right\}, u\right)\in EV_n$"' abgeschwächt werden, um redundante Gleichungen zu entfernen. Des Weiteren ist Gleichung \eqref{eq:spt1_redundant_1} redundant, denn
{
\allowdisplaybreaks
\begin{eqnarray*}
x\left({\left[n\right] \choose 2}\right) & = & \sum_{\left\{u, v\right\} \in {\left[n\right] \choose 2}} x_{\left\{u, v\right\}} \\
& = & \frac{1}{n} \left(\frac{1}{2} \sum_{\left(u, v, w\right) \in \left[n\right]^{\underline{3}}} x_{\left\{u, v\right\}} + \sum_{\left(u, v\right) \in \left[n\right]^{\underline{2}}} x_{\left\{u, v\right\}} \right) \\
& \stackrel{\eqref{eq:spt1_redundant_2}}{=} & \frac{1}{n} \left(\frac{1}{2} \sum_{\left(u, v, w\right) \in \left[n\right]^{\underline{3}}} \left(z_{\left(u, v\right), w} + z_{\left(v, u\right), w}\right) + \sum_{\left(u, v\right) \in \left[n\right]^{\underline{2}}} x_{\left\{u, v\right\}} \right) \\
& = & \frac{1}{n} \left(\sum_{\left(u, v, w\right) \in \left[n\right]^{\underline{3}}} z_{\left(u, v\right), w} + \sum_{\left(u, v\right) \in \left[n\right]^{\underline{2}}} x_{\left\{u, v\right\}} \right) \\
& = & \frac{1}{n} \sum_{\left(u, v\right) \in \left[n\right]^{\underline{2}}} \left( x_{\left\{u, v\right\}} + \sum_{w \in \left[n\right] \backslash \left\{u, v\right\}} z_{\left(u, w\right), v} 
\right) \\
& \stackrel{\eqref{eq:spt1_3}}{=} & \frac{1}{n} \sum_{\left(u, v\right) \in \left[n\right]^{\underline{2}}} 1 \\
& = & n-1.
\end{eqnarray*}
}

Dies bildet den Beweis des folgenden Lemmas:

\begin{Le}
\label{le:spt_erweitert_2}
$P^{K_n, spt, symm, 0}$ wird äquivalent durch folgende Gleichungen und Ungleichungen beschrieben:
\begin{eqnarray}
\left\{(x, z)\in\mathbb{R}^{[n]\choose 2}\times \mathbb{R}^{[n]^{\underline{3}}}:\right. & & \nonumber \\
x_{\left\{v, w\right\}} - z_{\left(v, w\right), u} - z_{\left(w, v\right), u} & = & 0\ \forall \left(\left\{v, w\right\}, u\right)\in EV_n \label{eq:spt_erweitert_2_1} 
\\
x_{\left\{v, w\right\}} + \sum_{u\in[n]\backslash\left\{v, w\right\}} z_{\left(v, u\right), w} & = & 1\ \forall \left(v, w\right)\in [n]^{\underline{2}} \label{eq:spt_erweitert_2_2} 
\\
x_{\left\{v, w\right\}} & \geq & 0\ \forall \left\{v, w\right\}\in {[n] \choose 2} \nonumber \\
z_{\left(v, w\right), u} & \geq & \left.0\ \forall \left(v, w, u\right)\in [n]^{\underline{3}} \nonumber \right\}.
\end{eqnarray}
\end{Le}

\begin{Bem}
Im Fall $n=2$ tritt Gleichung \eqref{eq:spt_erweitert_2_2} doppelt auf -- aber das soll hier keine Rolle spielen.
\end{Bem}

Als nächstes fällt auf, dass $x\geq 0$ nicht erforderlich ist. Dies sieht man für $n\geq 3$ ein, wenn man für $\left(\left\{v, w\right\}, u\right)\in EV_n$ in \eqref{eq:spt_erweitert_2_1} die Voraussetzung $z\geq 0$ einsetzt (für $n=2$ folgt direkt aus Gleichung \eqref{eq:spt_erweitert_2_2} $x_{\left\{1, 2\right\}} = 1$).

\begin{Le}
\label{le:spt_erweitert_dual2}
Es gilt:
\begin{align}
P^{K_n, spt, symm, 0} & = \left\{
\left(\begin{array}{c}
x \\
z
\end{array}\right)\in\mathbb{R}^{[n]\choose 2}\times \mathbb{R}^{[n]^{\underline{3}}}: \left(\begin{array}{c}
x \\
z
\end{array}\right)\textnormal{ erfüllt }\mathcal{I}^{n, symm, 0}_2\right\} \nonumber \\
\shortintertext{mit}
\mathcal{I}^{n, symm, 0}_2 & := \left\{ \right. \nonumber \\
x_{\left\{v_1, v_2\right\}} - z_{\left(v_1, v_2\right), u} - z_{\left(v_2, v_1\right), u} & = 0\ \forall \left(\left\{v_1, v_2\right\}, u\right)\in EV_n \label{eq:spt_erweitert_primal2_1} \\
x_{\left\{v_1, v_2\right\}} + \sum_{u\in[n]\backslash\left\{v_1, v_2\right\}} z_{\left(v_1, u\right), v_2} & = 1\ \forall \left(v_1, v_2\right)\in [n]^{\underline{2}} \label{eq:spt_erweitert_primal2_2} \\
z_{\left(v_1, v_2\right), v_3} & \geq \left.0\ \forall \left(v_1, v_2, v_3\right)\in [n]^{\underline{3}} \right\}. \label{eq:spt_erweitert_z}
\end{align}
\end{Le}

\begin{Bem}
\label{bem:spt_erweitert_dual2}
In der Formulierung von $P^{K_n, spt, symm, 0}$ durch $\mathcal{I}^{n, symm, 0}_2$ wird dieses Polytop durch
\begin{itemize}
\item $\frac{1}{2} n \left(n-1\right) \left(n-2\right)$ Gleichungen \eqref{eq:spt_erweitert_primal2_1}
\item $n \left(n-1\right)$ Gleichungen \eqref{eq:spt_erweitert_primal2_2}
\item $n \left(n-1\right) \left(n-2\right)$ Nichtnegativitätsbedingungen für $z$-Variablen (Ungleichungen \eqref{eq:spt_erweitert_z})
\end{itemize}
in
\begin{itemize}
\item $\frac{1}{2} n \left(n-1\right)$ $x$-Variablen
\item $n \left(n-1\right) \left(n-2\right)$ $z$-Variablen
\end{itemize}
beschrieben.
\end{Bem}

\section{Die Projektion von $P^{K_n, spt, symm, 0}$}

\begin{Sa}\label{sa:proj_spt_erw}
\begin{displaymath}
\proj_x P^{K_n, spt, symm, 0} = P^{K_n, spt}
\end{displaymath}
\end{Sa}
\begin{bew}

Sei T die Kantenmenge eines aufspannenden Baum in $K_n$.

Setze $x:=\chi\left(T\right)$ und $z\in \mathbb{R}^{[n]^{\underline{3}}}$ mittels
\begin{displaymath}
z_{\left(v, w\right), u} := \begin{cases}
1 & \left\{v, w\right\}\in T\wedge u\in R_{T\backslash \left\{\left\{v, w\right\}\right\}}\left(w\right) \\
0 & \textnormal{sonst}
\end{cases}
\end{displaymath}

Wir überprüfen, ob die Gleichungen \eqref{eq:spt_erweitert_2_1} und \eqref{eq:spt_erweitert_2_2} erfüllt sind:

\paragraph{Zu \eqref{eq:spt_erweitert_2_1}:} Wir müssen für vorgegebene $\left(\left\{v, w\right\}, u\right) \in EV_n$ die Gültigkeit von \eqref{eq:spt_erweitert_2_1} zeigen.

\subparagraph{Fall 1: $\left\{v, w\right\}\notin T$} In diesem Fall gilt $x_{\left\{v, w\right\}}=0$. Nach Definition von $z_{(v, w), u}$ gilt: $z_{(v, w), u}=z_{(w, v), u}=0$ und somit ist \eqref{eq:spt_erweitert_2_1} sicher erfüllt.

\subparagraph{Fall 2: $\left\{v, w\right\}\in T$} Dann ist $x_{\left\{v, w\right\}} = 1$. $u$ befindet sich in $T\backslash \left\{\left\{v, w\right\}\right\}$ entweder in der Zusammenhangskomponente von $v$ oder $w$ (und zwar genau einer). Somit ist entweder $z_{(v, w), u}=0$ und $z_{(w, v), u}=1$ oder umgekehrt. In jedem Fall ist \eqref{eq:spt_erweitert_2_1} erfüllt.

\paragraph{Zu \eqref{eq:spt_erweitert_2_2}:} Wir müssen für vorgegebene $\left(v, w\right) \in \left[n\right]^{\underline{2}}$ die Gültigkeit von \eqref{eq:spt_erweitert_2_2} zeigen.

\subparagraph{Fall 1: $\left\{v, w\right\}\notin T$} Dann ist $x_{\left\{v, w\right\}}=0$. Außerdem gibt es (wegen des Zusammenhangs von $T$ und der Eigenschaft eines Baums, dass der Weg zwischen zwei Knoten eindeutig bestimmt ist) zu $v$ genau ein $u$, so dass $\left\{v, u\right\}\in T$ ist und $w$ mit $u$ in einer Zusammenhangskomponente von $T\backslash \left\{\left\{v, u\right\}\right\}$ liegt, weswegen die Gleichung erfüllt ist.

\subparagraph{Fall 2: $\left\{v, w\right\}\in T$} In diesem Fall haben wir $x_{\left\{v, w\right\}}=1$. Weil $T$ ein Baum ist, ist jeder Weg eindeutig bestimmt. Allerdings bildet bereits $\left(v, \left\{v, w\right\}, w\right)$ einen $v$-$w$-Weg. Somit gilt für alle $u\in \left[n\right]\backslash \left\{v, w\right\}$: $z_{(v, u), w}=0$.
\\

Sei umgekehrt $\left(
\begin{array}{c}
x \\
z
\end{array}\right) \in P^{K_n, spt, symm, 0}$; es ist zu zeigen, dass seine $x$-Komponenten die Gleichungen und Ungleichungen des Spannbaum-Polytops erfüllen.

Die Bedingung $x_{\left\{v, w\right\}} \geq 0 \ \forall \left\{v, w\right\} \in {\left[n\right] \choose 2}$ folgt für $n\geq 3$ sofort aus \eqref{eq:spt_erweitert_2_1} unter Nutzung von $z_{\left(v, w\right), u} \geq 0 \ \forall \left(v, w, u\right) \in \left[n\right]^{\underline{3}}$ (für $n=2$ leitet man einfach $x_{\left\{1, 2\right\}} = 1$ her).

Sei für die restlichen Ungleichungen $S\subseteq \left[n\right]$.
{
\allowdisplaybreaks
\begin{eqnarray*}
x\left(E\left(S\right)\right) & = & \sum_{\left\{u, v\right\} \in {S \choose 2}} x_{\left\{u, v\right\}} \\
& = & \frac{1}{\left|S\right|} \left(\frac{1}{2} \sum_{\left(u, v, w\right) \in S^{\underline{3}}} x_{\left\{u, v\right\}} + \sum_{\left(u, v\right) \in S^{\underline{2}}} x_{\left\{u, v\right\}} \right) \\
& \stackrel{\eqref{eq:spt_erweitert_primal2_1}}{=} & \frac{1}{\left|S\right|} \left(\frac{1}{2} \sum_{\left(u, v, w\right) \in S^{\underline{3}}} \left(z_{\left(u, v\right), w} + z_{\left(v, u\right), w}\right) + \sum_{\left(u, v\right) \in S^{\underline{2}}} x_{\left\{u, v\right\}} \right) \\
& = & \frac{1}{\left|S\right|} \left(\sum_{\left(u, v, w\right) \in S^{\underline{3}}} z_{\left(u, v\right), w} + \sum_{\left(u, v\right) \in S^{\underline{2}}} x_{\left\{u, v\right\}} \right) \\
& \stackrel{z \geq 0}{\leq} & \frac{1}{\left|S\right|} \sum_{\left(u, v\right) \in S^{\underline{2}}} \left( x_{\left\{u, v\right\}} + \sum_{w \in \left[n\right] \backslash \left\{u, v\right\}} z_{\left(u, w\right), v} 
\right) \\
& \stackrel{\eqref{eq:spt_erweitert_primal2_2}}{=} & \frac{1}{\left|S\right|} \sum_{\left(u, v\right) \in S^{\underline{2}}} 1 \\
& = & \left|S\right|-1.
\end{eqnarray*}
}
Die Bedingung $x\left({\left[n\right] \choose 2}\right) = n-1$ folgt offensichtlich aus dieser Ungleichungskette unter Beachtung, dass für $S=\left[n\right]$ die Bedingung $z_{\left(v, w\right), u} \geq 0 \ \forall \left(v, w, u\right) \in \left[n\right]^{\underline{3}}$ nicht benutzt wurde oder (dort direkt aufgeschrieben) aus dem Beweis von Lemma \ref{le:spt_erweitert_dual2}.
\end{bew}

\begin{Bem}
Im Beweis von Satz \ref{sa:proj_spt_erw} haben wir für die z-Variable der Einbettung gesetzt:
\begin{displaymath}
z_{\left(v, w\right), u} := \begin{cases}
1 & \left\{v, w\right\}\in T\wedge u\in R_{T\backslash \left\{\left\{v, w\right\}\right\}}\left(w\right) \\
0 & \textnormal{sonst,}
\end{cases}
\end{displaymath}
wobei $T$ die Kantenmenge eines aufspannenden Baums ist.

Diese Bedingung "`$\left\{v, w\right\}\in T \wedge u\in R_{T\backslash \left\{\left\{v, w\right\}\right\}}\left(w\right)$"' (mit $\left(v, w, u\right) \in \left[n\right]^{\underline{3}}$) ist äquivalent zu
\begin{displaymath}
\left\{v, u\right\}\notin T\wedge\left(\left[n\right], T \mathbin{\dot{\cup}} \left\{v, u\right\}\right)\textnormal{ enthält einen Kreis, der die Kante }\left\{v, w\right\}\textnormal{ enthält}.
\end{displaymath} 
\end{Bem}

\chapter{Neue symmetrische erweiterte Formulierungen}

\section{$P^{K_n, spt, symm, pre-xy}$}

Wenn wir uns den Beweis von Satz \ref{sa:proj_spt_erw} genau anschauen, so stellen wir folgendes fest:
\begin{itemize}
\item Im Schritt
\begin{displaymath}
\frac{1}{\left|S\right|} \left(\frac{1}{2} \sum_{\left(u, v, w\right) \in S^{\underline{3}}} x_{\left\{u, v\right\}} + \sum_{\left(u, v\right) \in S^{\underline{2}}} x_{\left\{u, v\right\}} \right) 
\stackrel{\eqref{eq:spt_erweitert_2_1}}{=} \frac{1}{\left|S\right|} \left(\frac{1}{2} \sum_{\left(u, v, w\right) \in S^{\underline{3}}} \left(z_{\left(u, v\right), w} + z_{\left(v, u\right), w}\right) + \sum_{\left(u, v\right) \in S^{\underline{2}}} x_{\left\{u, v\right\}} \right)
\end{displaymath}
wurde die Bedingung \eqref{eq:spt_erweitert_2_1}
\begin{displaymath}
x_{\left\{v_1, v_2\right\}} - z_{\left(v_1, v_2\right), u} - z_{\left(v_2, v_1\right), u} = 0\ \forall \left(\left\{v_1, v_2\right\}, u\right)\in EV_n
\end{displaymath}
nicht direkt benutzt. Stattdessen reicht folgende schwächere Bedingung
\begin{eqnarray}
x_{\left\{v_1, v_2\right\}} + x_{\left\{v_2, v_3\right\}} + x_{\left\{v_3, v_1\right\}} 
- z_{\left(v_1, v_2\right), v_3} - z_{\left(v_2, v_1\right), v_3} & & \nonumber \\
- z_{\left(v_2, v_3\right), v_1} - z_{\left(v_3, v_2\right), v_1} 
- z_{\left(v_3, v_1\right), v_2} - z_{\left(v_1, v_3\right), v_2}
& = & 0\ \forall \left\{v_1, v_2, v_3\right\}\in {\left[n\right] \choose 3}, \label{eq:erw_symm_eq_neu1}
\end{eqnarray}
welche man offenbar durch Addition der durch $\left(\left\{v_1, v_2\right\}, v_3\right)$, $\left(\left\{v_2, v_3\right\}, v_1\right)$, $\left(\left\{v_3, v_1\right\}, v_2\right)$ indizierten Gleichungen \eqref{eq:spt_erweitert_2_1} erhält, aus:
\begin{eqnarray*}
\frac{1}{\left|S\right|} \left(\frac{1}{2} \sum_{\left(u, v, w\right) \in S^{\underline{3}}} x_{\left\{u, v\right\}} + \sum_{\left(u, v\right) \in S^{\underline{2}}} x_{\left\{u, v\right\}} \right) 
& = & \frac{1}{\left|S\right|} \left(\frac{1}{2} \cdot \frac{1}{3} \sum_{\left(u, v, w\right) \in S^{\underline{3}}} \left(x_{\left\{u, v\right\}} + x_{\left\{u, w\right\}} + x_{\left\{v, w\right\}}\right) \right. \\
& & \left. + \sum_{\left(u, v\right) \in S^{\underline{2}}} x_{\left\{u, v\right\}} \right) \\
& \stackrel{\eqref{eq:erw_symm_eq_neu1}}{=} & \frac{1}{\left|S\right|} \left(\frac{1}{2} \cdot \frac{1}{3} \sum_{\left(u, v, w\right) \in S^{\underline{3}}} \left(z_{\left(u, v\right), w} + z_{\left(v, u\right), w}\right.  \right. \\
& & \left. + z_{\left(v, w\right), u} + z_{\left(w, v\right), u} + z_{\left(w, u\right), v} + z_{\left(u, w\right), v} \right) \\
& & \left. + \sum_{\left(u, v\right) \in S^{\underline{2}}} x_{\left\{u, v\right\}} \right) \\
& = &  \frac{1}{\left|S\right|} \left(\frac{1}{2} \sum_{\left(u, v, w\right) \in S^{\underline{3}}} \left(z_{\left(u, v\right), w} + z_{\left(v, u\right), w}\right) \right. \\
& & \left.+ \sum_{\left(u, v\right) \in S^{\underline{2}}} x_{\left\{u, v\right\}} \right).
\end{eqnarray*}
\item Im Schritt
\begin{displaymath}
\frac{1}{\left|S\right|} \left(\sum_{\left(u, v, w\right) \in S^{\underline{3}}} z_{\left(u, v\right), w} + \sum_{\left(u, v\right) \in S^{\underline{2}}} x_{\left\{u, v\right\}} \right) \\
\stackrel{z \geq 0}{\leq} \frac{1}{\left|S\right|} \sum_{\left(u, v\right) \in S^{\underline{2}}} \left( x_{\left\{u, v\right\}} + \sum_{w \in \left[n\right] \backslash \left\{u, v\right\}} z_{\left(u, w\right), v} 
\right)
\end{displaymath}
wurde die Bedingung \eqref{eq:spt_erweitert_z}
\begin{displaymath}
z_{\left(v_1, v_2\right), v_3} \geq 0\ \forall \left(v_1, v_2, v_3\right)\in [n]^{\underline{3}}
\end{displaymath}
nicht direkt verwendet, sondern die schwächere Bedingung
\begin{equation}
z_{\left(v_1, w\right), v_2} + z_{\left(v_2, w\right), v_1} \geq 0\ \forall \left(\left\{v_1, v_2\right\}, w\right)\in EV_n \label{eq:erw_symm_eq_z}
\end{equation}
reicht aus.

Der entsprechende Schritt im Beweis sieht dann folgendermaßen aus:
\begin{eqnarray*}
\frac{1}{\left|S\right|} \left(\sum_{\left(u, v, w\right) \in S^{\underline{3}}} z_{\left(u, v\right), w} + \sum_{\left(u, v\right) \in S^{\underline{2}}} x_{\left\{u, v\right\}} \right) & = & \frac{1}{\left|S\right|} \left(
\sum_{v \in S} \sum_{\left(u, w\right) \in \left(S \backslash \left\{v\right\}\right)^{\underline{2}}} z_{\left(u, v\right), w} + \sum_{\left(u, v\right) \in S^{\underline{2}}} x_{\left\{u, v\right\}} \right) \\
& = & \frac{1}{\left|S\right|} \left(
\sum_{v \in S} \sum_{\left\{u, w\right\} \in {S \backslash \left\{v\right\} \choose 2}} \left(z_{\left(u, v\right), w} + z_{\left(w, v\right), u}\right) \right. \\
& & \left. + \sum_{\left(u, v\right) \in S^{\underline{2}}} x_{\left\{u, v\right\}} \right) \\
& \stackrel{\eqref{eq:erw_symm_eq_z}}{\leq} & \frac{1}{\left|S\right|} \left(
\sum_{v \in \left[n\right]} \sum_{\left\{u, w\right\} \in {S \backslash \left\{v\right\} \choose 2}} \left(z_{\left(u, v\right), w} + z_{\left(w, v\right), u}\right) \right. \\
& & \left. + \sum_{\left(u, v\right) \in S^{\underline{2}}} x_{\left\{u, v\right\}} \right) \\
& = & \frac{1}{\left|S\right|} \sum_{\left(u, v\right) \in S^{\underline{2}}} \left( x_{\left\{u, v\right\}} + \sum_{w \in \left[n\right] \backslash \left\{u, v\right\}} z_{\left(u, w\right), v} 
\right).
\end{eqnarray*}
\item Im Schritt
\begin{displaymath}
\frac{1}{\left|S\right|} \left(\sum_{\left(u, v\right) \in S^{\underline{2}}} \left( x_{\left\{u, v\right\}} + \sum_{w \in \left[n\right] \backslash \left\{u, v\right\}} z_{\left(u, w\right), v} 
\right)\right)
\stackrel{\eqref{eq:spt_erweitert_2_2}}{=}
\frac{1}{\left|S\right|} \sum_{\left(u, v\right) \in S^{\underline{2}}} 1
\end{displaymath}
wird die Gleichung \eqref{eq:spt_erweitert_2_2}
\begin{displaymath}
x_{\left\{v_2, v_2\right\}} + \sum_{u\in[n]\backslash\left\{v_1, v_2\right\}} z_{\left(v_1, u\right), v_2} = 1\ \forall \left(v_1, v_2\right)\in [n]^{\underline{2}}
\end{displaymath}
nicht direkt verwendet, sondern die schwächere Gleichung
\begin{equation}
2 x_{\left\{v_1, v_2\right\}} + \sum_{u\in[n]\backslash\left\{v_1, v_2\right\}} \left(z_{\left(v_1, u\right), v_2} + z_{\left(v_2, u\right), v_1} \right) = 2\ \forall \left\{v_1, v_2\right\}\in {[n] \choose 2}, \label{eq:erw_symm_eq_neu2}
\end{equation}
welche man durch Addition der durch $\left(v, w\right)$ und $\left(w, v\right)$ indizierten Gleichungen \eqref{eq:spt_erweitert_2_2} erhält, reicht aus.

Der entsprechende Schritt sieht dann folgendermaßen aus:
\begin{eqnarray*}
\frac{1}{\left|S\right|} \sum_{\left(u, v\right) \in S^{\underline{2}}} \left( x_{\left\{u, v\right\}} + \sum_{w \in \left[n\right] \backslash \left\{u, v\right\}} z_{\left(u, w\right), v} 
\right)
& = & 
\frac{1}{\left|S\right|} \sum_{\left\{u, v\right\} \in {S \choose 2}} \left(2 x_{\left\{u, v\right\}}\right. \\
& & \left.+ \sum_{w \in \left[n\right] \backslash \left\{u, v\right\}} \left(z_{\left(u, w\right), v} + z_{\left(v, w\right), u}\right)\right) \\
& \stackrel{\eqref{eq:erw_symm_eq_neu2}}{=} &
\frac{1}{\left|S\right|} \sum_{\left\{u, v\right\} \in {S \choose 2}} 2.
\end{eqnarray*}
\end{itemize}

Wenn wir nun noch zu dieser vereinfachten Gleichungs- und Ungleichungsfamilie Nichtnegativitätsbedingungen für die $x$-Variablen hinzufügen, so erhalten wir, wie wir soeben gesehen haben, eine vereinfachte erweiterte Formulierung des Spannbaum-Polytops. Dies notieren wir als Satz:

\begin{Sa}
\label{sa:pre_Pxy}
\begin{eqnarray*}
P^{K_n, spt, symm, pre-xy} := \left\{
\left(\begin{array}{c}
x \\
z
\end{array}\right)\in\mathbb{R}^{[n]\choose 2}\times \mathbb{R}^{[n]^{\underline{3}}}: \nonumber\right. & & \\
x_{\left\{v_1, v_2\right\}} + x_{\left\{v_2, v_3\right\}} + x_{\left\{v_3, v_1\right\}} 
- z_{\left(v_1, v_2\right), v_3} - z_{\left(v_2, v_1\right), v_3} & & \nonumber \\
- z_{\left(v_2, v_3\right), v_1} - z_{\left(v_3, v_2\right), v_1} 
- z_{\left(v_3, v_1\right), v_2} - z_{\left(v_1, v_3\right), v_2}
& = & 0\ \forall \left\{v_1, v_2, v_3\right\}\in {\left[n\right] \choose 3} \\
2 x_{\left\{v_1, v_2\right\}} + \sum_{u\in[n]\backslash\left\{v_1, v_2\right\}} \left(z_{\left(v_1, u\right), v_2} + z_{\left(v_2, u\right), v_1} \right) & = & 2\ \forall \left\{v_1, v_2\right\}\in {[n] \choose 2} \\
x_{\left\{v_1, v_2\right\}} & \geq & 0 \ \forall \left\{v_1, v_2\right\}\in {[n] \choose 2} \\
z_{\left(v_1, w\right), v_2}+z_{\left(v_2, w\right), v_1} & \geq & 0 \left. \ \forall \left(\left\{v_1, v_2\right\}, w\right) \in EV_n \nonumber \right\}
\end{eqnarray*}
bildet eine erweiterte Formulierung des Spannbaum-Polytops.
\end{Sa}

\section{$P^{K_n, symm, xy}$}

Nun wollen wir $P^{K_n, spt, symm, pre-xy}$ weiter betrachten und hieraus eine erweiterte Formulierung, die wir mit $P^{K_n, symm, xy}$ bezeichnen wollen, herleiten.

\begin{Def}
\label{def:P_p^symm_xy}
Das Polytop
\begin{eqnarray}
\left\{ \left(\begin{array}{c}
x \\
y
\end{array}\right)\in\mathbb{R}^{[n]\choose 2}\times \mathbb{R}^{EV_n}: \right.
\nonumber & & \\
\sum_{i=1}^3 \left(x_{\left\{v_{i}, v_{i \stackrel{\left[3\right]}{+} 1}\right\}} -  y_{\left\{v_{i}, v_{i \stackrel{\left[3\right]}{+} 1}\right\}, v_{i \stackrel{\left[3\right]}{+} 2}}\right) & = & 0\ \forall \left\{v_1, v_2, v_3\right\}\in {[n] \choose 3} \label{eq:sum_x_minus_sum_y} \\
2 x_{\left\{v_1, v_2\right\}} + \sum_{u\in[n]\backslash\left\{v_1, v_2\right\}} y_{\left\{v_1, v_2\right\}, u} & = & 2\ \forall \left\{v_1, v_2\right\}\in {[n] \choose 2} \label{eq:kante_minus_sum_z} \\
x_{\left\{v_1, v_2\right\}} & \geq & 0 \ \forall \left\{v_1, v_2\right\} \in {\left[n\right] \choose 2} \label{eq:xy_xgeq0} \\
y_{\left\{v_1, v_2\right\}, u} & \geq & \left.0 \ \forall \left(\left\{v_1, v_2\right\}, u\right) \in EV_n\right\} \label{eq:xy_ygeq0}
\end{eqnarray}
wollen wir mit $P^{K_n, spt, symm, xy}$ bezeichnen.
\end{Def}

\begin{Bem}
\label{bem:gleichungen_Pxy}
In Definition \ref{def:P_p^symm_xy} wird $P^{K_n, spt, symm, xy}$ durch
\begin{itemize}
\item $\frac{1}{6} n \left(n-1\right) \left(n-2\right)$ Gleichungen \eqref{eq:sum_x_minus_sum_y}
\item $\frac{1}{2} n \left(n-1\right)$ Gleichungen \eqref{eq:kante_minus_sum_z}
\item $\frac{1}{2} n \left(n-1\right)$ Nichtnegativitätsbedingungen für $x$-Variablen (Ungleichungen \eqref{eq:xy_xgeq0})
\item $\frac{1}{2} n \left(n-1\right) \left(n-2\right)$ Nichtnegativitätsbedingungen für $y$-Variablen (Ungleichungen \eqref{eq:xy_ygeq0})
\end{itemize}
in
\begin{itemize}
\item $\frac{1}{2} n \left(n-1\right)$ $x$-Variablen
\item $\frac{1}{2} n \left(n-1\right) \left(n-2\right)$ $y$-Variablen
\end{itemize}
beschrieben.
\end{Bem}

Dass die Ungleichungsbeschreibung von $P^{K_n, spt, symm, xy}$ eine erweiterte Formulierung des \linebreak Spann\-baum-Polytops bildet, ist Aussage des folgenden Satzes, für den wir in den folgenden Unterkapiteln zwei unterschiedliche Beweise liefern werden:

\begin{Sa}
\label{sa:P_p^xy}
$P^{K_n, spt, symm, xy}$ bildet eine erweiterte Formulierung des Spannbaum-Polytops.
\end{Sa}

\subsection{Beweis von Satz \ref{sa:P_p^xy} durch Betrachtung vorhergehender Beweise}

Wir wollen aus Satz \ref{sa:pre_Pxy} folgern, dass $P^{K_n, spt, symm, xy}$ eine erweiterte Formulierung des Spannbaum-Polytops bildet.

Dazu benutzen wir folgendes Lemma:
\begin{Le} \label{le:variablen_eliminieren}
Es seien $P \subseteq \mathbb{R}^{d_1}$, $Q \subseteq \mathbb{R}^{d_1} \times \mathbb{R}^{d_2}$ Polyeder mit
\begin{eqnarray*}
Q := \left\{ \left(\begin{array}{c}
x^1 \\
x^2
\end{array}\right) \in \mathbb{R}^{d_1} \times \mathbb{R}^{d_2}: \right. & & \\
A \left(\begin{array}{cc}
I & 0 \\
0 & S
\end{array}\right) \left(\begin{array}{c}
x^1 \\
x^2
\end{array}\right) & \leq & \left.b\right\}
\shortintertext{mit $S \in \mathbb{R}^{d'_2 \times d_2}$ mit $\rg S = d'_2$ und}
\proj_{\mathbb{R}^{d_1}} Q & = & P.
\end{eqnarray*}

Dann gilt mit
\begin{eqnarray*}
Q' := \left\{ \left(\begin{array}{c}
x^1 \\
x'^2
\end{array}\right) \in \mathbb{R}^{d_1} \times \mathbb{R}^{d'_2}: \right. & & \\
A \left(\begin{array}{c}
x^1 \\
x'^2
\end{array}\right) & \leq & \left.b\right\}
\shortintertext{die Gleichung}
\proj_{\mathbb{R}^{d_1}} Q' & = & P.
\end{eqnarray*}
\end{Le}
\begin{bew}
Da $P = \proj_{\mathbb{R}^{d_1}} Q$ ist, beweisen wir die Aussage, indem wir $\proj_{\mathbb{R}^{d_1}} Q = \proj_{\mathbb{R}^{d_1}} Q'$ zeigen.

$\proj_{\mathbb{R}^{d_1}} Q \subseteq \proj_{\mathbb{R}^{d_1}} Q'$:
Sei $x^1 \in \proj_{\mathbb{R}^{d_1}} Q$, also es gibt $x^2$ mit $\left(\begin{array}{c}
x^1 \\
x^2
\end{array}\right) \in Q$. Nach Konstruktion gilt aber dann $\left(\begin{array}{c}
x^1 \\
S x^2
\end{array}\right) \in Q'$.

$\proj_{\mathbb{R}^{d_1}} Q \supseteq \proj_{\mathbb{R}^{d_1}} Q'$:
Sei $x^1 \in \proj_{\mathbb{R}^{d_1}} Q'$. Demnach gibt es $x'^2$ mit $\left(\begin{array}{c}
x^1 \\
x'^2
\end{array}\right) \in Q'$. Weil $S$ vollen Zeilenrang hat, ist die durch $S$ induzierte lineare Abbildung surjektiv -- es gibt also ein $S^{-1}_r \in \mathbb{R}^{d_2 \times d'_2}$ mit $S \cdot S^{-1}_r = I \in \mathbb{R}^{d'_2 \times d'_2}$. Also gilt
$\left(\begin{array}{c}
x^1 \\
S^{-1}_r x'^2
\end{array}\right) \in Q$.
\end{bew}

Somit folgt Satz \ref{sa:P_p^xy} aus Satz \ref{sa:pre_Pxy} und Lemma \ref{le:variablen_eliminieren}, wobei in Lemma \ref{le:variablen_eliminieren} gesetzt werde:
\begin{itemize}
\item $d_1 := \left|{\left[n\right] \choose 2}\right|$
\item $d_2 := \left|[n]^{\underline{3}}\right|$
\item $d_2' := \left|EV_n\right|$
\end{itemize}
und $S$ sei die Matrix bezüglich der kanonischen Basis der durch
\begin{eqnarray*}
\mathbb{R}^{EV_n} & \rightarrow & \mathbb{R}^{[n]^{\underline{3}}} \\
e^{EV_n}_{\left\{v_1, v_2\right\}, u} & \mapsto & e^{[n]^{\underline{3}}}_{\left(v_1, u\right), v_2}+e^{[n]^{\underline{3}}}_{\left(v_2, u\right), v_1}.
\end{eqnarray*}
definierten linearen Abbildung.

\subsection{Ein direkter Beweis von Satz \ref{sa:P_p^xy}}

\label{sec:P_p^xy_direkt}

Nun wollen wir noch einen direkten Beweis von Satz \ref{sa:P_p^xy} angeben:

\begin{bew}
Dass sich jeder aufspannende Baum in $P^{K_n, spt, symm, xy}$ einbetten lässt, folgt aus der Herleitung der Gleichungen.

Kurz zusammengefasst: setze $x$ als den charakteristischen Vektor des aufspannenden Baums $T$ und $y$ mittels
\begin{displaymath}
y_{\left\{v_1, v_2\right\}, u} := \sum_{i=1}^2 \begin{cases}
1 & \left\{v_i, u\right\} \in T \wedge v_{3-i}\in R_{T\backslash \left\{\left\{v_i, u\right\}\right\}}\left(u\right) \\
0 & \textnormal{sonst}
\end{cases}
\end{displaymath}
mit $R$ gemäß Definition \ref{def:erreichbar_weg}.

\begin{Bem}
Man rechnet leicht nach, dass für $\left\{v_1, v_2\right\} \in {\left[n\right] \choose 2}$ gilt:
\begin{eqnarray*}
\left(y_{\left\{v_1, v_2\right\}, u} \right. & :=\left.\right) &  \sum_{i=1}^2 \begin{cases}
1 & \left\{v_i, u\right\} \in T \wedge v_{3-i}\in R_{T\backslash \left\{\left\{v_i, u\right\}\right\}}\left(u\right) \\
0 & \textnormal{sonst}
\end{cases} \\
& = & \left|w^E\left(v_1, v_2\right) \cap \left\{\left\{v_1, u\right\}, \left\{v_2, u\right\}\right\}\right|
\end{eqnarray*}
mit $w^E\left(v_1, v_2\right)$ nach Definition \ref{def:kanten_weg}.
\end{Bem}

Für die Umkehrung sei $S\subseteq \left[n\right]$ mit $\left|S\right| \geq 2$.

{
\allowdisplaybreaks
\begin{eqnarray*}
x\left(E(S)\right) & = & \sum_{\left\{v_1, v_2\right\}\in {S \choose 2}} x_{\left\{v_1, v_2\right\}} \\
& = & \frac{1}{2} \sum_{\left(v_1, v_2\right)\in S^{\underline{2}}} x_{\left\{v_1, v_2\right\}} \\
& = & \frac{1}{2 \left|S\right|} \left( \sum_{\left(v_1, v_2, v_3\right)\in S^{\underline{3}}} x_{\left\{v_1, v_2\right\}} + 2 \sum_{\left(v_1, v_2\right) \in S^{\underline{2}}} x_{\left\{v_1, v_2\right\}} \right) \\
& = & \frac{1}{2 \left|S\right|} \left(\frac{1}{3} \sum_{\left(v_1, v_2, v_3\right) \in S^{\underline{3}}} \left(x_{\left\{v_1, v_2\right\}} + x_{\left\{v_1, v_3\right\}} + x_{\left\{v_2, v_3\right\}}\right) + 2 \sum_{\left(v_1, v_2\right) \in S^{\underline{2}}} x_{\left\{v_1, v_2\right\}} \right) \\
& \stackrel{\eqref{eq:sum_x_minus_sum_y}}{=} & \frac{1}{2 \left|S\right|} \left(\frac{1}{3} \sum_{\left(v_1, v_2, v_3\right)\in S^{\underline{3}}} \left(y_{\left\{v_1, v_2\right\}, v_3} + y_{\left\{v_1, v_3\right\}, v_2} + y_{\left\{v_2, v_3\right\}, v_1}\right) \right. \\
& & \left.+ 2 \sum_{\left(v_1, v_2\right) \in S^{\underline{2}}} x_{\left\{v_1, v_2\right\}} \right) \\
& = & \frac{1}{2 \left|S\right|} \left(\sum_{\left(v_1, v_2, v_3\right) \in S^{\underline{3}}} y_{\left\{v_1, v_2\right\}, v_3} + 2 \sum_{\left(v_1, v_2\right)\in S^{\underline{2}}} x_{\left\{v_1, v_2\right\}} \right) \\
& = & \frac{1}{2 \left|S\right|} \sum_{\left(v_1, v_2\right) \in S^{\underline{2}}} \left(
2 x_{\left\{v_1, v_2\right\}} + 
\sum_{v_3\in S \backslash \left\{v_1, v_2\right\}} y_{\left\{v_1, v_2\right\}, v_3} \right) \\
& \stackrel{y\geq 0}{\leq} & \frac{1}{2 \left|S\right|} \sum_{\left(v_1, v_2\right) \in S^{\underline{2}}} \left(
2 x_{\left\{v_1, v_2\right\}} + 
\sum_{v_3\in \left[n\right] \backslash \left\{v_1, v_2\right\}} y_{\left\{v_1, v_2\right\}, v_3} \right) \\
& \stackrel{\eqref{eq:kante_minus_sum_z}}{=} & \frac{1}{2 \left|S\right|} \sum_{\left(v_1, v_2\right) \in S^{\underline{2}}} 2 \\
& = & \frac{1}{2 \left|S\right|} 2 \left|S\right| \left(\left|S\right|-1\right)\\
& = & \left|S\right|-1.
\end{eqnarray*}
}

Man sieht leicht, dass für $S=\left[n\right]$ Gleichheit gilt.

Da außerdem nach Konstruktion $x\geq 0$ erfüllt ist, ist alles gezeigt.
\end{bew}

\begin{Bem}
\label{bem:Pxy_noetige_yungleichungen}
Im Beweis von Satz \ref{sa:P_p^xy} wurde die Bedingung $y\geq 0$ nicht direkt benutzt, sondern stattdessen wurde ausgenutzt, dass für alle $S\subset \left[n\right]$ mit $2 \leq \left|S\right| \leq n-1$ gilt:
\begin{displaymath}
\sum_{\left\{v_1, v_2\right\} \in S, u\in \left[n\right] \backslash S} y_{\left\{v_1, v_2\right\}, u} \geq 0.
\end{displaymath}

Leider ist für uns nicht ersichtlich, wie man diese Eigenschaft ausnutzen könnte, um \linebreak $P^{K_n, spt, symm, xy}$ im Sinne von asymptotisch weniger Ungleichungen zu verbessern.
\end{Bem}


\chapter{Neue asymmetrische erweiterte Formulierungen}

\section{Einordnung in den mathematischen Kontext}

Bevor wir asymmetrische erweiterte Formulierungen des Spannbaum-Polytops entwickeln, wollen wir uns in diesem Abschnitt eine kurze Erklärung dazu abgeben, warum asymmetrische erweiterte Formulierungen des Spannbaum-Polytops auch in einem größerem Umfeld eine potentielle wissenschaftliche Bedeutung besitzen könnten.

Betrachten wir dazu das Perfekte-Matching-Polytop $P^{K_n, perfmatch}$ und das Mat\-ching-Polytop $P^{K_n, match}$ über dem Graphen $K_n$ (die konvexe Hülle aller perfekten Matchings bzw. Matchings über dem Graphen $K_n$).

Da $P^{K_n, perfmatch}$ für gerades $n$ eine Seite von $P^{K_n, match}$ bildet, ist es offensichtlich, dass wir für ein solches gerades $n$  eine erweiterte Formulierung von $P^{K_n, match}$ in eine erweiterte Formulierung $P^{K_n, perfmatch}$ überführen können, indem wir die zusätzliche Bedingung $x\left({\left[n\right] \choose 2}\right) = \frac{n}{2}$ einfügen.

Umgekehrt können wir (ebenfalls unter der Voraussetzung $n$ gerade) eine erweiterte Formulierung von $P^{K_n, perfmatch}$ in eine erweiterte Formulierung von $P^{match, K_n}$ überführen, indem wir $O\left(n^2\right)$ zusätzliche Ungleichungen analog zur Vorgehensweise in Abschnitt \ref{sec:konnektor_wald_erw} einfügen.

Laut \citep{schrijver2003combinatorial} gilt folgender Satz:

\begin{Sa}
Es gilt
\begin{eqnarray*}
P^{K_n, perfmatch} = \left\{ x \in \mathbb{R}^{\left[n\right] \choose 2}: \right. \\
x\left(\delta\left(S\right)\right) & \geq & 1\ \forall S \subseteq V: \left|S\right| \geq 3 \wedge \left|S\right| \textnormal{ ungerade} \\
x\left(\delta\left(v\right)\right) & = & 1\ \forall v \in V \\
x_e & \geq & \left.0 \ \forall e \in {\left[n\right] \choose 2}\right\} \\
P^{K_n, match} = \left\{ x \in \mathbb{R}^{\left[n\right] \choose 2}: \right. \\
x\left(E\left(S\right)\right) & \leq & \frac{\left|S\right| - 1}{2}\ \forall S \subseteq V: \left|S\right| \geq 3 \wedge \left|S\right| \textnormal{ ungerade} \\
x\left(\delta\left(v\right)\right) & \leq & 1\ \forall v \in V \\
x_e & \geq & \left.0 \ \forall e \in {\left[n\right] \choose 2}\right\}.
\end{eqnarray*}
\end{Sa}

Man sieht die Ähnlichkeit zwischen der Ungleichungsstruktur des Matching-Poly\-tops und der des Wald-Polytops (Satz \ref{sa:darstellung_spannbaum_waldpolytop}), die wir hier nochmal wiedergeben wollen:

\begin{eqnarray*}
P^{K_n, for} =  \left\{ x \in \mathbb{R}^E: \right. & & \nonumber \\
x\left(E\left(S\right)\right) & \leq & \left|S\right|-1\ \forall S \subseteq V: \left|S\right| \geq 2 \\
x_e & \geq & \left. 0\ \forall e \in E \right\}. \nonumber
\end{eqnarray*}

Auf der anderen Seite wird jedoch in \cite{yannakakis:comb_lin} folgender Satz bewiesen (Satz zitiert nach \cite{optima:ext_form_comb_opt}):

\begin{Sa}
Für ein gerades n ist die Größe jeder erweiterten Formulierung des Per\-fek\-te-Matching-Polytops über $K_n$ (mindestens) $\Omega \left( 
{
n \choose
\left\lfloor\frac{ \left(n-2\right)}{4} \right\rfloor
}
\right)$ (exponentiell in n).
\end{Sa}

Ob ein analoger Satz (Nichtexistenz einer kompakten erweiterten Formulierung) auch für asymmetrische erweiterte Formulierungen des (Perfekte-)Matching-Polytops gilt, ist ein offenes Problem.

Aufgrund der ähnlich aussehenden Ungleichungen des Wald- versus des Matching-Polytops ist es jedoch, falls eine kompakte asymmetrische erweiterte Formulierung von $P^{K_n, perfmatch}$ (bzw. $P^{K_n, match}$) existiert, durchaus plausibel, dass, um eine solche zu finden, sich Ideen hinter asymmetrischen erweiterten Formulierungen des Spannbaum- bzw. Wald-Polytops als hilfreich erweisen könnten.

\section{Definitionen}

Nun wollen wir noch für den Rest dieses Kapitel ein paar Symbole definieren:

\begin{Def}
Sei $S\subseteq \left[n\right]$. Dann definieren wir:
\begin{eqnarray*}
EV_n^{<} & := & \left\{\left(\left\{v, w\right\}, u\right) \in EV_n: \max\left(v, w\right) < u\right\}, \\
S^{\underline{3}^{1<3}} & := & \left\{\left(v, w, u\right) \in S^{\underline{3}}: v<u\right\} \\
S^{\underline{3}^{1,2<3}} & := & \left\{\left(v, w, u\right) \in S^{\underline{3}}: \max\left(v, w\right) < u\right\}.
\end{eqnarray*}
\end{Def}

\section{Elimination von Variablen und Gleichungen ($P^{K_n, spt, asymm, 3}$ und $P^{K_n, spt, asymm, 4}$)}

In diesem Kapitel wollen wir zeigen, wie wir durch alternative Aufsummierung der Gleichungen mit einem Bruchteil der Bedingungen gegenüber $P^{K_n, spt, symm, 0}$ eine erweiterte Formulierung des Spannbaum-Polytops erhalten können.

\begin{Sa}
\label{sa:asymmetrisch_erweitert3}
\begin{align}
P^{K_n, spt, asymm, 3} := \left\{
\left(\begin{array}{c}
x \\
z
\end{array}\right)\in \mathbb{R}^{[n]\choose 2}\times \mathbb{R}^{\left[n\right]^{\underline{3}^{1<3}}}: \nonumber\right. \\
x_{\left\{v, w\right\}} - z_{\left(v, w\right), u} - z_{\left(w, v\right), u} & = 0\ \forall \left(\left\{v, w\right\}, u\right)\in EV_n^< \label{eq:spt_erweitert_asymm3_prim_1} \\
x_{\left\{v, w\right\}} + \sum_{u\in[n]\backslash\left\{v, w\right\}} z_{\left(v, u\right), w} & = 1\ \forall \left(v, w\right)\in {[n]^{\underline{2}}}^< \label{eq:spt_erweitert_asymm3_prim_2} \\
x_{\left\{v, n\right\}} & \geq 0 \ \forall v\in \left[n-1\right] \label{eq:spt_erweitert_asymm3_x_n} \\
z_{\left(v, w\right), u} & \left. \geq 0 \ \forall \left(v, w, u\right)\in \left[n\right]^{\underline{3}^{1<3}} \label{eq:spt_erweitert_asymm3_z} \right\}
\end{align}
bildet eine erweiterte Formulierung des Spannbaum-Polytops.
\end{Sa}

\begin{Bem}
Die erweiterte Formulierung von $P^{K_n, spt, asymm, 3}$ aus Satz \ref{sa:asymmetrisch_erweitert3} wird durch
\begin{itemize}
\item $\frac{1}{6} n \left(n-1\right) \left(n-2\right)$ Gleichungen \eqref{eq:spt_erweitert_asymm3_prim_1}
\item $\frac{1}{2} n \left(n-1\right)$ Gleichungen \eqref{eq:spt_erweitert_asymm3_prim_2}
\item $n-1$ Nichtnegativitätsbedingungen für $x$-Komponenten \eqref{eq:spt_erweitert_asymm3_x_n}
\item $\frac{1}{2} n \left(n-1\right) \left(n-2\right)$ Nichtnegativitätsbedingungen für $z$-Komponenten \eqref{eq:spt_erweitert_asymm3_z}
\end{itemize}
in 
\begin{itemize}
\item $\frac{1}{2} n \left(n-1\right)$ $x$-Variablen
\item $\frac{1}{2} n \left(n-1\right) \left(n-2\right)$ $z$-Variablen
\end{itemize}
beschrieben.
\end{Bem}
\begin{bew}
Lediglich die Anzahl von Gleichungen vom Typ \eqref{eq:spt_erweitert_asymm3_prim_1} ist nicht unmittelbar ersichtlich.

Die wohl einfachste Möglichkeit, dies zu beweisen, ist festzustellen, dass sich jedes Element aus $EV_n^<$ auf natürliche Weise eindeutig mit einer dreielementigen Teilmenge von $\left[n\right]$ identifizieren lässt.

Von solchen gibt es $n \choose 3$ Stück.
\end{bew}

\begin{bew} (Satz \ref{sa:asymmetrisch_erweitert3})
Die Gleichungen \eqref{eq:spt_erweitert_asymm3_prim_1} und \eqref{eq:spt_erweitert_asymm3_prim_2} bilden eine Teilmenge der Gleichungen \eqref{eq:spt_erweitert_2_1} und \eqref{eq:spt_erweitert_2_2} -- somit ist die eine Inklusion offensichtlich.

Für die andere Richtung zeigen wir zuerst $x_{\left\{v, w\right\}}\geq 0 \ \forall \left\{v, w\right\} \in {\left[n\right] \choose 2}$. 

Für $\left(v, w\right)\in [n-1]^{\underline{2}^<}$ folgt aus der durch $\left(v, w\right)$ indizierten Gleichung \eqref{eq:spt_erweitert_asymm3_prim_1} (unter Nutzung von $z_{\left(v, w\right), u} \geq 0 \ \forall \left(v, w, u\right) \in \left[n\right]^{\underline{3}^{1,2\not>3}}$): $x_{\left\{v, w\right\}} \geq 0$. $x_{\left\{v, n\right\}} \geq 0$ für alle $v \in \left[n-1\right]$ gilt wegen \eqref{eq:spt_erweitert_asymm3_x_n}.

Nachdem diese Vorbereitung erledigt ist, sei $S\subseteq \left[n\right]$ mit $\left|S\right| \geq 2$. Setze
\begin{eqnarray*}
s_{max} := \max S.
\end{eqnarray*}

Dann gilt:
{
\allowdisplaybreaks
\begin{eqnarray*}
x\left(E(S)\right) & = & \sum_{\left\{v, w\right\} \in {S \backslash \left\{s_{max}\right\} \choose 2}} x_{\left\{v, w\right\}} + \sum_{v \in S \backslash \left\{s_{max}\right\}} x_{\left\{v, s_{max}\right\}} \\
& \stackrel{\eqref{eq:spt_erweitert_asymm3_prim_1}}{=} & \sum_{\left\{v, w\right\} \in {S \backslash \left\{s_{max}\right\} \choose 2}} \left(z_{\left(v, w\right), s_{max}} + z_{\left(w, v\right), s_{max}}\right) + \sum_{v \in S \backslash \left\{s_{max}\right\}} x_{\left\{v, s_{max}\right\}} \\
& = & \sum_{v \in S \backslash \left\{s_{max}\right\}} \left(x_{\left\{v, s_{max}\right\}} + \sum_{u \in S \backslash \left\{s_{max}, v\right\}} z_{\left(v, u\right), s_{max}}\right) \\
& \stackrel{z \geq 0}{\leq} & \sum_{v \in S \backslash \left\{s_{max}\right\}} \left(x_{\left\{v, s_{max}\right\}} + \sum_{u \in \left[n\right] \backslash \left\{s_{max}, v\right\}} z_{\left(v, u\right), s_{max}}\right) \\
& \stackrel{\eqref{eq:spt_erweitert_asymm3_prim_2}}{=} & \sum_{v \in S \backslash \left\{s_{max}\right\}} 1 \\
& = & \left|S\right| - 1.
\end{eqnarray*}
}

Die Bedingung $x\left({\left[n\right] \choose 2}\right) = n-1$ folgt offensichtlich aus dieser Ungleichungskette unter Beachtung, dass für $S=\left[n\right]$ die Bedingung $z_{\left(v, w\right), u} \geq 0 \ \forall \left(v, w, u\right) \in \left[n\right]^{\underline{3}^{1<3}}$ nicht benutzt wurde.
\end{bew}

Als nächstes beachte man, dass in der erweiterten Formulierung aus Satz \ref{sa:asymmetrisch_erweitert3} jede der Variablen $z_{\left(v, w\right), u}$ mit $w>u$ ausschließlich einmal auftritt (nämlich in der Gleichung \eqref{eq:spt_erweitert_asymm3_prim_2}) -- da erscheint es plausibel, dass wenn wir den Term
\begin{displaymath}
\sum_{k=u+1}^n z_{\left(v, k\right), u} \ \forall \left(v, w\right)\in \left[n-1\right]^{\underline{2}^<}
\end{displaymath}
als Schlupf in Gleichung \eqref{eq:spt_erweitert_asymm3_prim_2} betrachten und diesen weglassen (und somit aus den Gleichungen Ungleichungen machen), wir wieder eine erweiterte Formulierung erhalten. Dies ist exakt die Aussage des folgenden Satzes:

\begin{Sa}
\label{sa:asymmetrisch_erweitert4}
\begin{align}
P^{K_n, spt, asymm, 4} := \left\{
\left(\begin{array}{c}
x \\
z
\end{array}\right)\in \mathbb{R}^{[n]\choose 2} \times \mathbb{R}^{\left[n\right]^{\underline{3}^{1, 2<3}}}: \nonumber\right. \\
x_{\left\{v_1, v_2\right\}} - z_{\left(v_1, v_2\right), w} - z_{\left(v_2, v_1\right), w} & = 0\ \forall \left(\left\{v_1, v_2\right\}, w\right)\in EV_n^< \label{eq:spt_erweitert_asymm4_prim_1} \\
x_{\left\{v_1, v_2\right\}} + \sum_{u\in[v_2-1]\backslash\left\{v_1\right\}} z_{\left(v_1, u\right), v_2} & \leq 1\ \forall \left(v_1, v_2\right)\in {[n-1]^{\underline{2}}}^<  \label{eq:spt_erweitert_asymm4_prim_2} \\
x_{\left\{v, n\right\}} + \sum_{u\in[n-1]\backslash\left\{v\right\}} z_{\left(v, u\right), n} & = 1\ \forall v \in [n-1] \label{eq:spt_erweitert_asymm4_prim_2,5} \\
x_{\left\{v, n\right\}} & \geq 0 \ \forall v\in \left[n-1\right] \label{eq:spt_erweitert_asymm4_x_n} \\
z_{\left(v_1, v_2\right), v_3} & \left. \geq 0 \ \forall \left(v_1, v_2, v_3\right)\in \left[n\right]^{\underline{3}^{1, 2<3}} \right\} \label{eq:spt_erweitert_asymm4_z}
\end{align}
bildet eine erweiterte Formulierung des Spannbaum-Polytops.
\end{Sa}
\begin{bew}
Obgleich der Beweis analog zu dem Beweis von Satz \ref{sa:asymmetrisch_erweitert3} erfolgt, wollen wir ihn explizit aufschreiben.

Die Einbettung der Bäume und die Ungleichung $x_{\left\{v_1, v_2\right\}} \geq 0 \ \forall \left\{v_1, v_2\right\} \in {\left[n\right] \choose 2}$ sind offensichtlich.

Sei $S \subset \left[n\right]$ mit $2 \leq \left|S\right|$ und sei $s_{max} := \max S$.

Aus den Gleichungen \eqref{eq:spt_erweitert_asymm4_prim_2} und \eqref{eq:spt_erweitert_asymm4_prim_2,5} folgt sofort
\begin{equation}
x_{\left\{v_1, v_2\right\}} + \sum_{u\in[v_2-1]\backslash\left\{v_1\right\}} z_{\left(v_1, u\right), v_2} \leq 1\ \forall \left(v_1, v_2\right)\in {[n]^{\underline{2}}}^<. \label{eq:asymm4_2_new}
\end{equation}

Damit haben wir:
{
\allowdisplaybreaks
\begin{eqnarray*}
x\left({S \choose 2}\right) & = & \sum_{\left\{v_1, v_2\right\} \in {S \backslash \left\{s_{max}\right\} \choose 2}} x_{\left\{v_1, v_2\right\}} + \sum_{v \in S \backslash \left\{s_{max}\right\}} x_{\left\{v, s_{max}\right\}} \\
& \stackrel{\eqref{eq:spt_erweitert_asymm4_prim_1}}{=} & \sum_{\left(v_1, v_2\right) \in \left(S \backslash \left\{s_{max}\right\}\right)^{\underline{2}^<}} \left(z_{\left(v_1, v_2\right), s_{max}} + z_{\left(v_2, v_1\right), s_{max}}\right) + \sum_{v \in S \backslash \left\{s_{max}\right\}} x_{\left\{v, s_{max}\right\}} \\
& = & \sum_{v \in S \backslash \left\{s_{max}\right\}} \left(x_{\left\{v, s_{max}\right\}} + \sum_{u \in S \backslash \left\{s_{max}, v\right\}} z_{\left(v, u\right), s_{max}}\right) \\
& \stackrel{z \geq 0}{\leq} & \sum_{v \in S \backslash \left\{s_{max}\right\}} \left(x_{\left\{v, s_{max}\right\}} + \sum_{u \in \left[s_{max}-1\right] \backslash \left\{v\right\}} z_{\left(v, u\right), s_{max}}\right) \\
& \stackrel{\eqref{eq:asymm4_2_new}}{\leq} & \sum_{v \in S \backslash \left\{s_{max}\right\}} 1 \\
& = & \left|S\right| - 1.
\end{eqnarray*}
}

Für $S = \left[n\right]$ ist
{
\allowdisplaybreaks
\begin{eqnarray*}
x\left({\left[n\right] \choose 2}\right) & = & \sum_{\left\{v_1, v_2\right\} \in {\left[n-1\right] \choose 2}} x_{\left\{v_1, v_2\right\}} + \sum_{v \in \left[n-1\right]} x_{\left\{v, n\right\}} \\
& \stackrel{\eqref{eq:spt_erweitert_asymm4_prim_1}}{=} & \sum_{\left(v_1, v_2\right) \in \left[n-1\right]^{\underline{2}^<}} \left(z_{\left(v_1, v_2\right), n} + z_{\left(v_2, v_1\right), n}\right) + \sum_{v \in \left[n-1\right]} x_{\left\{v, n\right\}} \\
& = & \sum_{v \in \left[n-1\right]} \left(x_{\left\{v, n\right\}} + \sum_{u \in \left[n-1\right] \backslash \left\{v\right\}} z_{\left(v, u\right), n}\right) \\
& \stackrel{\eqref{eq:spt_erweitert_asymm4_prim_2,5}}{=} & \sum_{v \in \left[n-1\right] } 1 \\
& = & n - 1.
\end{eqnarray*}
}
\end{bew}

Analog zu Bemerkung \ref{bem:Pxy_noetige_yungleichungen} gilt:

\begin{Bem}
\label{bem:Pasymm4_noetige_yungleichungen}
Im Beweis von Satz \ref{sa:asymmetrisch_erweitert4} wurde die Bedingung $z_{\left(v_1, v_2\right), v_3} \geq 0 \ \forall \left(v_1, v_2, v_3\right)\in \left[n\right]^{\underline{3}^{1, 2<3}}$ nicht direkt benutzt, sondern stattdessen wurde ausgenutzt, dass für alle $S\subset \left[n\right]$ mit $\left|S\right| \geq 2$ bei $s_{max} := \max S$ gilt:
\begin{displaymath}
\sum_{v \in S \backslash \left\{s_{max}\right\}} \sum_{u \in \left[s_{max}-1\right] \backslash S} z_{\left(v, u\right), s_{max}} \geq 0.
\end{displaymath}

Der Schlupf in dieser Ungleichung bildet zusammen mit dem Schlupf der entsprechenden Ungleichungen aus \eqref{eq:spt_erweitert_asymm4_prim_2} den Schlupf in den Cycle-Elimination-Ungleichungen.

Leider ist für uns nicht ersichtlich, wie man dies ausnutzen könnte, um $P^{K_n, spt, asymm, 4}$ asymptotisch zu verbessern.
\end{Bem}

\begin{Bem}
\label{bem:asymm_erw4_eigenschaften}
$P^{K_n, spt, asymm, 4}$ wird in Satz \ref{sa:asymmetrisch_erweitert4} durch
\begin{itemize}
\item $\frac{1}{6} n \left(n-1\right) \left(n-2\right)$ Gleichungen \eqref{eq:spt_erweitert_asymm4_prim_1}
\item $\frac{1}{2} \left(n-1\right) \left(n-2\right)$ Ungleichungen \eqref{eq:spt_erweitert_asymm4_prim_2}
\item $n-1$ Gleichungen \eqref{eq:spt_erweitert_asymm4_prim_2,5}
\item $n-1$ Nichtnegativitätsbedingungen für $x$-Komponenten \eqref{eq:spt_erweitert_asymm4_x_n}
\item $\frac{1}{3} n \left(n-1\right) \left(n-2\right)$ Nichtnegativitätsbedingungen für $z$-Komponenten \eqref{eq:spt_erweitert_asymm4_z}
\end{itemize}
in 
\begin{itemize}
\item $\frac{1}{2} n \left(n-1\right)$ $x$-Variablen
\item $\frac{1}{3} n \left(n-1\right) \left(n-2\right)$ $z$-Variablen
\end{itemize}
beschrieben.
\end{Bem}

\chapter{Erweiterte Formulierungen des Wald- und Konnektor-Polytops}

\section{Grundlagen und erweiterte Formulierungen}

\label{sec:konnektor_wald_erw}

Man überprüft leicht die Gültigkeit des folgenden Lemmas:

\begin{Le}
Sei ein Graph $G = \left(V, E\right)$ vorgegeben. Dann gilt:
\begin{eqnarray*}
P^{G, for} = \proj_x \left\{\left(x, \widehat{x}\right) \in \mathbb{R}^E \times \mathbb{R}^E: \right. & & \\
\widehat{x} & \in & P^{G, spt} \\
x_e & \leq & \widehat{x}_e \ \forall e \in E \\
x_e & \geq & \left.0\ \forall e \in E\right\},
\shortintertext{sowie}
P^{G, con} = \proj_x \left\{\left(x, \widehat{x}\right) \in \mathbb{R}^E \times \mathbb{R}^E: \right. & & \\
\widehat{x} & \in & P^{G, spt} \\
x_e & \geq & \widehat{x}_e\ \forall e \in E \\
x_e & \leq & \left.1\ \forall e \in E\right\}.
\end{eqnarray*}
\end{Le}

Dies liefert mit Satz \ref{sa:darstellung_spannbaum_waldpolytop} unmittelbar (nicht-kompakte) erweiterte Formulierungen für das Wald- und das Konnektor-Polytop über dem Graphen $G$.

Wenn man dies explizit aufschreibt (unter Verwendung von Bemerkung \ref{bem:spt_multicut_reduziert} und Bemerkung \ref{bem:spt_multicut_austausch}), so erhält man:
\begin{eqnarray*}
P^{G, for} = \proj_x \left\{\left(x, \widehat{x}\right) \in \mathbb{R}^E \times \mathbb{R}^E: \right. & & \nonumber \\
\widehat{x}\left(E\left(S\right)\right) & \leq & \left|S\right|-1\ \forall S \subset \left[n\right]: \left|S\right| \geq 2 \\
\widehat{x}\left(E\right) & = & \left|V\right|-1 \nonumber \\
\widehat{x}_e & \geq & 0\ \forall e \in E \\
x_e & \leq & \widehat{x}_e\ \forall e \in E \\
x_e & \geq & \left.0\ \forall e \in E\right\} \\
= \proj_x \left\{\left(x, \widehat{x}\right) \in \mathbb{R}^E \times \mathbb{R}^E: \right. & & \nonumber \\
\widehat{x}\left(\delta_E\left(\mathcal{P}\right)\right) & \geq & \left|\mathcal{P}\right|-1\ \forall \mathcal{P} \textnormal{ mit } \mathcal{P} = \left\{S\right\} \mathbin{\dot{\cup}} \mathop{\dot{\bigcup}}_{v \in V \backslash S} \left\{\left\{v\right\}\right\} \\
& & \textnormal{ mit } S \subset \left[n\right] \textnormal{ und } \left|S\right| \geq 2 \\
\widehat{x}\left(E\right) & = & \left|V\right|-1 \nonumber \\
\widehat{x}_e & \geq & 0\ \forall e \in E \\
x_e & \leq & \widehat{x}_e\ \forall e \in E \\
x_e & \geq & \left.0\ \forall e \in E\right\}
\shortintertext{bzw.}
P^{G, con} = \proj_x \left\{\left(x, \widehat{x}\right) \in \mathbb{R}^E \times \mathbb{R}^E: \right. & & \nonumber \\
\widehat{x}_e & \leq & 1\ \forall e \in E \\
\widehat{x}\left(E\left(S\right)\right) & \leq & \left|S\right|-1\ \forall S \subset \left[n\right]: \left|S\right| \geq 3 \\
\widehat{x}\left(E\right) & = & \left|V\right|-1 \nonumber \\
\widehat{x}_e & \geq & 0\ \forall e \in E \\
x_e & \geq & \widehat{x}_e\ \forall e \in E \\
x_e & \leq & \left.1\ \forall e \in E\right\} \\
= \proj_x \left\{\left(x, \widehat{x}\right) \in \mathbb{R}^E \times \mathbb{R}^E: \right. & & \nonumber \\
\widehat{x}\left(E \backslash \left\{e\right\}\right) & \geq & n-2 \ \forall e \in E \\
x\left(\delta_E\left(\mathcal{P}\right)\right) & \geq & \left|\mathcal{P}\right|-1\ \forall \mathcal{P} \textnormal{ mit } \mathcal{P} = \left\{S\right\} \mathbin{\dot{\cup}} \mathop{\dot{\bigcup}}_{v \in V \backslash S} \left\{\left\{v\right\}\right\} \\
& & \textnormal{ mit } S \subset \left[n\right] \textnormal{ und } \left|S\right| \geq 3 \\
\widehat{x}\left(E\right) & = & \left|V\right|-1 \nonumber \\
\widehat{x}_e & \geq & 0\ \forall e \in E \\
x_e & \geq & \widehat{x}_e\ \forall e \in E \\
x_e & \leq & \left.1\ \forall e \in E\right\}.
\end{eqnarray*}

Man sieht also sofort, dass für die erweiterte Formulierung von $P^{G, for}$ die $\widehat{x}_e \geq 0$-Bedingungen redundant und für die erweiterte Formulierung von $P^{G, con}$ die $\widehat{x}_e \leq 1$-Bedingungen (und somit nach Bemerkung \ref{bem:spt_multicut_austausch} die $\widehat{x}\left(E \backslash \left\{e\right\}\right) \geq n-2$-Bedingungen) redundant sind.

Dies liefert die Grundlage für die folgende Definition und das folgende Lemma.

\begin{Def}
Sei $G=\left(V, E\right)$ ein Graph. Dann definieren wir
\begin{eqnarray*}
P^{G, \underline{spt}} & = &  \left\{ x \in \mathbb{R}^E: \right. \nonumber \\
& & x\left(E\left(S\right)\right) \leq \left|S\right|-1\ \forall S \subset \left[n\right]: \left|S\right| \geq 2 \\
& & x\left(E\right) = \left.\left|V\right|-1 \right\} \nonumber \\
& = & \left\{ x \in \mathbb{R}^E: \right. \nonumber \\
& & x\left(\delta_E\left(\mathcal{P}\right)\right) \geq \left|\mathcal{P}\right|-1\ \forall \mathcal{P} \textnormal{ mit } \mathcal{P} = \left\{S\right\} \mathbin{\dot{\cup}} \mathop{\dot{\bigcup}}_{v \in V \backslash S} \left\{\left\{v\right\}\right\} \textnormal{ mit } S \subset \left[n\right] \textnormal{ und } \left|S\right| \geq 2 \\
& & x\left(E\right) = \left. \left|V\right|-1 \right\} \nonumber \\
P^{G, \overline{spt}} & = & \left\{ x \in \mathbb{R}^E: \right. \nonumber \\
& & x\left(E\left(S\right)\right) \leq \left|S\right|-1\ \forall S \subset \left[n\right]: \left|S\right| \geq 3 \\
& & x\left(E\right) = \left|V\right|-1 \nonumber \\
& & x_e \geq \left. 0\ \forall e \in E \right\} \nonumber \\
& = & \left\{ x \in \mathbb{R}^E: \right. \nonumber \\
& & x\left(\delta_E\left(\mathcal{P}\right)\right) \geq \left|\mathcal{P}\right|-1\ \forall \mathcal{P} \textnormal{ mit } \mathcal{P} = \left\{S\right\} \mathbin{\dot{\cup}} \mathop{\dot{\bigcup}}_{v \in V \backslash S} \left\{\left\{v\right\}\right\} \textnormal{ mit } S \subset \left[n\right] \textnormal{ und } \left|S\right| \geq 3 \\
& & x\left(E\right) = \left|V\right|-1 \\
& & x_e \geq \left. 0\ \forall e \in E \right\} \nonumber \\
P^{G, \overline{\underline{spt}}} & = & \left\{ x \in \mathbb{R}^E: \right. \nonumber \\
& & x\left(E\left(S\right)\right) \leq \left|S\right|-1\ \forall S \subset \left[n\right]: \left|S\right| \geq 3 \\
& & x\left(E\right) =  \left.\left|V\right|-1 \nonumber \right\} \nonumber \\
& = & \left\{ x \in \mathbb{R}^E: \right. \nonumber \\
& & x\left(\delta_E\left(\mathcal{P}\right)\right) \geq \left|\mathcal{P}\right|-1\ \forall \mathcal{P} \textnormal{ mit } \mathcal{P} = \left\{S\right\} \mathbin{\dot{\cup}} \mathop{\dot{\bigcup}}_{v \in V \backslash S} \left\{\left\{v\right\}\right\} \textnormal{ mit } S \subset \left[n\right] \textnormal{ und } \left|S\right| \geq 3 \\
& & x\left(E\right) = \left.\left|V\right|-1 \right\}. \nonumber
\end{eqnarray*}
\end{Def}

\begin{Le}
Sei $P^1$ ein Polyeder mit $P^{G, spt} \subseteq P^1 \subseteq P^{G, \underline{spt}}$. Dann ist
\begin{eqnarray*}
P^{G, for} = \proj_x \left\{\left(x, \widehat{x}\right) \right. & \in & \mathbb{R}^E \times \mathbb{R}^E: \nonumber \\
\widehat{x} & \in & P^1 \\
x_e & \leq & \widehat{x}_e\ \forall e \in E \\
x_e & \geq & \left.0\ \forall e \in E\right\}.
\end{eqnarray*}

Sei $P^2$ ein Polyeder mit $P^{G, spt} \subseteq P^2 \subseteq P^{G, \overline{spt}}$. Dann ist
\begin{eqnarray*}
P^{G, con} = \proj_x \left\{\left(x, \widehat{x}\right) \right. & \in & \mathbb{R}^E \times \mathbb{R}^E: \nonumber \\
\widehat{x} & \in & P^2 \\
x_e & \geq & \widehat{x}_e\ \forall e \in E \\
x_e & \leq & \left.1\ \forall e \in E\right\}.
\end{eqnarray*}

Für $P^3$ mit $P^{G, spt} \subseteq P^3 \subseteq P^{G, \overline{\underline{spt}}}$ ist
\begin{eqnarray*}
P^{G, for} = \proj_x \left\{\left(x, \widehat{x}\right) \right. & \in & \mathbb{R}^E \times \mathbb{R}^E: \nonumber \\
\widehat{x} & \in & P^3 \\
x_e & \leq & \widehat{x}_e\ \forall e \in E \\
x_e & \geq & 0\ \forall e \in E \\
x_e & \leq & \left.1\ \forall e \in E\right\} \\
P^{G, con} = \proj_x \left\{\left(x, \widehat{x}\right) \right. & \in & \mathbb{R}^E \times \mathbb{R}^E: \nonumber \\
\widehat{x} & \in & P^3 \\
x_e & \geq & \widehat{x}_e\ \forall e \in E \\
x_e & \geq & 0\ \forall e \in E \\
x_e & \leq & \left.1\ \forall e \in E\right\}.
\end{eqnarray*}
\end{Le}

\begin{Kor}
\label{kor:erw_form_wald_konnektor}
Sei $G=\left(V, E\right)$ ein Graph, $M$ eine beliebige Menge und seien $P^1, P^2, P^3 \subseteq \mathbb{R}^{E} \times \mathbb{R}^M$ Polyeder.

Es gelte $P^{G, spt} \subseteq \proj_{\mathbb{R}^{E}} P^1 \subseteq P^{G, \underline{spt}}$. Dann ist
\begin{eqnarray*}
P^{G, for} = \proj_{\mathbb{R}^{E}} \left\{\left(x, \widehat{x}, \widehat{y}\right) \right. & \in & \mathbb{R}^E \times \mathbb{R}^E \times \mathbb{R}^M: \nonumber \\
\left(\widehat{x}, \widehat{y}\right) & \in & P^1 \\
x_e & \leq & \widehat{x}_e\ \forall e \in E \\
x_e & \geq & \left.0\ \forall e \in E\right\}.
\end{eqnarray*}

Es gelte $P^{G, spt} \subseteq \proj_{\mathbb{R}^{E}} P^2 \subseteq P^{G, \overline{spt}}$. Dann ist
\begin{eqnarray*}
P^{G, con} = \proj_{\mathbb{R}^{E}} \left\{\left(x, \widehat{x}, \widehat{y}\right) \right. & \in & \mathbb{R}^E \times \mathbb{R}^E \times \mathbb{R}^M: \nonumber \\
\left(\widehat{x}, \widehat{y}\right) & \in & P^2 \\
x_e & \geq & \widehat{x}_e\ \forall e \in E \\
x_e & \leq & \left.1\ \forall e \in E\right\}.
\end{eqnarray*}

Falls $P^{G, spt} \subseteq \proj_{\mathbb{R}^{E}} P^3 \subseteq P^{G, \overline{\underline{spt}}}$ gilt, ist
\begin{eqnarray*}
P^{G, for} = \proj_{\mathbb{R}^{E}} \left\{\left(x, \widehat{x}, \widehat{y}\right) \right. & \in & \mathbb{R}^E \times \mathbb{R}^E \times \mathbb{R}^M: \nonumber \\
\left(\widehat{x}, \widehat{y}\right) & \in & P^3 \\
x_e & \leq & \widehat{x}_e\ \forall e \in E \\
x_e & \geq & 0\ \forall e \in E \\
x_e & \leq & \left.1\ \forall e \in E\right\} \\
P^{G, con} = \proj_{\mathbb{R}^{E}} \left\{\left(x, \widehat{x}, \widehat{y}\right) \right. & \in & \mathbb{R}^E \times \mathbb{R}^E: \nonumber \\
\left(\widehat{x}, \widehat{y}\right) & \in & P^3 \\
x_e & \geq & \widehat{x}_e\ \forall e \in E \\
x_e & \geq & 0\ \forall e \in E \\
x_e & \leq & \left.1\ \forall e \in E\right\}.
\end{eqnarray*}
\end{Kor}

Da jede der vorgestellten erweiterten Formulierungen des Spannbaum-Polytops über $K_n$ die Voraussetzungen an $P_1$, $P_2$ und $P_3$ aus Korollar \ref{kor:erw_form_wald_konnektor} erfüllt, erhalten wir sofort erweiterte Formulierungen des Wald- und Konnektor-Polytops.

Es ist anzumerken, dass für die in dieser Arbeit dargestellten erweiterten Formulierungen des Spannbaum-Polytops beim Weglassen der $x \geq 0$-Bedingungen in diesen die Bedingungen an $P_1$ bzw. $P_3$ erfüllt bleiben, da im Beweis der Cycle-Elimination-Ungleichungen kein Gebrauch von den $x \geq 0$-Bedingungen gemacht wurde.

Als Abschluss dieses Abschnitts wollen wir die aus $P^{K_n, spt, symm, xy}$ und $P^{K_n, spt, asymm, 4}$ resultierenden erweiterten Formulierungen des Wald- und Konnektor-Poly\-tops explizit aufschreiben:

\begin{Bsp}
\begin{eqnarray*}
\left\{ \left(\begin{array}{c}
x \\
\widehat{x} \\
y
\end{array}\right)\in\mathbb{R}^{[n]\choose 2}\times \mathbb{R}^{[n]\choose 2} \times \mathbb{R}^{EV_n}: \right.
\nonumber & & \\
\sum_{i=1}^3 \left(\widehat{x}_{\left\{v_{i}, v_{i \stackrel{\left[3\right]}{+} 1}\right\}} -  y_{\left\{v_{i}, v_{i \stackrel{\left[3\right]}{+} 1}\right\}, v_{i \stackrel{\left[3\right]}{+} 2}}\right) & = & 0\ \forall \left\{v_1, v_2, v_3\right\}\in {[n] \choose 3} \\
2 \widehat{x}_{\left\{v_1, v_2\right\}} + \sum_{u\in[n]\backslash\left\{v_1, v_2\right\}} y_{\left\{v_1, v_2\right\}, u} & = & 2\ \forall \left\{v_1, v_2\right\}\in {[n] \choose 
2} \\
x_{\left\{v_1, v_2\right\}} - \widehat{x}_{\left\{v_1, v_2\right\}} & \leq & 0 \ \forall \left\{v_1, v_2\right\}\in {[n] \choose 2} \\
x_{\left\{v_1, v_2\right\}} & \geq & 0 \ \forall \left\{v_1, v_2\right\}\in {[n] \choose 2} \nonumber \\
y_{\left\{v_1, v_2\right\}, v_3} & \geq & 0 \ \forall \left(\left\{v_1, v_2\right\}, v_3\right) \in EV_n \left.\right\} \nonumber
\shortintertext{bildet eine erweiterte Formulierung von $P^{K_n, for}$ und}
\left\{ \left(\begin{array}{c}
x \\
\widehat{x} \\
y
\end{array}\right)\in\mathbb{R}^{[n]\choose 2}\times \mathbb{R}^{[n]\choose 2} \times \mathbb{R}^{EV_n}: \right.
\nonumber & & \\
\sum_{i=1}^3 \left(\widehat{x}_{\left\{v_{i}, v_{i \stackrel{\left[3\right]}{+} 1}\right\}} -  y_{\left\{v_{i}, v_{i \stackrel{\left[3\right]}{+} 1}\right\}, v_{i \stackrel{\left[3\right]}{+} 2}}\right) & = & 0\ \forall \left\{v_1, v_2, v_3\right\}\in {[n] \choose 3} \\
2 \widehat{x}_{\left\{v_1, v_2\right\}} + \sum_{u\in[n]\backslash\left\{v_1, v_2\right\}} y_{\left\{v_1, v_2\right\}, u} & = & 2\ \forall \left\{v_1, v_2\right\}\in {[n] \choose 2} \\
x_{\left\{v_1, v_2\right\}} & \leq & 1\ \forall \left\{v_1, v_2\right\}\in {[n] \choose 2} \\
\widehat{x}_{\left\{v_1, v_2\right\}} - x_{\left\{v_1, v_2\right\}} & \leq & 0\ \forall \left\{v_1, v_2\right\}\in {[n] \choose 2} \\
\widehat{x}_{\left\{v_1, v_2\right\}} & \geq & 0 \ \forall \left\{v_1, v_2\right\}\in {[n] \choose 2} \nonumber \\
y_{\left\{v_1, v_2\right\}, v_3} & \geq & 0 \ \forall \left(\left\{v_1, v_2\right\}, v_3\right) \in EV_n \left.\right\} \nonumber
\shortintertext{bildet eine erweiterte Formulierung von $P^{K_n, con}$.}
\end{eqnarray*}
\end{Bsp}

\begin{Bsp}
\begin{eqnarray*}
\left\{
\left(\begin{array}{c}
x \\
\widehat{x} \\
z
\end{array}\right)\in \mathbb{R}^{[n]\choose 2} \times \mathbb{R}^{[n]\choose 2} \times \mathbb{R}^{\left[n\right]^{\underline{3}^{1, 2<3}}}: \nonumber\right. \\
\widehat{x}_{\left\{v_1, v_2\right\}} - z_{\left(v_1, v_2\right), w} - z_{\left(v_2, v_1\right), w} & = & 0\ \forall \left(\left\{v_1, v_2\right\}, w\right)\in EV_n^< \\
\widehat{x}_{\left\{v_1, v_2\right\}} + \sum_{u\in[v_2-1]\backslash\left\{v_1\right\}} z_{\left(v_1, u\right), v_2} & \leq & 1\ \forall \left(v_1, v_2\right)\in {[n-1]^{\underline{2}}}^<  \\
\widehat{x}_{\left\{v, n\right\}} + \sum_{u\in[n-1]\backslash\left\{v\right\}} z_{\left(v, u\right), n} & = & 1\ \forall v \in [n-1] \\
x_{\left\{v_1, v_2\right\}} - \widehat{x}_{\left\{v_1, v_2\right\}} & \leq & 0 \ \forall \left\{v_1, v_2\right\}\in {[n] \choose 2} \\
x_{\left\{v_1, v_2\right\}} & \geq & 0 \ \forall \left\{v_1, v_2\right\}\in {[n] \choose 2} \nonumber \\
z_{\left(v_1, v_2\right), v_3} & \geq & 0 \left. \ \forall \left(v_1, v_2, v_3\right)\in \left[n\right]^{\underline{3}^{1, 2<3}} \right\}
\shortintertext{bildet eine erweiterte Formulierung von $P^{K_n, for}$ und}
\left\{
\left(\begin{array}{c}
x \\
\widehat{x} \\
z
\end{array}\right)\in \mathbb{R}^{[n]\choose 2} \times \mathbb{R}^{[n]\choose 2} \times \mathbb{R}^{\left[n\right]^{\underline{3}^{1, 2<3}}}: \nonumber\right. \\
\widehat{x}_{\left\{v_1, v_2\right\}} - z_{\left(v_1, v_2\right), w} - z_{\left(v_2, v_1\right), w} & = & 0\ \forall \left(\left\{v_1, v_2\right\}, w\right)\in EV_n^< \\
\widehat{x}_{\left\{v_1, v_2\right\}} + \sum_{u\in[v_2-1]\backslash\left\{v_1\right\}} z_{\left(v_1, u\right), v_2} & \leq & 1\ \forall \left(v_1, v_2\right)\in {[n-1]^{\underline{2}}}^<  \\
\widehat{x}_{\left\{v, n\right\}} + \sum_{u\in[n-1]\backslash\left\{v\right\}} z_{\left(v, u\right), n} & = & 1\ \forall v \in [n-1] \\
x_{\left\{v_1, v_2\right\}} & \leq & 1\ \forall \left\{v_1, v_2\right\}\in {[n] \choose 2} \\
\widehat{x}_{\left\{v_1, v_2\right\}} - x_{\left\{v_1, v_2\right\}} & \leq & 0\ \forall \left\{v_1, v_2\right\}\in {[n] \choose 2} \\
\widehat{x}_{\left\{v, n\right\}} & \geq & 0 \ \forall v\in \left[n-1\right] \\
z_{\left(v_1, v_2\right), v_3} & \geq & 0 \left. \ \forall \left(v_1, v_2, v_3\right)\in \left[n\right]^{\underline{3}^{1, 2<3}} \right\}
\shortintertext{bildet eine erweiterte Formulierung von $P^{K_n, con}$.}
\end{eqnarray*}
\end{Bsp}

\chapter{Vereinfachte erweiterte Formulierungen des Wald- und Konnektor-Polytops}

\label{sec:erw_all}

\section{Symmetrische erweiterte Formulierungen}


\begin{Sa}
\label{sa:erw_all_symm}
Das Polytop
\begin{eqnarray}
\left\{
\left(\begin{array}{c}
x \\
y
\end{array}\right)\in \mathbb{R}^{[n]\choose 2} \times\mathbb{R}^{EV_n}:\right.
\nonumber & & \\
\sum_{u \in \left[n\right] \backslash \left\{v_1, v_2, v_3\right\}} \left(y_{\left\{v_1, v_2\right\}, u} + y_{\left\{v_1, v_3\right\}, u} + y_{\left\{v_2, v_3\right\}, u}\right) & & \nonumber \\
+ 3 \left(y_{\left\{v_1, v_2\right\}, v_3} + y_{\left\{v_1, v_3\right\}, v_2} + y_{\left\{v_2, v_3\right\}, v_1}\right) & = & 6 \ \forall \left\{v_1, v_2, v_3\right\}\in {[n] \choose 3} \label{eq:simpl_sum} \\
2 x_{\left\{v_1, v_2\right\}} + \sum_{u \in \left[n\right] \backslash \left\{v_1, v_2\right\}} y_{\left\{v_1, v_2\right\}, u} & \substack{= \\ \leq \\ \geq} & 2 \label{eq:simpl_sum2} \\
x_{\left\{v_1, v_2\right\}} & \geq & 0 \ \forall \left\{v_1, v_2\right\} \in {\left[n\right] \choose 2} \nonumber \\
y_{\left\{v_1, v_2\right\}, u} & \geq & \left.0 \ \forall \left(\left\{v_1, v_2\right\}, u\right) \in EV_n\right\} \nonumber
\end{eqnarray}
bildet eine erweiterte Formulierung von $\substack{P^{K_n, spt} \\ P^{K_n, for} \\ {P^{K_n, con}}^+}$.
\end{Sa}

\begin{Prop}
\label{prop:erw_all_symm}
Sei $\left(
\begin{array}{c}
x \\
y
\end{array}
\right)$ ein Punkt eines Polyeders aus Satz \ref{sa:erw_all_symm} und sei $S \subseteq \left[n\right]$. Dann gilt
\begin{displaymath}
\sum_{\left\{v_1, v_2\right\} \in {S \choose 2}} \sum_{u \in \left[n\right] \backslash \left\{v_1, v_2\right\}} y_{\left\{v_1, v_2\right\}, u} \geq \left(\left|S\right| - 1\right) \cdot \left(\left|S\right| - 2\right)
\end{displaymath}
mit Gleichheit im Falle $S=\left[n\right]$.
\end{Prop}
\begin{bew}
\begin{eqnarray*}
\sum_{\left\{v_1, v_2\right\} \in {S \choose 2}} \sum_{u \in \left[n\right] \backslash \left\{v_1, v_2\right\}} y_{\left\{v_1, v_2\right\}, u} & = & \sum_{\left\{v_1, v_2, v_3\right\} \in {S \choose 3}} \left(
3 \left(y_{\left\{v_1, v_2\right\}, v_3} + y_{\left\{v_1, v_3\right\}, v_2} + y_{\left\{v_2, v_3\right\}, v_1}\right)
\right. \\
& & + \sum_{u \in  \left[n\right] \backslash \left\{v_1, v_2, v_3\right\}} \left(y_{\left\{v_1, v_2\right\}, u} + y_{\left\{v_1, v_3\right\}, u} + y_{\left\{v_2, v_3\right\}, u}\right) \\
& & \left.- 2 \sum_{u \in \left[n\right] \backslash S} \left(y_{\left\{v_1, v_2\right\}, u} + y_{\left\{v_1, v_3\right\}, u} + y_{\left\{v_2, v_3\right\}, u}\right)\right) \\
& \stackrel{y \geq 0}{\geq} & \sum_{\left\{v_1, v_2, v_3\right\} \in {S \choose 3}} \left(
3 \left(y_{\left\{v_1, v_2\right\}, v_3} + y_{\left\{v_1, v_3\right\}, v_2} + y_{\left\{v_2, v_3\right\}, v_1}\right)
\right. \\
& & \left.+ \sum_{u \in  \left[n\right] \backslash \left\{v_1, v_2, v_3\right\}} \left(y_{\left\{v_1, v_2\right\}, u} + y_{\left\{v_1, v_3\right\}, u} + y_{\left\{v_2, v_3\right\}, u}\right)\right) \\
& \stackrel{\eqref{eq:simpl_sum}}{=} & \left(\left|S\right| - 1\right) \cdot \left(\left|S\right| - 2\right).
\end{eqnarray*}
\end{bew}

\begin{bew} (Satz \ref{sa:erw_all_symm}) Die Einbettungen sind klar:
\begin{itemize}
\item Für einen Wald ergänze ihn zu einem Baum und setze $y$ entsprechend seiner kanonischen Einbettung.
\item Für ein Element aus $x + r \in {P^{K_n, con}}^+$ ($x$ -- charakteristischer Vektor eines Konnektors, $r \in \mathbb{R}^{\left[n\right] \choose 2}_{\geq 0}$) reduziere den Konnektor, welcher zu $x$ gehört, zu einem Baum und setze $y$ entsprechend der kanonischen Einbettung dieses Baums.
\end{itemize}

Sei $S \subseteq \left[n\right]$. Dann gilt:
\begin{eqnarray*}
\sum_{\left\{v_1, v_2\right\} \in {S \choose 2}} x_{\left\{v_1, v_2\right\}} & \stackrel{\eqref{eq:simpl_sum2}}{\substack{= \\ \leq}} & \frac{\left|S\right| \cdot \left(\left|S\right| - 1\right)}{2} - \frac{1}{2} \underbrace{\sum_{\left\{v_1, v_2\right\} \in {S \choose 2}} \sum_{u \in \left[n\right] \backslash \left\{v_1, v_2\right\}} y_{\left\{v_1, v_2\right\}, u}}_{\stackrel{\textnormal{Prop. \ref{prop:erw_all_symm}}}{\geq} \left(\left|S\right| - 1\right) \cdot \left(\left|S\right| - 2\right)} \\
& \leq & \frac{\left|S\right| \cdot \left(\left|S\right| - 1\right)}{2} - \frac{\left(\left|S\right| - 1\right) \cdot \left(\left|S\right| - 2\right)}{2} \\
& = & \left|S\right| - 1.
\end{eqnarray*}

Sei umgekehrt $\mathcal{P}$ eine Partition von $\left[n\right]$. Dann gilt
\begin{eqnarray*}
\sum_{\left\{v_1, v_2\right\} \in \delta\left(\mathcal{P}\right)} x_{\left\{v_1, v_2\right\}} & \stackrel{\eqref{eq:simpl_sum2}}{\substack{= \\ \geq}} & \frac{n \cdot \left(n - 1\right)}{2} - \sum_{P \in \mathcal{P}} \frac{\left|P\right| \cdot \left(\left|P\right| - 1\right)}{2} \\
& &  - \frac{1}{2} \sum_{\left\{P_1, P_2\right\} \in {\mathcal{P} \choose 2}} \sum_{\substack{v_1 \in P_1, \\
v_2 \in P_2}} \sum_{u \in \left[n\right] \backslash \left\{v_1, v_2\right\}} y_{\left\{v_1, v_2\right\}, u} \\
& = & \frac{n \cdot \left(n - 1\right)}{2} - \sum_{P \in \mathcal{P}} \frac{\left|P\right| \cdot \left(\left|P\right| - 1\right)}{2} \\
& &  - \frac{1}{2} \underbrace{\sum_{\left\{v_1, v_2\right\} \in {\left[n\right] \choose 2}} \sum_{u \in \left[n\right] \backslash \left\{v_1, v_2\right\}} y_{\left\{v_1, v_2\right\}, u}}_{\stackrel{\textnormal{Prop. \ref{prop:erw_all_symm}}}{=} \left(n - 1\right) \cdot \left(n - 2\right)} \\
& &  + \frac{1}{2} \sum_{P \in \mathcal{P}} \underbrace{\sum_{\left\{v_1, v_2\right\} \in {\mathcal{P} \choose 2}} \sum_{u \in \left[n\right] \backslash \left\{v_1, v_2\right\}} y_{\left\{v_1, v_2\right\}, u}}_{\stackrel{\textnormal{Prop. \ref{prop:erw_all_symm}}}{\geq} \left(\left|P\right| - 1\right) \cdot \left(\left|P\right| - 2\right)} \\
& \geq & \frac{n \cdot \left(n - 1\right)}{2} - \sum_{P \in \mathcal{P}} \frac{\left|P\right| \cdot \left(\left|P\right| - 1\right)}{2}  \\
& & - \frac{\left(n - 1\right) \cdot \left(n - 2\right)}{2} + \sum_{P \in \mathcal{P}} \frac{\left(\left|P\right| - 1\right) \cdot \left(\left|P\right| - 2\right)}{2} \\
& = & \left(n - 1\right) - \underbrace{\sum_{P \in \mathcal{P}} \left(\left|P\right| - 1\right)}_{= n - \left|\mathcal{P}\right|} \\
& = & \left|\mathcal{P}\right| - 1.
\end{eqnarray*}
\end{bew}

\section{Asymmetrische erweiterte Formulierungen}

\begin{Sa}
\label{sa:asymmetrisch_erweitert_konnektor4}
Das Polytop
\begin{eqnarray}
\left\{
\left(\begin{array}{c}
x \\
z
\end{array}\right)\in \mathbb{R}^{[n]\choose 2} \times \mathbb{R}^{\left[n\right]^{\underline{3}^{1, 2<3}}}: \nonumber\right. \\
z_{\left(v_1, v_2\right), n} + z_{\left(v_2, v_1\right), n} - z_{\left(v_1, v_2\right), w} - z_{\left(v_2, v_1\right), w} & = & 0\ \forall \left(\left\{v_1, v_2\right\}, w\right)\in EV_{n-1}^< \label{eq:sptkon_erweitert_asymm4_prim_1_1} \\
z_{\left(v_1, v_2\right), n} + z_{\left(v_2, v_1\right), n} + \sum_{u\in[v_2-1]\backslash\left\{v_1\right\}} z_{\left(v_1, u\right), v_2} & \leq & 1\ \forall \left(v_1, v_2\right)\in {[n-1]^{\underline{2}}}^<  \label{eq:sptkon_erweitert_asymm4_prim_2} \\
x_{\left\{v_1, v_2\right\}} - z_{\left(v_1, v_2\right), n} - z_{\left(v_2, v_1\right), n} & \substack{= \\ \leq \\ \geq} & 0\ \forall \left\{v_1, v_2\right\}\in {\left[n-1\right] \choose 2} \label{eq:sptkon_erweitert_asymm4_prim_1_2} \\
x_{\left\{v, n\right\}} + \sum_{u\in[n-1]\backslash\left\{v\right\}} z_{\left(v, u\right), n} & \substack{= \\ \leq \\ \geq} & 1\ \forall v \in [n-1] \label{eq:sptkon_erweitert_asymm4_prim_2,5} \\
x_{\left\{v, n\right\}} & \geq & 0 \ \forall v\in \left[n-1\right]\textnormal{, falls $P^{K_n, spt}$ oder ${P^{K_n, con}}^+$} \nonumber \\
x_{\left\{v_1, v_2\right\}} & \geq & 0 \ \forall \left\{v_1, v_2\right\}\in {\left[n\right] \choose 2}\textnormal{, falls $P^{K_n, for}$} \nonumber \\
z_{\left(v_1, v_2\right), v_3} & \geq & \left. 0 \ \forall \left(v_1, v_2, v_3\right)\in \left[n\right]^{\underline{3}^{1, 2<3}} \right\} \nonumber
\end{eqnarray}
bildet eine erweiterte Formulierung von $\substack{P^{K_n, spt} \\ P^{K_n, for} \\ {P^{K_n, con}}^+}$.
\end{Sa}

\begin{Prop}
\label{prop:asymmetrisch_erweitert_konnektor4}
Sei $\left(
\begin{array}{c}
x \\
z
\end{array}
\right)$ ein Punkt eines Polyeders aus Satz \ref{sa:asymmetrisch_erweitert_konnektor4} und sei $S \subseteq \left[n-1\right]$. Dann gilt
\begin{displaymath}
\sum_{\left\{v_1, v_2\right\} \in {S \choose 2}} \left(z_{\left(v_1, v_2\right), n} + z_{\left(v_2, v_1\right), n}\right) \leq \left|S\right| - 1.
\end{displaymath}
\end{Prop}
\begin{bew}
\begin{eqnarray*}
\sum_{\left\{v_1, v_2\right\} \in {S \choose 2}} \left(z_{\left(v_1, v_2\right), n} + z_{\left(v_2, v_1\right), n}\right) & \stackrel{\eqref{eq:sptkon_erweitert_asymm4_prim_1_1}}{=} & \sum_{\left\{v_1, v_2\right\} \in {S \backslash \left\{\max S\right\} \choose 2}} \left(z_{\left(v_1, v_2\right), \max S} + z_{\left(v_2, v_1\right), \max S}\right) \\
& & + \sum_{v \in S \backslash \left\{\max S\right\}} \left(z_{\left(v, \max S\right), n} + z_{\left(\max S, v\right), n}\right) \\
& = & \sum_{v_1 \in S \backslash \left\{\max S\right\}} \left(z_{\left(v_1, \max S\right), n} + z_{\left(\max S, v_1\right), n} \right. \\
& & \left. + \sum_{v_2 \in S \backslash \left\{\max S, v_1\right\}} z_{\left(v_1, v_2\right), \max S}\right) \\
& \stackrel{z \geq 0}{\leq} & \sum_{v_1 \in S \backslash \left\{\max S\right\}} \left(z_{\left(v_1, \max S\right), n} + z_{\left(\max S, v_1\right), n} \right. \\
& & \left. + \sum_{v_2 \in \left[n\right] \backslash \left\{\max S, v_1\right\}} z_{\left(v_1, v_2\right), \max S}\right) \\
& \stackrel{\eqref{eq:sptkon_erweitert_asymm4_prim_2}}{\leq} & \left|S\right| - 1.
\end{eqnarray*}
\end{bew}

\begin{bew} (Satz \ref{sa:asymmetrisch_erweitert_konnektor4})
Das Thema Einbettungen ist klar (analog Beweis Satz \ref{sa:erw_all_symm}).

Sei $S \subseteq \left[n\right)$. Der erste Fall ist $n \notin S$
\begin{eqnarray*}
\sum_{\left\{v_1, v_2\right\} \in {S \choose 2}} x_{\left\{v_1, v_2\right\}} & \stackrel{\eqref{eq:sptkon_erweitert_asymm4_prim_1_2}}{\substack{= \\ \leq}} & \underbrace{\sum_{\left\{v_1, v_2\right\} \in {S \choose 2}} \left(z_{\left(v_1, v_2\right), n} + z_{\left(v_2, v_1\right), n}\right)}_{\stackrel{\textnormal{Prop. \ref{prop:asymmetrisch_erweitert_konnektor4}}}{\leq} \left|S\right| - 1} \\
& \leq & \left|S\right| - 1.
\end{eqnarray*}

Der zweite Fall ist $n \in S$:
\begin{eqnarray*}
\sum_{\left\{v_1, v_2\right\} \in {S \choose 2}} x_{\left\{v_1, v_2\right\}} & = & \sum_{\left\{v_1, v_2\right\} \in {S\backslash \left\{n\right\} \choose 2}} x_{\left\{v_1, v_2\right\}} + \sum_{v \in S\backslash \left\{n\right\}} x_{\left\{v, n\right\}} \\
& \stackrel{\eqref{eq:sptkon_erweitert_asymm4_prim_1_2}, \eqref{eq:sptkon_erweitert_asymm4_prim_2,5}}{\substack{= \\ \leq}} & \sum_{\left\{v_1, v_2\right\} \in {S\backslash \left\{n\right\} \choose 2}} \left(z_{\left(v_1, v_2\right), n} + z_{\left(v_2, v_1\right), n}\right) + \sum_{v \in S\backslash \left\{n\right\}} \left(1-\sum_{u\in[n-1]\backslash\left\{v\right\}} z_{\left(v, u\right), n}\right) \\
& = & \left|S\right| - 1 - \sum_{v \in S\backslash \left\{n\right\}} \sum_{u \in \left[n-1\right] \backslash S} z_{\left(v, u\right), n} \\
& \stackrel{z \geq 0}{\leq} & \left|S\right| - 1.
\end{eqnarray*}

Sei $\mathcal{P}$ eine Partitionierung der Menge $\left[n\right]$, sei $P^n$ das Element von $\mathcal{P}$, welches $n$ enthält und sei $\mathcal{P}^{\backslash n} := \mathcal{P} \backslash \left\{ P^n \right\}$, sowie $\mathcal{P}^* := \left(\mathcal{P} \backslash \left\{ P^n \right\}\right) \mathbin{\dot{\cup}} \left\{\left\{P^n \backslash \left\{n\right\}\right\}\right\}$.

\begin{eqnarray*}
\sum_{\left\{P_1, P_2\right\} \in {\mathcal{P} \choose 2}} \sum_{\left\{v_1, v_2\right\} \in {\delta\left(P_1, P_2\right)}} x_{\left\{v_1, v_2\right\}} & \stackrel{\eqref{eq:sptkon_erweitert_asymm4_prim_1_2}, \eqref{eq:sptkon_erweitert_asymm4_prim_2,5}}{\substack{= \\ \geq}} & \sum_{\left\{P_1, P_2\right\} \in
{\mathcal{P}^* \choose 2}} \sum_{\left\{v_1, v_2\right\} \in {\delta\left(P_1, P_2\right)}} \left(z_{\left(v_1, v_2\right), n} + z_{\left(v_2, v_1\right), n}\right) \\
& & + \sum_{P \in \mathcal{P}^{\backslash n}} \sum_{v \in P} \left(1 - \sum_{u\in[n-1]\backslash\left\{v\right\}} z_{\left(v, u\right), n}\right) \\
& = & n - \left|P^n\right| + \sum_{P_1 \in \mathcal{P}^{\backslash n}} \sum_{v_1 \in P_1} \left(\sum_{P_2 \in \mathcal{P}^* \backslash \left\{P_1\right\}} \sum_{v_2 \in P_2} z_{\left(v_1, v_2\right), n} \right. \\
& & \left. - \sum_{v_2\in[n-1]\backslash\left\{v_1\right\}} z_{\left(v_1, v_2\right), n} \right) \\
& & + \sum_{v_1 \in P^n \backslash \left\{n\right\}} \sum_{P \in \mathcal{P} \backslash \left\{ P^n \right\}} \sum_{v_2 \in P} z_{\left(v_1, v_2\right), n} \\
& = & n - \left|P^n\right| - \sum_{P \in \mathcal{P}^{\backslash n}} \underbrace{\sum_{\left\{v_1, v_2\right\} \in {P \choose 2}} \left(z_{\left(v_1, v_2\right), n} + z_{\left(v_2, v_1\right), n}\right)}_{\stackrel{\textnormal{Prop. \ref{prop:asymmetrisch_erweitert_konnektor4}}}{\leq} \left|P\right| - 1} \\
& & + \underbrace{\sum_{v_1 \in P^n \backslash \left\{n\right\}} \sum_{P \in \mathcal{P} \backslash \left\{ P^n \right\}} \sum_{v_2 \in P} z_{\left(v_1, v_2\right), n}}_{\stackrel{z \geq 0}{\geq} 0} \\
& \geq & n - \left|P^n\right| - \sum_{P \in \mathcal{P}^{\backslash n}} \left(\left|P\right| - 1\right) \\
& = & n - \left|P^n\right| - \sum_{P \in \mathcal{P}} \left|P\right| +  \left(\left|\mathcal{P}\right| - 1\right) \\
& = & \left|\mathcal{P}\right| - 1.
\end{eqnarray*}
\end{bew}

\chapter{Erweiterte Formulierungen des Branching- und Arboreszenz-Polytops}

\label{sec:branch_arb}

\begin{Def}
Sei $D=\left(V, A\right)$ ein Digraph (also $V$ endlich und $A \subseteq V^{\underline{2}}$). Bezeichne $\vec{K}_n$ den vollständigen Digraphen auf der Knotenmenge $\left[n\right]$ (also $\left(\left[n\right], \left[n\right]^{\underline{2}}\right)$).
\end{Def}

\begin{Def}
Sei $D=\left(V, A\right)$ ein Digraph und sei $D':=\left(V', A'\right)$ mit $V'\subseteq V$ und $A' \subseteq A$. Dann heißt $D'$ \emph{Unter-Digraph} von $D$.
\end{Def}

\begin{Def}
Ein Digraph $D$ heißt \emph{Branching}, wenn $D$ keine ungerichteten Kreise enthält und jeder Knoten höchstens einen eingehenden Bogen besitzt.

Ein schwach zusammenhängendes Branching nennen wir \emph{Arboreszenz}.
\end{Def}

\begin{Def}
Sei $D=\left(V, A\right)$ ein Digraph und sei $D':=\left(V, A'\right)$ ein Unter-Digraph von $D$.

Wenn $D'$ ein Branching ist, so bezeichnen wir $D'$ als \emph{Branching in $D$}.

Falls $D'$ eine Arboreszenz bildet, so nennen wir $D'$ \emph{Arboreszenz in $D$}.
\end{Def}

\begin{Def}
Sei $D=\left(V, A\right)$ ein Digraph. Dann definieren wir das Branching- bzw. Arbores\-zenz-Polytop über $D$ mittels
\begin{eqnarray*}
P^{D, branch} & := & \conv \left\{\chi^A_{S}: S \subseteq A \wedge \left(V, S\right) \textnormal{ Branching in }D\right\}, \\
P^{D, arb} & := & \conv \left\{\chi^A_{S}: S \subseteq A \wedge \left(V, S\right) \textnormal{ Arboreszenz in }D\right\}.
\end{eqnarray*}
\end{Def}

In \cite{schrijver2003combinatorial} wird folgender Satz bewiesen:
\begin{Sa}
Sei $D=\left(V, A\right)$ ein Digraph. Dann gilt
\begin{eqnarray*}
P^{D, branch} = \left\{x \in \mathbb{R}^A: \right. & & \\
x\left(A\left(S\right)\right) & \leq & \left|S\right| - 1 \ \forall S \subseteq V: \left|S\right| \geq 2 \\
x\left(\delta^{in}\left(v\right)\right) & \leq & 1 \ \forall v \in V \\
x_a & \geq & \left. 0 \ \forall a \in A\right\}
\end{eqnarray*}
und
\begin{eqnarray*}
P^{D, arb} = \left\{x \in \mathbb{R}^A: \right. & & \\
x\left(A\left(S\right)\right) & \leq & \left|S\right| - 1 \ \forall S \subset V: \left|S\right| \geq 2 \\
x\left(A\right) & = & \left|V\right| - 1 \\
x\left(\delta^{in}\left(v\right)\right) & \leq & 1 \ \forall v \in V \\
x_a & \geq & \left. 0 \ \forall a \in A\right\},
\end{eqnarray*}
wobei wie allgemein üblich
\begin{displaymath}
\delta^{in}\left(v\right) := \left\{\left(u, v\right): u \in V, \left(u, v\right) \in A\right\}
\end{displaymath}
definiert sei.
\end{Sa}

Diese Ungleichungen besitzen eine große Ähnlichkeit zu der äußeren Darstellung des Wald- und Spannbaum-Polytops aus Satz \ref{sa:darstellung_spannbaum_waldpolytop}. Es stellt sich die Frage, ob wir die erweiterten Formulierungen des Spannbaum- und Wald-Polytops auf das Arboreszenz- und Branching-Polytop verallgemeinern können. Dass die Antwort positiv ist, ist Aussage des folgenden Satzes:

\begin{Sa}
\label{sa:grundlage_arboreszenz_erw}
Sei
\begin{displaymath}
P := \left\{x \in \mathbb{R}^d: A x \leq b\right\} 
\end{displaymath}
ein Polyeder und sei
\begin{eqnarray*}
P' := \left\{x \in \mathbb{R}^{d'}: \right. & & \\
A T x & \leq & b \\
A' x & \leq & \left. b' \right\} 
\end{eqnarray*}
mit $T \in \mathbb{R}^{d \times d'}$.

Außerdem solle für
\begin{displaymath}
Q := \left\{\left(\begin{array}{c}
x \\
z
\end{array}\right) \in \mathbb{R}^d \times \mathbb{R}^n : B \left(\begin{array}{c}
x \\
z
\end{array}\right) \leq c\right\} 
\end{displaymath}
gelten:
\begin{displaymath}
\proj_{\mathbb{R}^d} Q = P.
\end{displaymath}
Mit anderen Worten: die Ungleichungsbeschreibung von $Q$ bildet mit der kanonischen Projektion in den Raum der $x$-Variablen eine erweiterte Formulierung von $P$.

Dann ist
\begin{eqnarray*}
Q' := \left\{\left(\begin{array}{c}
x \\
z
\end{array}\right) \in \mathbb{R}^{d'} \times \mathbb{R}^{n}: \right. & & \\
B \left(\begin{array}{cc}
T & 0 \\
0 & I
\end{array}\right) \left(\begin{array}{c}
x \\
z
\end{array}\right) & \leq & c \\
A' x & \leq & \left. b' \right\} 
\end{eqnarray*}
eine erweiterte Formulierung von $P'$.
\end{Sa}
\begin{bew}
Wir zeigen: $\proj_{\mathbb{R}^{d'}} Q' = P'$. Dazu zeigen wir beide Inklusionen.

$\proj_{\mathbb{R}^{d'}} Q' \subseteq P'$: 

Sei $x \in \proj_{\mathbb{R}^{d'}} Q'$, also es gibt ein $z$, so dass $\left(\begin{array}{c}
x \\
z
\end{array}\right) \in Q'$ ist, woraus
\begin{eqnarray}
B \left(\begin{array}{c}
T x \\
z
\end{array}\right) & \leq & c \nonumber \\
A' x & \leq & b'. \label{eq:A'xleqb}
\end{eqnarray}
folgt.

Demnach ist $T x \in P$. Also gilt $A T x \leq b$. Aber wegen \eqref{eq:A'xleqb} gilt auch $A' x \leq b$ -- somit ist $x \in P'$. \\

$\proj_{\mathbb{R}^{d'}} Q' \supseteq P'$: 

Sei nun $x \in P'$.  Damit ist
\begin{eqnarray*}
A T x & \leq & b \\
A' x & \leq & b'
\end{eqnarray*}
erfüllt.

Ziel ist es ein $z \in \mathbb{R}^{n}$ mit $\left(\begin{array}{c}
x \\
z
\end{array}\right) \in Q'$ zu finden.

Da $T x \in P$ ist und $\proj_{\mathbb{R}^{d'}} Q = P$ ist, gibt es ein $z$ mit $\left(\begin{array}{c}
T x \\
z
\end{array}\right) \in Q$. Hieraus folgt unmittelbar, dass mit diesem $z$ die Ungleichungen
\begin{displaymath}
B \left(\begin{array}{cc}
T & 0 \\
0 & I
\end{array}\right) \left(\begin{array}{c}
x \\
z
\end{array}\right) \leq c
\end{displaymath}
erfüllt sind. Da außerdem nach Voraussetzung $A' x \leq b'$ ist, muss demnach $\left(\begin{array}{c}
x \\
z
\end{array}\right) \in Q'$ sein.
\end{bew}

Somit erhalten wir unter Nutzung der erweiterten Formulierungen des Spannbaum-Polytops bzw. des Wald-Polytops über $K_n$ aus Kapitel \ref{sec:erw_all} folgendes Korollar:

\begin{Kor}
\begin{eqnarray*}
P^{\vec{K}_n, arb, symm, xy} := \left\{ \left(\begin{array}{c}
x \\
y
\end{array}\right)\in\mathbb{R}^{[n]^{\underline{2}}}\times \mathbb{R}^{EV_n}: \right.
\nonumber & & \\
\sum_{u \in \left[n\right] \backslash \left\{v_1, v_2, v_3\right\}} \left(y_{\left\{v_1, v_2\right\}, u} + y_{\left\{v_1, v_3\right\}, u} + y_{\left\{v_2, v_3\right\}, u}\right) & & \\
+ 3 \left(y_{\left\{v_1, v_2\right\}, v_3} + y_{\left\{v_1, v_3\right\}, v_2} + y_{\left\{v_2, v_3\right\}, v_1}\right) & = & 6 \ \forall \left\{v_1, v_2, v_3\right\}\in {[n] \choose 3} \\
2 \left(x_{\left(v_1, v_2\right)} + x_{\left(v_2, v_1\right)}\right) + \sum_{u\in[n]\backslash\left\{v_1, v_2\right\}} y_{\left\{v_1, v_2\right\}, u} & = & 2\ \forall \left\{v_1, v_2\right\}\in {[n] \choose 2} \\
x\left(\delta^{in}\left(v\right)\right) & \leq & 1 \ \forall v \in V \\
x_{\left(v_1, v_2\right)} & \geq & 0 \ \forall \left(v_1, v_2\right) \in \left[n\right]^{\underline{2}} \nonumber \\
y_{\left\{v_1, v_2\right\}, u} & \geq & 0\nonumber \left. \ \forall \left(\left\{v_1, v_2\right\}, u\right) \in EV_n \right\} \\
P^{\vec{K}_n, arb, asymm, 4} := \left\{
\left(\begin{array}{c}
x \\
z
\end{array}\right)\in \mathbb{R}^{[n]^{\underline{2}}} \times \mathbb{R}^{\left[n\right]^{\underline{3}^{1, 2<3}}}: \nonumber\right. & & \\
z_{\left(v_1, v_2\right), n} + z_{\left(v_2, v_1\right), n} - z_{\left(v_1, v_2\right), w} - z_{\left(v_2, v_1\right), w} & = & 0\ \forall \left(\left\{v_1, v_2\right\}, w\right)\in EV_{n-1}^< \\
z_{\left(v_1, v_2\right), n} + z_{\left(v_2, v_1\right), n} + \sum_{u\in[v_2-1]\backslash\left\{v_1\right\}} z_{\left(v_1, u\right), v_2} & \leq & 1\ \forall \left(v_1, v_2\right)\in {[n-1]^{\underline{2}}}^< \\
x_{\left(v_1, v_2\right)} + x_{\left(v_2, v_1\right)} - z_{\left(v_1, v_2\right), n} - z_{\left(v_2, v_1\right), n} & = & 0\ \forall \left\{v_1, v_2\right\}\in {\left[n-1\right] \choose 2} \\
x_{\left(v, n\right)} + x_{\left(n, v\right)} + \sum_{u\in[n-1]\backslash\left\{v\right\}} z_{\left(v, u\right), n} & = & 1\ \forall v \in [n-1] \\
x\left(\delta^{in}\left(v\right)\right) & \leq & 1 \ \forall v \in V \\
x_{\left(v_1, v_2\right)} & \geq & 0 \ \forall \left(v_1, v_2\right) \in \left[n\right]^{\underline{2}} \nonumber \\
z_{\left(v_1, v_2\right), u} & \geq & 0\nonumber \left. \ \forall \left(v_1, v_2, u\right) \in \left[n\right]^{\underline{3}^{1, 2 < 3}}\right\}
\end{eqnarray*}
bilden erweiterte Formulierungen des Arboreszenz-Polytops $P^{\vec{K}_n, arb}$ und
\begin{eqnarray*}
P^{\vec{K}_n, branch, symm, xy} := \left\{ \left(\begin{array}{c}
x \\
y
\end{array}\right)\in\mathbb{R}^{[n]^{\underline{2}}}\times \mathbb{R}^{EV_n}: \right.
\nonumber & & \\
\sum_{u \in \left[n\right] \backslash \left\{v_1, v_2, v_3\right\}} \left(y_{\left\{v_1, v_2\right\}, u} + y_{\left\{v_1, v_3\right\}, u} + y_{\left\{v_2, v_3\right\}, u}\right) & & \nonumber \\
+ 3 \left(y_{\left\{v_1, v_2\right\}, v_3} + y_{\left\{v_1, v_3\right\}, v_2} + y_{\left\{v_2, v_3\right\}, v_1}\right) & = & 6 \ \forall \left\{v_1, v_2, v_3\right\}\in {[n] \choose 3} \\
2 \left(x_{\left(v_1, v_2\right)} + x_{\left(v_2, v_1\right)}\right) + \sum_{u\in[n]\backslash\left\{v_1, v_2\right\}} y_{\left\{v_1, v_2\right\}, u} & \leq & 2\ \forall \left\{v_1, v_2\right\}\in {[n] \choose 2} \\
x\left(\delta^{in}\left(v\right)\right) & \leq & 1 \ \forall v \in V \\
x_{\left(v_1, v_2\right)} & \geq & 0 \ \forall \left(v_1, v_2\right) \in \left[n\right]^{\underline{2}} \nonumber \\
y_{\left\{v_1, v_2\right\}, u} & \geq & 0\nonumber \left. \ \forall \left(\left\{v_1, v_2\right\}, u\right) \in EV_n \right\} \\
P^{\vec{K}_n, branch, asymm, 4} := \left\{
\left(\begin{array}{c}
x \\
z
\end{array}\right)\in \mathbb{R}^{[n]^{\underline{2}}} \times \mathbb{R}^{\left[n\right]^{\underline{3}^{1, 2<3}}}: \nonumber\right. & & \\
z_{\left(v_1, v_2\right), n} + z_{\left(v_2, v_1\right), n} - z_{\left(v_1, v_2\right), w} - z_{\left(v_2, v_1\right), w} & = & 0\ \forall \left(\left\{v_1, v_2\right\}, w\right)\in EV_{n-1}^< \\
z_{\left(v_1, v_2\right), n} + z_{\left(v_2, v_1\right), n} + \sum_{u\in[v_2-1]\backslash\left\{v_1\right\}} z_{\left(v_1, u\right), v_2} & \leq & 1\ \forall \left(v_1, v_2\right)\in {[n-1]^{\underline{2}}}^< \\
x_{\left(v_1, v_2\right)} + x_{\left(v_2, v_1\right)} - z_{\left(v_1, v_2\right), n} - z_{\left(v_2, v_1\right), n} & \leq & 0\ \forall \left\{v_1, v_2\right\}\in {\left[n-1\right] \choose 2} \\
x_{\left(v, n\right)} + x_{\left(n, v\right)} + \sum_{u\in[n-1]\backslash\left\{v\right\}} z_{\left(v, u\right), n} & \leq & 1\ \forall v \in [n-1] \\
x\left(\delta^{in}\left(v\right)\right) & \leq & 1 \ \forall v \in V \\
x_{\left(v_1, v_2\right)} & \geq & 0 \ \forall \left(v_1, v_2\right) \in \left[n\right]^{\underline{2}} \nonumber \\
z_{\left(v_1, v_2\right), u} & \geq & 0\nonumber \left. \ \forall \left(v_1, v_2, u\right) \in \left[n\right]^{\underline{3}^{1, 2 < 3}}\right\}
\end{eqnarray*}
bilden erweiterte Formulierungen des Branching-Polytops $P^{\vec{K}_n, branch}$.
\end{Kor}

\part{Eigenschaften der erweiterten Formulierungen}

\chapter{Der Zusammenhang zwischen $P^{K_n, spt, symm, 0}$ und $P^{K_n, spt, asymm, 4}$}

Nun wollen anschauen, ob für $P^{K_n, spt, asymm, 4}$ die Aussage
\begin{equation}
\proj_{\mathbb{R}^{\left[n\right] \choose 2} \times \mathbb{R}^{\left[n\right]^{\underline{3}^{1,2<3}}}} P^{K_n, spt, symm, 0} = P^{K_n, spt, asymm, 4} \label{eq:proj_asymm4}
\end{equation}
erfüllt ist.

Für $n=2$ ist es trivial nachzurechnen, dass Gleichung \eqref{eq:proj_asymm4} gilt.

Umgekehrt folgt aus Satz \ref{sa:einb_asymm_4} weiter hinten in dieser Arbeit, dass Gleichung \eqref{eq:proj_asymm4} für $n \geq 4$ nicht erfüllt ist.

Das muss deswegen gelten, weil nach Satz \ref{sa:einbettung_normale_erweiterung} die Einbettung jedes Baums in $P^{K_n, spt, symm, 0}$ eindeutig bestimmt ist und somit diese Eindeutigkeit auch für die Projektion gelten muss. Satz \ref{sa:einb_asymm_4} zeigt jedoch, dass diese Eindeutigkeit für $n \geq 4$ nicht erfüllt ist.

Somit ist nur der Fall $n = 3$ zu untersuchen.

\begin{Le}
\label{le:proj_Pasymm4_n3}
Es gilt
\begin{displaymath}
\proj_{\mathbb{R}^{\left[3\right] \choose 2} \times \mathbb{R}^{\left[3\right]^{\underline{3}^{1,2<3}}}} P^{K_3, spt, symm, 0} = P^{K_3, spt, asymm, 4}.
\end{displaymath}
\end{Le}
\begin{bew}
Wir müssen zeigen, dass sich jeder Punkt $\left(\begin{array}{c}
x \\
z
\end{array}\right) \in P^{K_3, spt, asymm, 4}$ zu einem Punkt aus $P^{K_3, spt, symm, 0}$ erweitern lässt (die Gültigkeit der umgekehrten Inklusion folgt aus der Herleitung von $P^{K_3, spt, asymm, 4}$).

Sei also $\left(\begin{array}{c}
x \\
z
\end{array}\right) \in P^{K_3, spt, asymm, 4} \subset \mathbb{R}^{\left[3\right] \choose 2} \times \mathbb{R}^{\underline{3}^{1, 2 < 3}}$ gegeben.

Wir beweisen, dass es ein $\left(
\begin{array}{c}
x \\
\widetilde{z}
\end{array}
\right) \in P^{K_3, spt, symm, 0} \subset \mathbb{R}^{\left[n\right] \choose 2} \times \mathbb{R}^{\underline{3}}$ mit $\widetilde{z}_{\left(1, 2\right), 3} = z_{\left(1, 2\right), 3}$ und $\widetilde{z}_{\left(2, 1\right), 3} = z_{\left(2, 1\right), 3}$ gibt.

Setze dazu
\begin{eqnarray*}
\widetilde{z}_{\left(1, 3\right), 2} & = & \widetilde{z}_{\left(2, 3\right), 1} \\
& = & 1 - x_{\left\{1, 2\right\}} \\
\widetilde{z}_{\left(3, 1\right), 2} & = & 1 - x_{\left(2, 3\right)} \\
\widetilde{z}_{\left(3, 2\right), 1} & = & 1 - x_{\left(1, 3\right)}.
\end{eqnarray*}

$\widetilde{z}_{\left(v_1, v_2\right), v_3} \geq 0 \ \forall \left(v_1, v_2, v_3\right) \in \left[3\right]^{\underline{3}}$ ist trivialerweise erfüllt -- somit sind nur Gleichungen \eqref{eq:spt_erweitert_primal2_1} und \eqref{eq:spt_erweitert_primal2_2} zu überprüfen.

Gleichung \eqref{eq:spt_erweitert_primal2_2}
\begin{eqnarray*}
x_{\left\{v_1, v_2\right\}} + \sum_{u\in[3]\backslash\left\{v_1, v_2\right\}} \widetilde{z}_{\left(v_1, u\right), v_2} & = & 1\ \forall \left(v_1, v_2\right)\in [3]^{\underline{2}}
\end{eqnarray*}
gilt nach Konstruktion wegen Gleichung \eqref{eq:spt_erweitert_asymm4_prim_2,5}.

Somit müssen wir nur absichern, dass Gleichung \eqref{eq:spt_erweitert_primal2_1}
\begin{eqnarray*}
x_{\left\{v_1, v_2\right\}} - \widetilde{z}_{\left(v_1, v_2\right), u} - \widetilde{z}_{\left(v_2, v_1\right), u} & = 0\ \forall \left(\left\{v_1, v_2\right\}, u\right)\in EV_3
\end{eqnarray*}
erfüllt ist.

Dies ist nur für $\left(\left\{v, w\right\}, u\right) \in \left\{\left(\left\{1, 3\right\}, 2\right), \left(\left\{2, 3\right\}, 1\right)\right\}$ zu untersuchen 
(für die Belegung $\left(\left\{v_1, v_2\right\}, u\right) = \left(\left\{1, 2\right\}, 3\right)$ ist dies Aussage von Gleichung \eqref{eq:spt_erweitert_asymm4_prim_1}.

Für diese Belegungen folgt dies jedoch unmittelbar aus $x_{\left(1, 2\right)} + x_{\left(1, 3\right)} + x_{\left(2, 3\right)} = 2$.
\end{bew}

Aus Lemma \ref{le:proj_Pasymm4_n3} und den Vorbetrachtungen folgt damit folgendes Korollar:
\begin{Kor}
Für $n\geq 4$ gilt:
\begin{displaymath}
\proj_{\mathbb{R}^{\left[n\right] \choose 2} \times \mathbb{R}^{\left[n\right]^{\underline{3}^{1,2<3}}}} P^{K_n, spt, symm, 0} \neq P^{K_n, spt, asymm, 4}.
\end{displaymath}
Für $n\leq 3$ ist jedoch
\begin{displaymath}
\proj_{\mathbb{R}^{\left[n\right] \choose 2} \times \mathbb{R}^{\left[n\right]^{\underline{3}^{1,2<3}}}} P^{K_n, spt, symm, 0} = P^{K_n, spt, asymm, 4}
\end{displaymath}
erfüllt.
\end{Kor}

\chapter{Die Notwendigkeit der Nichtnegativitätsbedingungen von $P^{K_n, spt, symm, xy}$}

In diesem Kapitel soll es um die Frage gehen: sind Nichtnegativitätsbedingungen von $P^{K_n, spt, symm, xy}$ in dem Sinne irredundant, dass, wenn man sie weglässt, das Ergebnis keine erweiterte Formulierung des Spannbaum-Polytops mehr darstellt?

Diese Frage wird in den folgenden beiden Sätzen beantwortet:

\begin{Sa}
Sämtliche $x_{\left\{v_1, v_2\right\}}\geq 0 \ \forall \left\{v_1, v_2\right\} \in {\left[n\right] \choose 2}$-Bedingungen in der Beschreibung von $P^{K_n, spt, symm, xy}$ in Definition \ref{def:P_p^symm_xy} sind für $n\geq 4$ in dem Sinne tatsächlich erforderlich, dass ein Weglassen einer einzelnen dieser Nichtnegativitätsbedingungen dazu führt, dass keine erweiterte Formulierung des Spannbaum-Polytops mehr vorliegt.

Für $n\leq 3$ sind sie in dem Sinne verzichtbar, dass das Polytop beim Weglassen sämtlicher Nichtnegativitätsbedingungen für $x$-Variablen unverändert bleibt.
\end{Sa}
\begin{bew}
Für $n = 2$ folgt sofort aus Gleichung \eqref{eq:kante_minus_sum_z}, dass $x_{\left\{1, 2\right\}} = 1$ ist.

Für $n = 3$ seien $v_1, v_2, v_3$ im Folgenden stets paarweise verschieden gewählt.

Wir rechnen
\begin{eqnarray*}
3 x_{\left\{v_1, v_2\right\}} 
& \stackrel{\eqref{eq:kante_minus_sum_z}}{=} & 2-y_{\left\{v_1, v_2\right\}, v_3}+x_{\left\{v_1, v_2\right\}}  \\
& \stackrel{\eqref{eq:sum_x_minus_sum_y}}{=} & 2+y_{\left\{v_2, v_3\right\}, v_1}+y_{\left\{v_1, v_3\right\}, v_2}-x_{\left\{v_1, v_3\right\}}-x_{\left\{v_2, v_3\right\}}.
\end{eqnarray*}

Es gilt $y_{\left\{v_2, v_3\right\}, v_1}, y_{\left\{v_1, v_3\right\}, v_2} \geq 0$, sowie es folgt aus Gleichung \eqref{eq:kante_minus_sum_z} unter Nutzung von $y_{\left\{v_1, v_3\right\}, v_2}$, $y_{\left\{v_2, v_3\right\}, v_1} \geq 0$, dass $x_{\left\{v_1, v_3\right\}}, x_{\left\{v_2, v_3\right\}} \leq 1$ sind. Hieraus folgt die zu beweisende Aussage
\begin{displaymath}
x_{\left\{v_1, v_2\right\}} \geq 0.
\end{displaymath}

Nun zu $n\geq 4$: Um die Notation zu vereinfachen, werden wir die Aussage für die Ungleichung $x_{\left\{n-1, n\right\}} \geq 0$ beweisen. Jede beliebige andere Kante zeigt man durch Ausnutzung der Symmetrie analog.

Setze
\begin{displaymath}
x_{\left\{v_1, v_2\right\}} := \begin{cases}
\frac{n+1}{2\left(n-1\right)} & \min\left(v, w\right) \leq n-2 \wedge \max\left(v_1, v_2\right)\in \left\{n-1, n\right\} \\
-\frac{n-3}{n-1} & \left\{v_1, v_2\right\} = \left\{n-1, n\right\} \\
0 & \textnormal{sonst}
\end{cases}
\end{displaymath}
und
\begin{displaymath}
y_{\left\{v_1, v_2\right\}, u} := \begin{cases}
1 & v_1, v_2\notin \left\{n-1, n\right\} \wedge u\in \left\{n-1, n\right\} \\
\frac{1}{n-1} & \min\left(v_1, v_2\right) \leq n-2 \wedge \max\left(v_1, v_2\right)\in \left\{n-1, n\right\} \wedge u\leq n-2 \\
\frac{4}{n-1} & \left\{v_1, v_2\right\} = \left\{n-1, n\right\} \\
0 & \textnormal{sonst}
\end{cases}
\end{displaymath}

Wir müssen zeigen, dass dieser Punkt die Gleichungen \eqref{eq:sum_x_minus_sum_y} und \eqref{eq:kante_minus_sum_z} erfüllt.

Zu \eqref{eq:sum_x_minus_sum_y}, also
\begin{displaymath}
\sum_{i=1}^3 \left(x_{\left\{v_{i}, v_{i \stackrel{\left[3\right]}{+} 1}\right\}} - y_{\left\{v_{i}, v_{i \stackrel{\left[3\right]}{+} 1}\right\}, v_{i \stackrel{\left[3\right]}{+} 2}}\right) = 0
\end{displaymath}
mit $\left\{v_1, v_2, v_3\right\} \in {\left[n\right] \choose 3}$:

\paragraph{Fall 1: $n-1, n\notin \left\{v_1, v_2, v_3\right\}$} Dann gilt nach offenbar
\begin{eqnarray*}
\sum_{i=1}^3 \left(x_{\left\{v_{i}, v_{i \stackrel{\left[3\right]}{+} 1}\right\}} -  y_{\left\{v_{i}, v_{i \stackrel{\left[3\right]}{+} 1}\right\}, v_{i \stackrel{\left[3\right]}{+} 2}}\right) & = & 0,
\end{eqnarray*}
da alle Summanden $0$ sind.

\paragraph{Fall 2: Genau einer der beiden Werte $n-1, n$ liegt in $\left\{v_1, v_2, v_3\right\}$} Dann haben wir 
\begin{eqnarray*}
\sum_{i=1}^3 x_{\left\{v_{i}, v_{i \stackrel{\left[3\right]}{+} 1}\right\}} & = & 2 \cdot \frac{n+1}{2\left(n-1\right)} + 0 \\
& = & \frac{n+1}{n-1}
\end{eqnarray*}
und
\begin{eqnarray*}
\sum_{i=1}^3 z_{
\left\{
v_{i}, 
v_{i \stackrel{\left[3\right]}{+} 1}
\right\}, 
v_{i \stackrel{\left[3\right]}{+} 2}} & = & 2\cdot \frac{1}{n-1} + 1 \\
& = & \frac{n+1}{n-1}.
\end{eqnarray*}
Also ist die Differenz dieser beiden Terme sicher 0.

\paragraph{Fall 3: $n-1, n\in \left\{v_1, v_2, v_3\right\}$} Dann ist 
\begin{eqnarray*}
\sum_{i=1}^3 x_{\left\{v_{i}, v_{i \stackrel{\left[3\right]}{+} 1}\right\}} & = & 2 \cdot \frac{n+1}{2\left(n-1\right)} - \frac{n-3}{n-1} \\
& = & \frac{4}{n-1}
\end{eqnarray*}
und
\begin{eqnarray*}
\sum_{i=1}^3 z_{
\left\{
v_{i}, 
v_{i \stackrel{\left[3\right]}{+} 1}
\right\}, 
v_{i \stackrel{\left[3\right]}{+} 2}} & = & 0 + 0 + \frac{4}{n-1} \\
& = & \frac{4}{n-1}.
\end{eqnarray*}
Auch hier ist die Differenz sicherlich 0.
\\

Zu Gleichung \eqref{eq:kante_minus_sum_z} 
\begin{displaymath}
2 x_{\left\{v_1, v_2\right\}} + \sum_{u\in[n]\backslash\left\{v_1, v_2\right\}} y_{\left\{v_1, v_2\right\}, u} = 2
\end{displaymath}
mit $\left\{v_1, v_2\right\} \in {\left[n\right] \choose 2}$:

\paragraph{Fall 1: $n-1, n\notin \left\{v_1, v_2\right\}$} Dann gilt nach offenbar
\begin{eqnarray*}
2 \underbrace{x_{\left\{v_1, v_2\right\}}}_{ = 0} + \sum_{u\in[n]\backslash\left\{v_1, v_2\right\}} \underbrace{y_{\left\{v_1, v_2\right\}, u}}_{ = \begin{cases}
1 & u\in \left\{n-1, n\right\} \\
0 & \textnormal{sonst}
\end{cases}} & = & 2.
\end{eqnarray*}

\paragraph{Fall 2: Genau einer der beiden Werte $n-1, n$ liegt in $\left\{v_1, v_2\right\}$} Hier haben wir
\begin{eqnarray*}
2 \underbrace{x_{\left\{v_1, v_2\right\}}}_{= \frac{n+1}{2\left(n-1\right)}} + \sum_{u\in[n]\backslash\left\{v_1, v_2\right\}} \underbrace{y_{\left\{v_1, v_2\right\}, u}}_{ = \begin{cases}
\frac{1}{n-1} & u\notin \left\{n-1, n\right\} \\
0 & u\in \left\{n-1, n\right\}
\end{cases}} & = & \frac{n+1}{n-1} + \left(n-2\right) \cdot \frac{1}{n} \\
& = & 2.
\end{eqnarray*}

\paragraph{Fall 3: $\left\{n-1, n\right\} \in \left\{v_1, v_2\right\}$} Dann ist
\begin{eqnarray*}
2 \underbrace{x_{\left\{v_1, v_2\right\}}}_{= -\frac{n-3}{n-1}} + \sum_{u\in[n]\backslash\left\{v_1, v_2\right\}} \underbrace{y_{\left\{v_1, v_2\right\}, u}}_{= \frac{4}{n-1}}
& = & - 2 \cdot \frac{n-3}{n-1} + \left(n-2\right) \frac{4}{n-1} \\
& = & \frac{2 n-2}{n-1} \\
& = & 2.
\end{eqnarray*}
\end{bew}

\begin{Sa}
Sämtliche $y_{\left\{v_1, v_2\right\}, u} \geq 0 \ \forall \left(\left\{v_1, v_2\right\}, u\right) \in EV_n$-Bedingungen in der Beschreibung von $P^{K_n, spt, symm, xy}$ in Definition \ref{def:P_p^symm_xy} sind für $n\geq 3$\footnote{Für $n=2$ ist es nicht sinnvoll, der Frage nachzugehen, da es in diesem Fall keine $y$-Variablen gibt.} tatsächlich in dem Sinne erforderlich, dass ein Weglassen einer einzelnen dieser Nichtnegativitätsbedingungen dazu führt, dass keine erweiterte Formulierung des Spannbaum-Polytops mehr vorliegt.
\end{Sa}
\begin{bew}
Ohne Einschränkung (wieder zur Vereinfachung der Notation) werden wir die Notwendigkeit für $y_{\left\{n-1, n\right\}, n-2} \geq 0$ zeigen. Für jede andere $y$-Komponente erfolgt der Beweis analog.

Setze\footnote{Man beachte, dass wir für $n=3$  haben: $\left[n-3\right] = \emptyset$. Daher existieren die Variablen, zu deren Definition durch $n-3$ dividiert wird, für $n=3$ gar nicht, weswegen diese Konstruktion auch für $n=3$ funktioniert.}
\begin{displaymath}
x_{\left\{v_1, v_2\right\}} := \begin{cases}
\frac{2}{n-3} & \left\{v_1, v_2\right\} \in {[n-3] \choose 2} \\
\frac{1}{n-3} & n-2 \in \left\{v_1, v_2\right\} \wedge \left[n-3\right] \cap \left\{v_1, v_2\right\} \neq \emptyset \\
0 & {[n-2] \choose 2} \cap \left\{v_1, v_2\right\} \neq \emptyset \wedge \left\{v_1, v_2\right\} \cap \left\{n-1, n\right\} \neq \emptyset \\
2 & \left\{v_1, v_2\right\} = \left\{n-1, n\right\} \\
\end{cases}
\end{displaymath}
und
\begin{displaymath}
y_{\left\{v_1, v_2\right\}, u} := \begin{cases}
\frac{2}{n-3} & v_1, v_2 \in \left[n-2\right] \wedge u \in \left[n-3\right] \\
\frac{1}{n-3} & \left[n-3\right] \cap \left\{v_1, v_2\right\} \neq \emptyset \wedge \left\{n-1, n\right\} \cap \left\{v_1, v_2\right\} \neq \emptyset \wedge u\in \left[n-2\right] \\
1 & \left[n-3\right] \cap \left\{v_1, v_2\right\} \neq \emptyset \wedge \left\{n-1, n\right\} \cap \left\{v_1, v_2\right\} \neq \emptyset \wedge u\in \left\{n-1, n\right\} \\
2 & \left(\left\{v, w\right\}, u\right) \in \left\{ \left(\left\{n-2, n\right\}, n-1\right), \left(\left\{n-2, n-1\right\}, n\right)\right\} \\
-2 & \left(\left\{v, w\right\}, u\right) = \left(\left\{n-1, n\right\}, n-2\right) \\
0 & \textnormal{sonst}
\end{cases}
\end{displaymath}

Nun müssen wir zeigen, dass dieser Punkt die Gleichungen \eqref{eq:sum_x_minus_sum_y} und \eqref{eq:kante_minus_sum_z} erfüllt.

Zu \eqref{eq:sum_x_minus_sum_y}, also
\begin{displaymath}
\sum_{i=1}^3 \left(x_{\left\{v_{i}, v_{i \stackrel{\left[3\right]}{+} 1}\right\}} - y_{\left\{v_{i}, v_{i \stackrel{\left[3\right]}{+} 1}\right\}, v_{i \stackrel{\left[3\right]}{+} 2}}\right) = 0
\end{displaymath}
mit $\left\{v_1, v_2, v_3\right\} \in {\left[n\right] \choose 3}$:

\paragraph{Fall 1: $\left\{v_1, v_2, v_3\right\} \in {\left[n-3\right] \choose 3}$} Hier hat man
\begin{eqnarray*}
\sum_{i=1}^3 \left(\underbrace{x_{\left\{v_{i}, v_{i \stackrel{\left[3\right]}{+} 1}\right\}}}_{= \frac{2}{n-3}} - \underbrace{y_{\left\{v_{i}, v_{i \stackrel{\left[3\right]}{+} 1}\right\}, v_{i \stackrel{\left[3\right]}{+} 2}}}_{= \frac{2}{n-3}}\right) & = & 0.
\end{eqnarray*}

\paragraph{Fall 2: $\left|\left\{v_1, v_2, v_3\right\} \cap \left[n-3\right]\right| = 2 \wedge n-2 \in \left\{v_1, v_2, v_3\right\}$} Dann ist
\begin{eqnarray*}
\sum_{i=1}^3 \left(x_{\left\{v_{i}, v_{i \stackrel{\left[3\right]}{+} 1}\right\}} - y_{\left\{v_{i}, v_{i \stackrel{\left[3\right]}{+} 1}\right\}, v_{i \stackrel{\left[3\right]}{+} 2}}\right) & = & \underbrace{\sum_{i=1}^3 x_{\left\{v_{i}, v_{i \stackrel{\left[3\right]}{+} 1}\right\}}}_{= \frac{1}{n-3} + \frac{1}{n-3} + \frac{2}{n-3}} - \underbrace{\sum_{i=1}^3 y_{\left\{v_{i}, v_{i \stackrel{\left[3\right]}{+} 1}\right\}, v_{i \stackrel{\left[3\right]}{+} 2}}}_{= \frac{2}{n-3} + \frac{2}{n-3} + 0 } \\
& = & 0.
\end{eqnarray*}

\paragraph{Fall 3: $\left|\left\{v_1, v_2, v_3\right\} \cap \left[n-3\right]\right| = 2 \wedge \left|\left\{n-1, n\right\} \cap \left\{v_1, v_2, v_3\right\}\right| = 1$} Dann gilt
\begin{eqnarray*}
\sum_{i=1}^3 \left(x_{\left\{v_{i}, v_{i \stackrel{\left[3\right]}{+} 1}\right\}} - y_{\left\{v_{i}, v_{i \stackrel{\left[3\right]}{+} 1}\right\}, v_{i \stackrel{\left[3\right]}{+} 2}}\right) & = & \underbrace{\sum_{i=1}^3 x_{\left\{v_{i}, v_{i \stackrel{\left[3\right]}{+} 1}\right\}}}_{= \frac{2}{n-3} + 0 + 0} - \underbrace{\sum_{i=1}^3 y_{\left\{v_{i}, v_{i \stackrel{\left[3\right]}{+} 1}\right\}, v_{i \stackrel{\left[3\right]}{+} 2}}}_{= \frac{1}{n-3} + \frac{1}{n-3} + 0 } \\
& = & 0.
\end{eqnarray*}

\paragraph{Fall 4: $\left|\left\{v_1, v_2, v_3\right\} \cap \left[n-3\right]\right| = 1 \wedge n-2 \in\left\{v_1, v_2, v_3\right\} \wedge  \left|\left\{n-1, n\right\} \cap \left\{v_1, v_2, v_3\right\}\right| = 1$} In diesem Fall haben wir
\begin{eqnarray*}
\sum_{i=1}^3 \left(x_{\left\{v_{i}, v_{i \stackrel{\left[3\right]}{+} 1}\right\}} - y_{\left\{v_{i}, v_{i \stackrel{\left[3\right]}{+} 1}\right\}, v_{i \stackrel{\left[3\right]}{+} 2}}\right) & = & \underbrace{\sum_{i=1}^3 x_{\left\{v_{i}, v_{i \stackrel{\left[3\right]}{+} 1}\right\}}}_{= \frac{1}{n-3} + 0 + 0} - \underbrace{\sum_{i=1}^3 y_{\left\{v_{i}, v_{i \stackrel{\left[3\right]}{+} 1}\right\}, v_{i \stackrel{\left[3\right]}{+} 2}}}_{= \frac{1}{n-3} + 0 + 0} \\
& = & 0.
\end{eqnarray*}

\paragraph{Fall 5: $\left|\left\{v_1, v_2, v_3\right\} \cap \left[n-3\right]\right| = 1 \wedge  \left|\left\{n-1, n\right\} \cap \left\{v_1, v_2, v_3\right\}\right| = 2$} Dann ist
\begin{eqnarray*}
\sum_{i=1}^3 \left(x_{\left\{v_{i}, v_{i \stackrel{\left[3\right]}{+} 1}\right\}} - y_{\left\{v_{i}, v_{i \stackrel{\left[3\right]}{+} 1}\right\}, v_{i \stackrel{\left[3\right]}{+} 2}}\right) & = & \underbrace{\sum_{i=1}^3 x_{\left\{v_{i}, v_{i \stackrel{\left[3\right]}{+} 1}\right\}}}_{= 2 + 0 + 0} - \underbrace{\sum_{i=1}^3 y_{\left\{v_{i}, v_{i \stackrel{\left[3\right]}{+} 1}\right\}, v_{i \stackrel{\left[3\right]}{+} 2}}}_{= 1 + 1 + 0} \\
& = & 0.
\end{eqnarray*}

\paragraph{Fall 6: $n-2 \in \left\{v_1, v_2, v_3\right\} \wedge  \left|\left\{n-1, n\right\} \cap \left\{v_1, v_2, v_3\right\}\right| = 2$} Hier haben wir
\begin{eqnarray*}
\sum_{i=1}^3 \left(x_{\left\{v_{i}, v_{i \stackrel{\left[3\right]}{+} 1}\right\}} - y_{\left\{v_{i}, v_{i \stackrel{\left[3\right]}{+} 1}\right\}, v_{i \stackrel{\left[3\right]}{+} 2}}\right) & = & \underbrace{\sum_{i=1}^3 x_{\left\{v_{i}, v_{i \stackrel{\left[3\right]}{+} 1}\right\}}}_{= 0 + 0 + 2} - \underbrace{\sum_{i=1}^3 y_{\left\{v_{i}, v_{i \stackrel{\left[3\right]}{+} 1}\right\}, v_{i \stackrel{\left[3\right]}{+} 2}}}_{= 2 + 2 + \left(-2\right)} \\
& = & 0.
\end{eqnarray*}

Zu Gleichung \eqref{eq:kante_minus_sum_z} 
\begin{displaymath}
2 x_{\left\{v_1, v_2\right\}} + \sum_{u\in[n]\backslash\left\{v_1, v_2\right\}} y_{\left\{v_1, v_2\right\}, u} = 2
\end{displaymath}
mit $\left\{v_1, v_2\right\} \in {\left[n\right] \choose 2}$:

\paragraph{Fall 1: $\left\{v_1, v_2\right\} \in {\left[n-3\right] \choose 2}$} Hier hat man
\begin{eqnarray*}
2 \underbrace{x_{\left\{v_1, v_2\right\}}}_{= \frac{2}{n-3}} + \underbrace{\sum_{u\in[n]\backslash\left\{v_1, v_2\right\}} y_{\left\{v_1, v_2\right\}, u}}_{= \left(n-5\right) \cdot \frac{2}{n-3} + 3\cdot 0} & = & 2\cdot \frac{2}{n-3} + \left(n-5\right) \cdot \frac{2}{n-3} \\
& = & 2.
\end{eqnarray*}

\paragraph{Fall 2: $\left\{v_1, v_2\right\} \cap \left[n-3\right] \neq \emptyset \wedge n-2 \in \left\{v_1, v_2\right\}$} Es gilt
\begin{eqnarray*}
2 \underbrace{x_{\left\{v_1, v_2\right\}}}_{= \frac{1}{n-3}} + \underbrace{\sum_{u\in[n]\backslash\left\{v_1, v_2\right\}} y_{\left\{v_1, v_2\right\}, u}}_{= \left(n-4\right) \cdot \frac{2}{n-3} + 2\cdot 0} & = & 2\cdot \frac{1}{n-3} + \left(n-4\right) \cdot \frac{2}{n-3} \\
& = & 2.
\end{eqnarray*}

\paragraph{Fall 3: $\left\{v_1, v_2\right\} \cap \left[n-3\right] \neq \emptyset \wedge \left\{n-1, n\right\} \cap \left\{v_1, v_2\right\} \neq \emptyset$} In diesem Fall ist
\begin{eqnarray*}
2 \underbrace{x_{\left\{v_1, v_2\right\}}}_{= 0} + \underbrace{\sum_{u\in[n]\backslash\left\{v_1, v_2\right\}} y_{\left\{v_1, v_2\right\}, u}}_{= \left(n-3\right) \cdot \frac{1}{n-3} + 1\cdot 1} & = & \left(n-3\right) \cdot \frac{1}{n-3} + 1 \\
& = & 2.
\end{eqnarray*}

\paragraph{Fall 4: $n-2 \in \left\{v_1, v_2\right\} \wedge \left\{n-1, n\right\} \cap \left\{v_1, v_2\right\} \neq \emptyset$} Hier hat man
\begin{eqnarray*}
2 \underbrace{x_{\left\{v_1, v_2\right\}}}_{= 0} + \underbrace{\sum_{u\in[n]\backslash\left\{v_1, v_2\right\}} y_{\left\{v_1, v_2\right\}, u}}_{= \left(n-3\right) \cdot 0 + 2\cdot 1} & = & 2.
\end{eqnarray*}

\paragraph{Fall 5: $\left\{v_1, v_2\right\} = \left\{n-1, n\right\}$} Dann ist
\begin{eqnarray*}
2 \underbrace{x_{\left\{v_1, v_2\right\}}}_{= 2} + \underbrace{\sum_{u\in[n]\backslash\left\{v_1, v_2\right\}} y_{\left\{v_1, v_2\right\}, u}}_{= \left(n-3\right) \cdot 0 + 1\cdot \left(-2\right)} & = & 2.
\end{eqnarray*}
\end{bew}

\chapter{Nicht-ganzzahlige Ecken von $P^{K_n, spt, symm, 0}$ und $P^{K_n, spt, symm, xy}$}

\label{chap:nichtganzz_ecken}

Es ist klar, dass die kanonische Einbettung eines Baums in $P^{K_n, spt, symm, 0}$ und $P^{K_n, spt, symm, xy}$ eine Ecke des entsprechenden Polytops bildet.

Dass diese beiden Polytope für $n\geq 4$ jedoch auch nicht-ganzzahlige Ecken besitzen, wird den Inhalt des vorliegenden Kapitels bilden.

Zuvor betrachten wir jedoch noch ein paar mathematische Grundlagen in Bezug auf die genannte Thematik.

\section{Ecken vom Polyedern}

\label{sec:ecken}

Nachdem wir in Definition \ref{def:ecke} bereits Ecken von allgemeinen konvexen Mengen eingeführt haben, wollen wir nun Ecken von Polyedern betrachten.

Wir interessieren uns aus offensichtlichen Gründen im Rahmen dieser Arbeit für Ecken von Polyedern, die in äußerer Darstellung gegeben sind. Hierzu ist folgender Satz wohlbekannt:
\begin{Sa}
\label{sa:ecke_polyeder}
Sei
\begin{displaymath}
P := \left\{x \in \mathbb{R}^d:
\begin{array}{ccc}
A_1 x & \leq & b_1 \\
A_2 x & = & b_2
\end{array}
\right\}
\end{displaymath}
mit $A := \left(
\begin{array}{cc}
A_1 \\
A_2
\end{array}
\right)$
und $b := \left(
\begin{array}{cc}
b_1 \\
b_2
\end{array}
\right)$. Sei $x\in P$. $x$ ist genau dann eine Ecke von $P$, wenn es $d$ Zeilenindizes $J$ gibt, so dass gilt:
\begin{displaymath}
A_{J, *} x = b_J,
\end{displaymath}
und $A_{J, *}$ vollen Rang hat, d. h. die Zeilen bzw. Spalten sind linear unabhängig.
\end{Sa}

Wir wollen nun ein einfacher zu überprüfendes Kriterium beweisen, welches wir später nutzen werden:

\begin{Le}
\label{sa:ecken_verwendet}
Sei
\begin{displaymath}
P := \left\{x \in \mathbb{R}^d: 
A x = b \wedge x \geq 0
\right\}
\end{displaymath}
mit $A \in \mathbb{R}^{m \times d}$ und sei $x\in P$.

Dann ist $x$ genau dann eine Ecke von $P$, wenn für alle $y \in \mathbb{R}^d$ mit
\begin{itemize}
\item für alle $i \in \left[d\right]$ gilt: $x_i = 0 \Rightarrow y_i = 0$
\item $A y = b$
\end{itemize}
gilt: $x=y$.

Mit anderen Worten: aus der Eigenschaft, dass $A x = b$ gilt, lässt sich aus den Nullkomponenten von $x$ der Wert der restlichen Komponenten rekonstruieren und dieser rekonstruierte Vektor liegt in $P$.
\end{Le}
\begin{bew}
Sei $x$ Ecke von P. Dann gibt es $d$ linear unabhängige Zeilen der Matrix $\left(\begin{array}{c}
A \\
I
\end{array}
\right)$ mit Indizes $J$, so dass die Lösung des linearen Gleichungssystems
\begin{displaymath}
\left(
\begin{array}{c}
A \\
I
\end{array}
\right)_{J, *} x = \left(
\begin{array}{c}
b \\
0^d
\end{array}
\right)_{J, *}
\end{displaymath}
eindeutig ist und $x$ diese Lösung darstellt.

Somit ist die Lösung von $A x = b$ durch Vorgabe der Nullkomponenten von $x$ eindeutig bestimmt.

Zur umkehrten Richtung. Sei
\begin{displaymath}
\widetilde{J} := \left\{m+i: x_i = 0\right\}.
\end{displaymath}
Die Bedingung sagt, dass die Lösung des linearen Gleichungssystems
\begin{displaymath}
\left(
\begin{array}{c}
A \\
I
\end{array}
\right)_{\widetilde{J} \mathbin{\dot{\cup}} \left[m\right], *} x = \left(
\begin{array}{c}
b \\
0^d
\end{array}
\right)_{\widetilde{J} \mathbin{\dot{\cup}} \left[m\right], *}
\end{displaymath}
eindeutig bestimmt ist. Dies heißt jedoch nichts anderes, als dass die Matrix $d$ linear unabhängige Zeilen hat. Da $x \in P$ ist, ist somit $x$ eine Ecke von $P$.
\end{bew}

Man überzeugt sich leicht von der Gültigkeit des folgenden Lemmas:
\begin{Le}
Sei
\begin{displaymath}
P := \left\{x \in \mathbb{R}^d: 
\begin{array}{ccc}
A_1 x & \leq & b_1 \\
A_2 x & = & b_2 \\
A_3 x & = & b_3
\end{array}
\right\}
\end{displaymath}
und sei $x$ eine Ecke von $P$. Dann ist $x$ auch eine Ecke von
\begin{displaymath}
P := \left\{x \in \mathbb{R}^d: 
\begin{array}{ccc}
A_1 x & \leq & b_1 \\
A_2 x & \leq & b_2 \\
A_3 x & = & b_3
\end{array}
\right\}.
\end{displaymath}
\end{Le}

Welche Bedeutung hat dieses Lemma für die Arbeit? Es sagt nichts anderes aus, als dass wenn wir für eine der in der Arbeit entworfenen erweiterten Formulierungen des Spannbaum-Polytops eine Ecke gefunden haben, diese auch für die zugehörige erweiterte Formulierung des Wald- oder der Dominante des Konnektor-Polytops aus Kapitel \ref{sec:erw_all} eine Ecke bildet.

\section{Symmetrien von Ecken}

\subsection{Algebraische Hintergründe}

Die in diesem Unterkapitel dargelegten Fakten wurden \cite{bosch2006algebra} und \cite{huppert1998character} entnommen, in welchen der Leser zahlreiches weitergehendes Material finden kann.

\begin{Def}
Sei $\left(G, \cdot\right)$ eine endliche Gruppe und $X$ endlich. Sei $\circ: G \times X \rightarrow X$. Wenn
\begin{itemize}
\item $g_2 \circ \left(g_1 \circ x\right) = \left(g_2 \cdot g_1\right) \circ x$ für alle $g_1, g_2 \in G$ und $x \in X$
\item $1 \circ x = x$ für alle $x \in X$
\end{itemize}
gilt, wobei $1$ das neutrale Element von $G$ sei, so bezeichnen wir $\circ$ als \emph{Linksaktion von $G$ auf $X$} und $X$ als \emph{G-Menge}.
\end{Def}

\begin{Bem}
Ab jetzt werden wir die $\cdot$- und $\circ$-Operatoren weglassen, da stets offensichtlich ist, auf welchen Operator sich bezogen wird.
\end{Bem}

\begin{Bem}
\label{bem:produkt_G-Mengen}
Seien $X$ und $X'$ G-Mengen. Dann wird $X \times X'$ mittels
\begin{displaymath}
g \left(x, x'\right) := \left(g x, g x'\right)
\end{displaymath}
mit $g \in G$, $x \in X$ und $x' \in X'$ zu einer G-Menge.
\end{Bem}

\begin{Def}
Sei $x \in X$. Dann definieren wir
\begin{displaymath}
G x := \left\{g x: g \in G\right\}
\end{displaymath}
als \emph{(G-)Bahn} oder \emph{(G-)Orbit} von $x$.
\end{Def}

\begin{Bem}
\label{bem:gbahnen}
Sei $X$ eine G-Menge.

Dann gibt es eine eindeutige Zerlegung
\begin{displaymath}
X = \mathop{\dot{\bigcup}}_{j=1}^t X_j,
\end{displaymath}
wobei die $X_j$ G-Bahnen von Punkten aus $X$ bilden.
\end{Bem}

\begin{Def}
Sei $X$ eine G-Menge und $x \in X$. Dann definieren wir
\begin{displaymath}
G_x := \left\{g \in G: g x = x\right\}
\end{displaymath}
als die \emph{Stabilisator-Untergruppe} (oder \emph{Stabilisator}) von $x$.
\end{Def}

\begin{Le} (Bahn-Formel)
\label{le:bahnformel}
Sei $X$ eine G-Menge. Dann gilt für alle $x\in X$:
\begin{displaymath}
\left|G\right| = \left|G_x\right| \cdot \left|G x\right|
\end{displaymath}
\end{Le}

\begin{Def}
\label{def:darstellung}
Sei $V$ ein $K$-Vektorraum von endlicher Dimension $\dim_K V$ und $G$ eine endliche Gruppe. Eine \emph{Darstellung} $D$ von $G$ auf $\ V$ ist ein Gruppen-Homo\-mor\-phismus $D: G \rightarrow GL\left(V\right)$.

Wir nennen $V$ in diesem Fall einen \emph{G-Modul}.
\end{Def}

\begin{Def}
Sei $G$ eine Gruppe und $K$ ein Körper. Dann definieren wir die Gruppenalgebra $K G$ von $G$ über $K$ als
\begin{displaymath}
K G := \bigoplus_{g \in G} K g,
\end{displaymath}
in welcher die Multiplikation zweier Basis-Elemente von $K G$ gemäß der Multiplikation in $G$ erfolgt.
\end{Def}

\begin{Bem}
Die Gruppenalgebra $K G$ ist assoziativ und ihr neutrales Element ist das neutrale Element der Gruppe $G$ (als Element des Vektorraums $K G$).
\end{Bem}

\begin{Bem}
\label{bem:aequivalenz_darstellung_modul}
Sei $V$ im Sinne der Definition \ref{def:darstellung} ein G-Modul, dann wird $V$ mittels
\begin{displaymath}
\left(\sum_{g \in G} a_g g\right) v := \sum_{g \in G} a_g D\left(g\right) v
\end{displaymath}
für $v \in V$ und $a_g \in K$ zu einem $K G$-(Links-)Modul.

Sei umgekehrt $V$ ein (Links-)Modul der Gruppenalgebra $K G$. Dann erhalten wir eine Darstellung $D$ von $G$ über $K$ mittels
\begin{displaymath}
D\left(g\right) v := g v
\end{displaymath}
für $g \in G$ und $v \in V$. 

Um dies einzusehen, sei $g_1, g_2 \in G$ und $v \in V$.

\begin{eqnarray*}
D \left(g_2 g_1\right) v & = & \left(g_2 g_1\right) v \\
& = & g_2 \left(g_1 v\right) \\
& = & D\left(g_2\right) \left(D\left(g_1\right) v\right).
\end{eqnarray*}

Somit bildet $D$ einen (Gruppen-)Homomorphismus $G \rightarrow GL\left(V\right)$. Die Invertierbarkeit von $D\left(g\right)$ für alle $g \in G$ (womit $D\left(g\right)$ für alle $g \in G$ tatsächlich in $GL\left(V\right)$ liegt) folgt nach wohlbekanntem Beweisschema:

Weil das neutrale Element von $G$ (als Element von $K G$) das Einselement der Gruppenalgebra $K G$ bildet, folgt aus Eigenschaften von Linksmoduln:
\begin{eqnarray*}
D\left(1_G\right) v & = & 1_G v \\
& = & v,
\end{eqnarray*}
für alle $v \in V$, also wird das $1_G$ auf das neutrale Element von $GL\left(V\right)$ abgebildet.

Nun rechne für $g \in G$:
\begin{eqnarray*}
D\left(g^{-1}\right) D\left(g\right) & = &  D\left(g^{-1} g\right) \\
& = & D\left(1_G\right) \\
& = & 1_{GL\left(V\right)},
\end{eqnarray*}
womit $D\left(g\right)$ für alle $g \in G$ invertierbar ist -- also tatsächlich in $GL\left(V\right)$ liegt.
\end{Bem}

\begin{Def}
\label{def:G-Menge-Modul}
Sei $X$ eine G-Menge und $K$ ein Körper. Dann definieren wir einen $K$-Vektorraum
\begin{displaymath}
V\left(X\right) := \bigoplus_{x \in X} K v_{x}
\end{displaymath}
der Dimension $\left|X\right|$.

Durch
\begin{displaymath}
g v_{x} = v_{g x}
\end{displaymath}
für $g\in G$ und $x \in X$ wird $V\left(X\right)$ zu einem $K G$-Modul.
\end{Def}

\begin{Bem}
Sei $X = \mathop{\dot{\bigcup}}_{j=1}^t X_j$ mit $X_j$ G-Bahnen wie in Bemerkung \ref{bem:gbahnen}. Dann ist
\begin{displaymath}
V\left(X\right) := \bigoplus_{j=1}^t K V\left(X_j\right)
\end{displaymath}
eine Zerlegung von $V\left(X\right)$ als G-Modul.
\end{Bem}

\subsection{Anwendung auf $P^{K_n, symm, 0}$ und $P^{K_n, symm, xy}$}

Sei $S_n$ die symmetrische Gruppe (also Menge der bijektiven Abbildungen $\left[n\right] \rightarrow \left[n\right]$ unter der Verknüpfung
\begin{displaymath}
\left(\varphi_2 \varphi_1\right) \left(x\right) = \varphi_2 \left(\varphi_1 \left(x\right)\right) 
\end{displaymath}
für $\varphi_1, \varphi_2 \in S_n$ und $x\in \left[n\right]$).

Durch
\begin{eqnarray*}
\varphi \left\{v_1, v_2\right\} & := & \left\{\varphi\left(v_1\right), \varphi\left(v_2\right)\right\} \\
\varphi \left(v_1, v_2, v_3\right) & := & \left(\varphi\left(v_1\right), \varphi\left(v_2\right), \varphi\left(v_3\right)\right) \\
\varphi \left(\left\{v_1, v_2\right\}, v_3\right) & := & \left(\left\{\varphi\left(v_1\right), \varphi\left(v_2\right)\right\}, \varphi\left(v_3\right)\right)
\end{eqnarray*}
mit $\varphi \in S_n$ und $v_1, v_2, v_3$ paarweise verschieden werden ${\left[n\right] \choose 2}$, $\left[n\right]^{\underline{3}}$ und $EV_n$ zu $S_n$-Mengen.

Mittels Bemerkung \ref{bem:produkt_G-Mengen} werden somit ${\left[n\right] \choose 2} \times \left[n\right]^{\underline{3}}$ und ${\left[n\right] \choose 2} \times EV_n$ zu $S_n$-Mengen.

Nach Bemerkung \ref{bem:aequivalenz_darstellung_modul} und Definition \ref{def:G-Menge-Modul} induziert dies $S_n$-Darstellungen $D^{{\left[n\right] \choose 2} \times \left[n\right]^{\underline{3}}}$ bzw. $D^{{\left[n\right] \choose 2} \times EV_n}$
auf dem $\mathbb{R}^{{\left[n\right] \choose 2} \times \left[n\right]^{\underline{3}}}$ bzw.$\mathbb{R}^{{\left[n\right] \choose 2} \times EV_n}$.

Den Zusammenhang zu den Ecken von $P^{K_n, spt, symm, 0}$ bzw. $P^{K_n, spt, symm, xy}$\footnote{diese Aussage lässt sich leicht auf die anderen symmetrischen erweiterten Formulierungen erweitern} liefert folgendes Lemma:

\begin{Le}
Sei $\underbrace{\left(
\begin{array}{c}
x \\
z
\end{array}
\right)}_{=: \widehat{x}}$ bzw. $\underbrace{\left(
\begin{array}{c}
x \\
y
\end{array}
\right)}_{=: \widehat{x}}$ eine Ecke von $P^{K_n, spt, symm, 0}$ bzw. $P^{K_n, spt, symm, xy}$ und sei $\varphi \in S_n$.

Dann ist auch $\varphi \widehat{x}$\footnote{$\varphi \widehat{x}$ ist gemäß Bemerkung \ref{bem:aequivalenz_darstellung_modul} als Abkürzung für $D^{{\left[n\right] \choose 2} \times \left[n\right]^{\underline{3}}}\left(\varphi\right) \widehat{x}$ bzw. $D^{{\left[n\right] \choose 2} \times EV_n}\left(\varphi\right) \widehat{x}$ definiert} eine Ecke von $P^{K_n, spt, symm, 0}$ bzw. $P^{K_n, spt, symm, xy}$.
\end{Le}
\begin{bew}
Aus der in Lemma \ref{le:spt_erweitert_dual2} bzw. Definition \ref{def:P_p^symm_xy} gegebenen Gleichungs-/Unglei\-chungs\-beschrei\-bung von $P^{K_n, spt, symm, 0}$ bzw. $P^{K_n, spt, symm, xy}$ folgt, dass wenn $a^T \widehat{x} = b$ bzw. $a^T \widehat{x} \leq b$ eine für $P^{K_n, spt, symm, 0}$ bzw. $P^{K_n, spt, symm, xy}$ gültige Gleichung bzw. Ungleichung ist, auch $a^T \left(\varphi \widehat{x}\right) = b$ bzw. $a^T \left(\varphi \widehat{x}\right) \leq b$ eine für $P^{K_n, spt, symm, 0}$ bzw. $P^{K_n, spt, symm, xy}$ gültige Gleichung bzw. Ungleichung darstellt.

Der Rest folgt aus Satz \ref{sa:ecke_polyeder}.
\end{bew}

Wenn wir somit eine Ecke gefunden haben und ihre $S_n$-Bahn weitere Elemente enthält, so wissen wir, dass diese ebenfalls Ecken bilden.

\section{W-Bäume und W-Wege}

\begin{Def}
Sei $V := \mathcal{V} \mathbin{\dot{\cup}} \underbrace{\left(\bigcup_{i=1}^k W_i\right)}_{=: \mathcal{W}} $ mit $\left|W_i \cap W_j\right| \leq 1$ für $i \neq j$ und $\left|W_i\right| \geq 3$ für alle $i\in \left[k\right]$.

Es sei explizit klargestellt, dass $\mathcal{V}$ auch leer sein darf.

$\left(V, E\right)$ heißt ein \emph{W-Graph}, wenn $\bigcup_{i=1}^k {W_i \choose 2} \subseteq E$ ist.
\end{Def}

\begin{Def}
Ein W-Weg in einem W-Graphen $\left(V, E\right)$ ist ein Weg, welcher für jedes $W_i \in \mathcal{W}$ höchstens eine Kante aus $W_i \choose 2$ benutzt.
\end{Def}

\begin{Def}
Ein W-Graph $\left(V, E\right)$ ist ein \emph{W-Baum}, wenn es zwischen je zwei Knoten $u_1, u_2 \in V$ einen eindeutigen W-Weg gibt.

Seine  Knotenmenge wollen wir mit $w_W^V\left(u_1, u_2\right)$ und 
seine Kantenmenge wir mit $w_W^E\left(u_1, u_2\right)$ bezeichnen.
\end{Def}

Um uns im Folgenden Schreibarbeit zu sparen (da in den nun auftretenden W-Bäumen zusätzliche Bedingungen gelten), wollen wir definieren:

\begin{Def}
Ein 4-W-Baum ist ein W-Baum mit der Eigenschaft, dass für alle $W_i \in \mathcal{W}$ gilt: $\left|W_i\right| = 4$.
\end{Def}

\section{Nicht-ganzzahlige Ecken von $P^{K_n, spt, symm, 0}$}

\subsection{Nicht-ganzzahlige Ecken von $P^{K_4, spt, symm, 0}$}

\label{sec:nicht_ganzz_P_p^4symm0}

\begin{Le}
\label{le:lgs_ecke}
Das homogene lineare Gleichungssystem
{
\allowdisplaybreaks
\begin{align}
x_{\left\{1, 2\right\}} + x_{\left\{1, 4\right\}} = & 0 \label{eq:1214} \\
x_{\left\{1, 2\right\}} + x_{\left\{2, 3\right\}} = & 0 \label{eq:1223} \\
x_{\left\{1, 2\right\}} + x_{\left\{1, 3\right\}} = & 0 \label{eq:1213} \\
x_{\left\{1, 3\right\}} + x_{\left\{3, 4\right\}} = & 0 \\
x_{\left\{1, 4\right\}} + x_{\left\{2, 4\right\}} = & 0 \\
x_{\left\{2, 3\right\}} + x_{\left\{2, 4\right\}} = & 0 \\
x_{\left\{1, 3\right\}} + x_{\left\{2, 3\right\}} = & 0 \label{eq:1323} \\
x_{\left\{2, 4\right\}} + x_{\left\{3, 4\right\}} = & 0 \\
x_{\left\{1, 4\right\}} + x_{\left\{3, 4\right\}} = & 0 \label{eq:1434}
\end{align}
}
hat als einzige Lösung die Nulllösung.
\end{Le}
\begin{bew}
Durch Addition von \eqref{eq:1223} und \eqref{eq:1434} folgt
\begin{equation}
x_{\left\{1, 2\right\}} + x_{\left\{2, 3\right\}} + x_{\left\{3, 4\right\}} + x_{\left\{1, 4\right\}} = 0. \label{eq:12233414}
\end{equation}

Wenn wir Gleichungen \eqref{eq:1223}, \eqref{eq:1213} und \eqref{eq:1323} aufaddieren, so erhalten wir
\begin{equation}
2 \left(x_{\left\{1, 2\right\}} + x_{\left\{1, 3\right\}} + x_{\left\{2, 3\right\}}\right) = 0. \label{eq:121323}
\end{equation}

Durch Addition von \eqref{eq:1213}, $2 \cdot $\eqref{eq:1434} und \eqref{eq:1323} und Subtraktion von \eqref{eq:1223} folgt
\begin{equation}
2 \left(x_{\left\{1, 3\right\}} + x_{\left\{1, 4\right\}} + x_{\left\{3, 4\right\}}\right) = 0. \label{eq:131434}
\end{equation}

Wenn wir \eqref{eq:121323} und \eqref{eq:131434} addieren und hiervon $4 \cdot$ \eqref{eq:12233414} abziehen, erhalten wir
\begin{align*}
2 x_{\left\{1, 3\right\}} = & 0,
\shortintertext{also}
x_{\left\{1, 3\right\}} = & 0.
\end{align*}

Hieraus folgt durch elementares iteratives Rückwärtseinsetzen, dass auch die restlichen Variablen gleich $0$ sind.
\end{bew}

\begin{Le}
\label{le:ecke_P_p^4,symm,0_1}
$\left(
\begin{array}{c}
x \\
z
\end{array}
\right)\in \mathbb{R}^{[4]\choose 2}\times \mathbb{R}^{[4]^{\underline{3}}}$ mit
\begin{align*}
x_{\left\{v_1, v_2\right\}} = & \frac{1}{2}\ \forall \left\{v_1, v_2\right\}\in {[4]\choose 2} \\
z_{\left(v_1, v_2\right), v_3} = & \begin{cases}
\frac{1}{2} & \left(1, 2, 3, 4\right) \mapsto \left(v_1, v_2, v_3, v_4\right)\textnormal{ gerade Permutation} \\
0 & \left(1, 2, 3, 4\right) \mapsto \left(v_1, v_2, v_3, v_4\right)\textnormal{ ungerade Permutation}
\end{cases}
\end{align*}
mit $\left(v_1, v_2, v_3, v_4\right) \in \left[4\right]^{\underline{4}}$ ist eine Ecke von $P^{K_4, spt, symm, 0}$.
\end{Le}
\begin{bew}
Wir müssen zeigen, dass wenn wir die $z_{\left(v_1, v_2\right), v_3}$ mit der Eigenschaft, dass $\left(1, 2, 3, 4\right) \mapsto \left(v_1, v_2, v_3, v_4\right)$ eine ungerade Permutation bildet, auf 0 setzen, die Werte der restlichen Variablen aus den $P^{K_4, spt, symm, 0}$ bestimmenden Gleichungen berechnet werden können.

Wenn wir in $\mathcal{I}^{n, symm, 0}_2$ diese Variablen auf $0$ setzen, erhalten wir
{
\allowdisplaybreaks
\begin{align*}
x_{\left\{1, 2\right\}} - z_{\left(1, 2\right), 3} = & 0 \\
x_{\left\{1, 2\right\}} - z_{\left(2, 1\right), 4} = & 0 \\
x_{\left\{1, 3\right\}} - z_{\left(3, 1\right), 2} = & 0 \\
x_{\left\{1, 3\right\}} - z_{\left(1, 3\right), 4} = & 0 \\
x_{\left\{1, 4\right\}} - z_{\left(1, 4\right), 2} = & 0 \\
x_{\left\{1, 4\right\}} - z_{\left(4, 1\right), 3} = & 0 \\
x_{\left\{2, 3\right\}} - z_{\left(2, 3\right), 1} = & 0 \\
x_{\left\{2, 3\right\}} - z_{\left(3, 2\right), 4} = & 0 \\
x_{\left\{2, 4\right\}} - z_{\left(4, 2\right), 1} = & 0 \\
x_{\left\{2, 4\right\}} - z_{\left(2, 4\right), 3} = & 0 \\
x_{\left\{3, 4\right\}} - z_{\left(3, 4\right), 1} = & 0 \\
x_{\left\{3, 4\right\}} - z_{\left(4, 3\right), 2} = & 0 \\
x_{\left\{1, 2\right\}} + z_{\left(1, 4\right), 2} = & 1 \\
x_{\left\{1, 2\right\}} + z_{\left(2, 3\right), 1} = & 1 \\
x_{\left\{1, 3\right\}} + z_{\left(1, 2\right), 3} = & 1 \\
x_{\left\{1, 3\right\}} + z_{\left(3, 4\right), 1} = & 1 \\
x_{\left\{1, 4\right\}} + z_{\left(1, 3\right), 4} = & 1 \\
x_{\left\{1, 4\right\}} + z_{\left(4, 2\right), 1} = & 1 \\
x_{\left\{2, 3\right\}} + z_{\left(2, 4\right), 3} = & 1 \\
x_{\left\{2, 3\right\}} + z_{\left(3, 1\right), 2} = & 1 \\
x_{\left\{2, 4\right\}} + z_{\left(2, 1\right), 4} = & 1 \\
x_{\left\{2, 4\right\}} + z_{\left(4, 3\right), 2} = & 1 \\
x_{\left\{3, 4\right\}} + z_{\left(3, 2\right), 4} = & 1 \\
x_{\left\{3, 4\right\}} + z_{\left(4, 1\right), 3} = & 1.
\end{align*}
}

Dieses lineare Gleichungssystem vereinfacht man mittels elementarer Umformungen zu
{
\allowdisplaybreaks
\begin{align}
x_{\left\{1, 2\right\}} - z_{\left(1, 2\right), 3} = & 0 \nonumber \\
x_{\left\{1, 2\right\}} - z_{\left(2, 1\right), 4} = & 0 \nonumber \\
x_{\left\{1, 3\right\}} - z_{\left(3, 1\right), 2} = & 0 \nonumber \\
x_{\left\{1, 3\right\}} - z_{\left(1, 3\right), 4} = & 0 \nonumber \\
x_{\left\{1, 4\right\}} - z_{\left(1, 4\right), 2} = & 0 \nonumber \\
x_{\left\{1, 4\right\}} - z_{\left(4, 1\right), 3} = & 0 \nonumber \\
x_{\left\{2, 3\right\}} - z_{\left(2, 3\right), 1} = & 0 \nonumber \\
x_{\left\{2, 3\right\}} - z_{\left(3, 2\right), 4} = & 0 \nonumber \\
x_{\left\{2, 4\right\}} - z_{\left(4, 2\right), 1} = & 0 \nonumber \\
x_{\left\{2, 4\right\}} - z_{\left(2, 4\right), 3} = & 0 \nonumber \\
x_{\left\{3, 4\right\}} - z_{\left(3, 4\right), 1} = & 0 \nonumber \\
x_{\left\{3, 4\right\}} - z_{\left(4, 3\right), 2} = & 0 \nonumber \\
x_{\left\{1, 2\right\}} + x_{\left\{1, 4\right\}} = & 1 \label{eq:ecke1_fst} \\
x_{\left\{1, 2\right\}} + x_{\left\{2, 3\right\}} = & 1 \\
x_{\left\{1, 3\right\}} + x_{\left\{1, 2\right\}} = & 1 \\
x_{\left\{1, 3\right\}} + x_{\left\{3, 4\right\}} = & 1 \\
x_{\left\{1, 4\right\}} + x_{\left\{1, 3\right\}} = & 1 \\
x_{\left\{1, 4\right\}} + x_{\left\{2, 4\right\}} = & 1 \\
x_{\left\{2, 3\right\}} + x_{\left\{2, 4\right\}} = & 1 \\
x_{\left\{2, 3\right\}} + x_{\left\{1, 3\right\}} = & 1 \\
x_{\left\{2, 4\right\}} + x_{\left\{1, 2\right\}} = & 1 \\
x_{\left\{2, 4\right\}} + x_{\left\{3, 4\right\}} = & 1 \\
x_{\left\{3, 4\right\}} + x_{\left\{2, 3\right\}} = & 1 \\
x_{\left\{3, 4\right\}} + x_{\left\{1, 4\right\}} = & 1. \label{eq:ecke1_lst}
\end{align}
}
Nach Lemma \ref{le:lgs_ecke} ist die Lösung des aus den Gleichungen \eqref{eq:ecke1_fst} bis \eqref{eq:ecke1_lst} bestehenden linearen Gleichungssystems -- sofern sie existiert -- eindeutig bestimmt (man beachte, dass gegenüber dem linearen Gleichungssystem aus Lemma \ref{le:lgs_ecke} einige Gleichungen hinzugefügt wurden -- dies kann jedoch nicht zu einem Rangabfall der Koeffizientenmaxtrix führen). Somit ist auch die Lösung des gesamten linearen Gleichungssystems -- sofern sie existiert -- eindeutig bestimmt.

Elementares Nachrechnen ergibt, dass die angegebene Variablenbelegung tatsächlich eine Lösung bildet.
\end{bew}

\begin{Le}
\label{le:ecke_P_p^4,symm,0_2}
$\left(
\begin{array}{c}
x \\
z
\end{array}
\right)\in \mathbb{R}^{[4]\choose 2}\times \mathbb{R}^{[4]^{\underline{3}}}$ mit
\begin{align*}
x_{\left\{v_1, v_2\right\}} = & \frac{1}{2}\ \forall \left\{v_1, v_2\right\}\in {[4]\choose 2} \\
z_{\left(v_1, v_2\right), v_3} = & \begin{cases}
\frac{1}{2} & 4 \notin \left\{v_1, v_2\right\} \wedge v_2 = v_1 +_{\left[3\right]} 1 \\
0 & 4 \notin \left\{v_1, v_2\right\} \wedge v_2 = v_1 -_{\left[3\right]} 1 \\
\frac{1}{2} & 4 \in \left\{v_1, v_2\right\} \wedge \left(1, 2, 3, 4\right) \mapsto \left(v_1, v_2, v_3, v_4\right)\textnormal{ gerade Permutation} \\
0 & 4 \in \left\{v_1, v_2\right\} \wedge \left(1, 2, 3, 4\right) \mapsto \left(v_1, v_2, v_3, v_4\right)\textnormal{ ungerade Permutation}
\end{cases}
\end{align*}
mit $\left(v_1, v_2, v_3, v_4\right) \in \left[4\right]^{\underline{4}}$ bildet eine Ecke von $P^{K_4, spt, symm, 0}$.
\end{Le}
\begin{bew}
Wir arbeiten analog zum Beweis von Lemma \ref{le:ecke_P_p^4,symm,0_2} und betrachten wieder das lineare Gleichungssystem für die von $0$ verschiedenen Variablen:
Wenn wir in $\mathcal{I}^{n, symm, 0}_2$ diese Variablen auf $0$ setzen, erhalten wir
{
\allowdisplaybreaks
\begin{align*}
x_{\left\{1, 2\right\}} - z_{\left(1, 2\right), 3} = & 0 \\
x_{\left\{1, 2\right\}} - z_{\left(1, 2\right), 4} = & 0 \\
x_{\left\{1, 3\right\}} - z_{\left(3, 1\right), 2} = & 0 \\
x_{\left\{1, 3\right\}} - z_{\left(3, 1\right), 4} = & 0 \\
x_{\left\{1, 4\right\}} - z_{\left(1, 4\right), 2} = & 0 \\
x_{\left\{1, 4\right\}} - z_{\left(4, 1\right), 3} = & 0 \\
x_{\left\{2, 3\right\}} - z_{\left(2, 3\right), 1} = & 0 \\
x_{\left\{2, 3\right\}} - z_{\left(2, 3\right), 4} = & 0 \\
x_{\left\{2, 4\right\}} - z_{\left(4, 2\right), 1} = & 0 \\
x_{\left\{2, 4\right\}} - z_{\left(2, 4\right), 3} = & 0 \\
x_{\left\{3, 4\right\}} - z_{\left(3, 4\right), 1} = & 0 \\
x_{\left\{3, 4\right\}} - z_{\left(4, 3\right), 2} = & 0 \\
x_{\left\{1, 2\right\}} + z_{\left(1, 4\right), 2} = & 1 \\
x_{\left\{1, 2\right\}} + z_{\left(2, 3\right), 1} = & 1 \\
x_{\left\{1, 3\right\}} + z_{\left(1, 2\right), 3} = & 1 \\
x_{\left\{1, 3\right\}} + z_{\left(3, 4\right), 1} = & 1 \\
x_{\left\{1, 4\right\}} + z_{\left(1, 2\right), 4} = & 1 \\
x_{\left\{1, 4\right\}} + z_{\left(4, 2\right), 1} = & 1 \\
x_{\left\{2, 3\right\}} + z_{\left(2, 4\right), 3} = & 1 \\
x_{\left\{2, 3\right\}} + z_{\left(3, 1\right), 2} = & 1 \\
x_{\left\{2, 4\right\}} + z_{\left(2, 3\right), 4} = & 1 \\
x_{\left\{2, 4\right\}} + z_{\left(4, 3\right), 2} = & 1 \\
x_{\left\{3, 4\right\}} + z_{\left(3, 1\right), 4} = & 1 \\
x_{\left\{3, 4\right\}} + z_{\left(4, 1\right), 3} = & 1.
\end{align*}
}

Auch dieses lineare Gleichungssystem lässt sich durch Substituieren der auftretenden $z$-Variablen vereinfachen, nämlich zu
{
\allowdisplaybreaks
\begin{align*}
x_{\left\{1, 2\right\}} - z_{\left(1, 2\right), 3} = & 0 \\
x_{\left\{1, 2\right\}} - z_{\left(1, 2\right), 4} = & 0 \\
x_{\left\{1, 3\right\}} - z_{\left(3, 1\right), 2} = & 0 \\
x_{\left\{1, 3\right\}} - z_{\left(3, 1\right), 4} = & 0 \\
x_{\left\{1, 4\right\}} - z_{\left(1, 4\right), 2} = & 0 \\
x_{\left\{1, 4\right\}} - z_{\left(4, 1\right), 3} = & 0 \\
x_{\left\{2, 3\right\}} - z_{\left(2, 3\right), 1} = & 0 \\
x_{\left\{2, 3\right\}} - z_{\left(2, 3\right), 4} = & 0 \\
x_{\left\{2, 4\right\}} - z_{\left(4, 2\right), 1} = & 0 \\
x_{\left\{2, 4\right\}} - z_{\left(2, 4\right), 3} = & 0 \\
x_{\left\{3, 4\right\}} - z_{\left(3, 4\right), 1} = & 0 \\
x_{\left\{3, 4\right\}} - z_{\left(4, 3\right), 2} = & 0 \\
x_{\left\{1, 2\right\}} + x_{\left\{1, 4\right\}} = & 1 \\
x_{\left\{1, 2\right\}} + x_{\left\{2, 3\right\}} = & 1 \\
x_{\left\{1, 3\right\}} + x_{\left\{1, 2\right\}} = & 1 \\
x_{\left\{1, 3\right\}} + x_{\left\{3, 4\right\}} = & 1 \\
x_{\left\{1, 4\right\}} + x_{\left\{1, 2\right\}} = & 1 \\
x_{\left\{1, 4\right\}} + x_{\left\{2, 4\right\}} = & 1 \\
x_{\left\{2, 3\right\}} + x_{\left\{2, 4\right\}} = & 1 \\
x_{\left\{2, 3\right\}} + x_{\left\{1, 3\right\}} = & 1 \\
x_{\left\{2, 4\right\}} + x_{\left\{2, 3\right\}} = & 1 \\
x_{\left\{2, 4\right\}} + x_{\left\{3, 4\right\}} = & 1 \\
x_{\left\{3, 4\right\}} + x_{\left\{1, 3\right\}} = & 1 \\
x_{\left\{3, 4\right\}} + x_{\left\{1, 4\right\}} = & 1.
\end{align*}
}

Wenn wir auftretende Doppelungen in den Gleichungen entfernen, so ergibt sich daraus
{
\allowdisplaybreaks
\begin{align}
x_{\left\{1, 2\right\}} - z_{\left(1, 2\right), 3} = & 0 \nonumber \\
x_{\left\{1, 2\right\}} - z_{\left(1, 2\right), 4} = & 0 \nonumber \\
x_{\left\{1, 3\right\}} - z_{\left(3, 1\right), 2} = & 0 \nonumber \\
x_{\left\{1, 3\right\}} - z_{\left(3, 1\right), 4} = & 0 \nonumber \\
x_{\left\{1, 4\right\}} - z_{\left(1, 4\right), 2} = & 0 \nonumber \\
x_{\left\{1, 4\right\}} - z_{\left(4, 1\right), 3} = & 0 \nonumber \\
x_{\left\{2, 3\right\}} - z_{\left(2, 3\right), 1} = & 0 \nonumber \\
x_{\left\{2, 3\right\}} - z_{\left(2, 3\right), 4} = & 0 \nonumber \\
x_{\left\{2, 4\right\}} - z_{\left(4, 2\right), 1} = & 0 \nonumber \\
x_{\left\{2, 4\right\}} - z_{\left(2, 4\right), 3} = & 0 \nonumber \\
x_{\left\{3, 4\right\}} - z_{\left(3, 4\right), 1} = & 0 \nonumber \\
x_{\left\{3, 4\right\}} - z_{\left(4, 3\right), 2} = & 0 \nonumber \\
x_{\left\{1, 2\right\}} + x_{\left\{1, 4\right\}} = & 1 \label{eq:ecke2_fst} \\
x_{\left\{1, 2\right\}} + x_{\left\{2, 3\right\}} = & 1 \\
x_{\left\{1, 3\right\}} + x_{\left\{1, 2\right\}} = & 1 \\
x_{\left\{1, 3\right\}} + x_{\left\{3, 4\right\}} = & 1 \\
x_{\left\{1, 4\right\}} + x_{\left\{2, 4\right\}} = & 1 \\
x_{\left\{2, 3\right\}} + x_{\left\{2, 4\right\}} = & 1 \\
x_{\left\{2, 3\right\}} + x_{\left\{1, 3\right\}} = & 1 \\
x_{\left\{2, 4\right\}} + x_{\left\{3, 4\right\}} = & 1 \\
x_{\left\{3, 4\right\}} + x_{\left\{1, 4\right\}} = & 1. \label{eq:ecke2_lst}
\end{align}
}
Nach Lemma \ref{le:lgs_ecke} ist die Lösung des aus den Gleichungen \eqref{eq:ecke2_fst} bis \eqref{eq:ecke2_lst} bestehenden linearen Gleichungssystems -- sofern sie existiert -- eindeutig bestimmt. Somit ist auch die Lösung des gesamten linearen Gleichungssystems -- sofern sie existiert -- eindeutig bestimmt.

Dass die angegebene Variablenbelegung tatsächlich eine Lösung bildet, folgt durch elementares Nachrechnen.
\end{bew}

Nun wollen wir uns anschauen, wie die $S_4$-Bahnen der Ecken aus Lemma \ref{le:ecke_P_p^4,symm,0_1} und Lemma \ref{le:ecke_P_p^4,symm,0_2} aussehen.

Man überprüft leicht, dass der Stabilisator der Ecke aus Lemma \ref{le:ecke_P_p^4,symm,0_1} die Gruppe $A_4$ ist -- was nach Bahnformel (Lemma \ref{le:bahnformel}) zu einer Bahnlänge von 2 korrespondiert.

Für die Ecke aus Lemma \ref{le:ecke_P_p^4,symm,0_2} erhalten wir als Stabilisator die Menge aller Permutationen $\pi$ mit $\sgn \pi = 1$ und $\pi(4)=4$ -- was äquivalent zur vom (3-)Zyklus $\left(1\ 2\ 3\right)$ erzeugten Untergruppe von $S_4$ ist.

Nach Bahnformel ist die Bahnlänge also 8 -- wir haben also 8 Ecken.

In \citep{sorgatz2011} wurden Computerberechnungen mit polymake \cite{polymake} dokumentiert, welche ergeben, dass alle Ecken von $P^{K_4, spt, symm, 0}$ Bilder der kanonischen Einbettung\footnote{für das Argument, warum man von \emph{der} kanonischen Einbettung sprechen kann, vgl. Kapitel \ref{sec:einb_standard_symm}} eines aufspannenden Baums von $K_4$ in $P^{K_4, spt, symm, 0}$ oder eine der genannten 10 nicht-ganzzahligen Ecken sind.

\subsection{Nicht-ganzzahlige Ecken von $P^{K_n, spt, symm, 0}$ für $n\geq 4$}

\begin{Def}
\label{def:nichtganzz_ecken}
Seien $\widehat{z}^1, \ldots, \widehat{z}^{10}\in \mathbb{R}^{[4]^{\underline{3}}}$ derart, dass
\begin{itemize}
\item $\widehat{z}^i \neq \widehat{z}^j$ für $i\neq j$
\item $\widehat{z}^i_{\left(v_1, v_2\right), v_3} \in \left\{0, \frac{1}{2}\right\}\ \forall 1 \leq i \leq 10, \left(v_1, v_2, v_3\right) \in [4]^{\underline{3}}$
\item Für $x\in \mathbb{R}^{[4]\choose 2}$ mit $x_{\left\{v_1, v_2\right\}} = \frac{1}{2}$ ist für alle $1\leq i \leq 10$ $\left(
\begin{array}{c}
x \\
\widehat{z}^i
\end{array}
\right)$ eine Ecke von $P^{K_4, spt, symm, 0}$.
\end{itemize}
\end{Def}

Aus den in Abschnitt \ref{sec:nicht_ganzz_P_p^4symm0} bewiesenen Sätzen folgt, dass solche $\widehat{z}^1, \ldots, \widehat{z}^{10}$ tatsächlich existieren.

\begin{Sa}
Sei $\left(\left[n\right], E\right)$ ein 4-W-Baum.

Weiterhin soll für jedes $W \in \mathcal{W}$ ein $j\left(W\right) \in\left[10\right]$ beliebig (aber fest) vorgegeben sein. Außerdem sei jedem $v \in \left[n\right]$ und $W \in \mathcal{W}$ mit $v \notin W$ ein $\sigma_{W}\left(v\right) \in \left\{-1, 1\right\}$ zugeordnet. 

Falls $\left|\mathcal{W}\right| \geq 1$ ist, so ist der folgendermaßen konstruierte Vektor $\left(
\begin{array}{c}
x \\
z
\end{array}
\right) \in \mathbb{R}^{n \choose 2} \times \mathbb{R}^{n^{\underline{3}}}$ eine nicht-ganzzahlige Ecke von $P^{K_n, spt, symm, 0}$:
{
\allowdisplaybreaks
\begin{align*}
x_{\left\{v_1, v_2\right\}} := & \begin{cases}
1 & \left\{v_1, v_2\right\} \in E \backslash \bigcup_{W \in \mathcal{W}} {W \choose 2} \\
\frac{1}{2} & \left\{v_1, v_2\right\} \in \bigcup_{W \in \mathcal{W}} {W \choose 2} \\
0 & \textnormal{sonst}
\end{cases} \\
z_{\left(v_1, v_2\right), v_3} := & \begin{cases}
\widehat{z}^{j\left(W\right)}_{\left(v_1, v_2\right), v_3} & \exists W \in \mathcal{W}: v_1, v_2, v_3 \in W \\
\frac{1}{2} & \exists W \in \mathcal{W}: v_1, v_2\in W \wedge v_3 \notin W \wedge \left\{v_1, v_2\right\} \in w_{\mathcal{W}}^E\left(v_1, v_3\right) \\
0 & \exists W \in \mathcal{W}: v_1, v_2\in W \wedge v_3 \notin W \wedge \left\{v_1, v_2\right\} \in w_{\mathcal{W}}^E\left(v_2, v_3\right) \\
1 & \left\{v_1, v_2\right\} \in E \backslash \bigcup_{W \in \mathcal{W}} {W \choose 2} \wedge \left\{v_1, v_2\right\} \in w_{\mathcal{W}}^E\left(v_1, v_3\right) \\
0 & \left\{v_1, v_2\right\} \in E \backslash \bigcup_{W \in \mathcal{W}} {W \choose 2} \wedge \left\{v_1, v_2\right\} \notin w_{\mathcal{W}}^E\left(v_1, v_3\right) \\
0 & \left\{v_1, v_2\right\} \notin E
\end{cases}
\end{align*}
}

Zur Konstruktion von $z_{\left(v_1, v_2\right), v_3}$ im verbleibenden Fall 
$\exists W \in \mathcal{W}: v_1, v_2\in W \wedge v_3 \notin W \wedge \left\{v_1, v_2\right\} \notin w_{\mathcal{W}}^E\left(v_1, v_3\right), w_W^E\left(v_2, v_3\right)$:

Bestehe $W$ aus den Knoten $w_1, \ldots, w_4$ (Nummerierung beliebig, aber fest).

Sei $v_1 = w_{i_1}$, $v_2 = w_{i_2}$, $v_3$ liege mit $w_{i_3}$ in $\left(\left[n\right], E \backslash {W \choose 2}\right)$ (wobei das $W$ wie in der Fallunterscheidung oben sei) in einer Zusammenhangskomponente und sei $w_{i_4} \in W$ mit $w_{i_4} \neq w_{i_1}, w_{i_2}, w_{i_3}$ (wobei $i_1, \ldots, i_4 \in \left[4\right]$ sei).

Setze dann
\begin{displaymath}
z_{\left(v_1, v_2\right), v_3} := \begin{cases}
\frac{1}{2} & \sgn \left(\left(w_1, w_2, w_3, w_4\right) \mapsto \left(w_{i_1}, w_{i_2}, w_{i_3}, w_{i_4}\right)\right) = \sigma_W\left(v_3\right) \\
0 & \sgn \left(\left(w_1, w_2, w_3, w_4\right) \mapsto \left(w_{i_1}, w_{i_2}, w_{i_3}, w_{i_4}\right)\right) = -\sigma_W\left(v_3\right).
\end{cases}
\end{displaymath}
\end{Sa}
\begin{bew}
Wir müssen lediglich zeigen, dass für die genannten Punkte die Gleichungen
\begin{align}
x_{\left\{v, w\right\}} - z_{\left(v, w\right), u} - z_{\left(w, v\right), u} = & 0\ \forall \left(\left\{v, w\right\}, u\right)\in EV_n \label{eq:ecke_P_p^0_1} \\
x_{\left\{v, w\right\}} + \sum_{u\in[n]\backslash\left\{v, w\right\}} z_{\left(v, u\right), w} = & 1\ \forall \left(v, w\right)\in [n]^{\underline{2}} \label{eq:ecke_P_p^0_2}
\end{align}
erfüllt sind ($z_{\left(v, w\right), u} \geq 0 \ \forall \left(v, w, u\right) \in \left[n\right]^{\underline{3}}$ gilt trivialerweise), sowie dass aus dem Wert der Nullkomponenten die Werte der restlichen Komponenten folgen.

\paragraph{Schritt 1} $x_{\left\{v_1, v_2\right\}} = \frac{1}{2}$ für $\left\{v_1, v_2\right\} \in \bigcup_{W \in \mathcal{W}} {W \choose 2}$, $z_{\left(v_1, v_2\right), v_3} = \widehat{z}^{j}_{\left(v_1, v_2\right), v_3}$ für $\left(v_1, v_2, v_3\right) \in \bigcup_{W \in \mathcal{W}} W^{\underline{3}}$, Gleichung \eqref{eq:ecke_P_p^0_1} gilt für $\left(\left\{v, w\right\}, u\right)\in EV_n$ mit $\exists W \in \mathcal{W}: v, w, u \in W$ und Gleichung \eqref{eq:ecke_P_p^0_2} gilt für $\left(v, w\right)\in \bigcup_{W \in \mathcal{W}} W^{\underline{2}}$

Dies folgt aus Definition \ref{def:nichtganzz_ecken} und den dazu vorhergehenden Sätzen.

\paragraph{Schritt 2} $x_{\left\{v_1, v_2\right\}} = 1$ für $\left\{v_1, v_2\right\} \in E \backslash \bigcup_{W \in \mathcal{W}} {W \choose 2}$ und für beliebiges $v_3 \in \left[n\right]$ gilt:
\begin{displaymath}
z_{\left(v_1, v_2\right), v_3} = \begin{cases}
1 & \left\{v_1, v_2\right\} \in w_{\mathcal{W}}^E\left(v_1, v_3\right) \\
0 & \left\{v_1, v_2\right\} \notin w_{\mathcal{W}}^E\left(v_1, v_3\right).
\end{cases}
\end{displaymath}
Außerdem: Gleichung \eqref{eq:ecke_P_p^0_1} gilt für $\left(\left\{v, w\right\}, u\right)\in EV_n$ mit $\left\{v, w\right\} \in E \backslash \bigcup_{W \in \mathcal{W}} {W \choose 2}$ und Gleichung \eqref{eq:ecke_P_p^0_2} gilt für $\left(v, w\right)$ mit $\left\{v, w\right\} \in E \backslash \bigcup_{W \in \mathcal{W}} {W \choose 2}$.

Betrachten wir dazu Gleichung \eqref{eq:ecke_P_p^0_2} für genannte $\left(v, w\right) = \left(v_1, v_2\right)$. Wegen $z_{\left(v_1, u\right), v_2} = 0$ für beliebige $u \in \left[n\right] \backslash \left\{v_1, v_2\right\}$ muss $x_{\left\{v_1, v_2\right\}} = 1$ sein.

Sei $u \in \left[n\right] \backslash \left\{v_1, v_2\right\}$. Weil entweder $\left\{v_1, v_2\right\} \in w_{\mathcal{W}}^E\left(v_1, u\right)$ oder $\left\{v_1, v_2\right\} \in w_{\mathcal{W}}^E\left(v_2, u\right)$ gelten muss, nimmt genau eines der  $z_{\left(v_1, v_2\right), u}$, $z_{\left(v_2, v_1\right), u}$ den Wert 0 an -- dieses ist bereits als Nullkomponente vorgegeben. Aus Gleichung \eqref{eq:ecke_P_p^0_1} folgt, dass das andere der $z_{\left(v_1, v_2\right), u}$, $z_{\left(v_2, v_1\right), u}$ den Wert $1$ annimmt.

\paragraph{Schritt 3} Gleichung \eqref{eq:ecke_P_p^0_1} gilt für $\left(\left\{v, w\right\}, u\right)\in EV_n$ mit $\left\{v, w\right\} \in {\left[n\right] \choose 2} \backslash \bigcup_{W \in \mathcal{W}} {W \choose 2}$ und Gleichung \eqref{eq:ecke_P_p^0_2} gilt für $\left(v, w\right)$ mit $\left\{v, w\right\} \in {\left[n\right] \choose 2} \backslash \bigcup_{W \in \mathcal{W}} {W \choose 2}$ unter der Voraussetzung, dass die erste Kante des W-Weges von $v$ nach $w$ nicht in $\bigcup_{W \in \mathcal{W}} {W \choose 2}$ liegt.

Wenn $\left\{v, w\right\} \in E$ ist, folgt die Aussage bereits aus Schritt 2. Daher können wir ohne Einschränkung voraussetzen: $\left\{v, w\right\} \notin E$.

Dann ist $x_{\left\{v, w\right\}} = z_{\left(v, w\right), u} = z_{\left(w, v\right), u} = 0$ und Gleichung \eqref{eq:ecke_P_p^0_1} gilt trivialerweise.

Außerdem gibt es genau einen Knoten $w^*$ mit $\left\{v, w^*\right\} \in w_W^E\left(v, w\right)$. Für $u \neq w^*$ gilt $z_{\left(v, u\right), w} = 0$. Aus Gleichung \eqref{eq:ecke_P_p^0_2} folgt somit $z_{\left(v, w^*\right), w} = 1$, womit außerdem Gleichung \eqref{eq:ecke_P_p^0_2} erfüllt ist.

\paragraph{Schritt 4} Sei $\left\{w, w'\right\} \in {W \choose 2}$ mit $W \in \mathcal{W}$. Außerdem sei $u \in \left[n\right] \backslash W$. Dann nimmt $z_{\left(w, w'\right), u}$ den im Satz angegebenen Wert an und Gleichung \eqref{eq:ecke_P_p^0_1} gilt für $\left(\left\{w, w'\right\}, u\right)$.

Wir wissen aus Schritt 1, dass $x_{\left\{w, w'\right\}} = \frac{1}{2}$ gilt. 

Falls $\left\{w, w'\right\} \in w_{\mathcal{W}}^E\left(w, u\right)$ ist, so wissen wir, dass $z_{\left(w', w\right), u} = 0$ ist. Somit muss, damit Gleichung \eqref{eq:ecke_P_p^0_1} erfüllt ist, $z_{\left(w, w'\right), u} = \frac{1}{2}$ sein.

Sei nun $w'' \in W$ von $w, w'$ verschieden. Dann ist nach Konstruktion genau eines der $z_{\left(w, w''\right), u}$ bzw. $z_{\left(w'', w\right), u}$ gleich 0. Somit muss (wieder wegen $x_{\left\{w, w'\right\}} = \frac{1}{2}$ und Gleichung \eqref{eq:ecke_P_p^0_1}) das andere den Wert $\frac{1}{2}$ annehmen.

\paragraph{Schritt 5} Es existiere ein $W  \in \mathcal{W}$ mit $v_1 \in W$. Sei $v_2 \in \left[n\right] \backslash \left\{v_1\right\}$. Außerdem sei die erste Kante des W-Weges von $v_1$ nach $v_2$ aus $\bigcup_{W \in \mathcal{W}} {W \choose 2}$, wobei diese Kante von $\left\{v_1, v_2\right\}$ verschieden sei.
Dann ist Gleichung \eqref{eq:ecke_P_p^0_2} für $\left(v_1, v_2\right)$ erfüllt.

Sei $\left(v_1, w\right)$ diese erste Kante und seien $w', w''$ die verbleibenden beiden Knoten aus $W$.

Nach Schritt 4 ist $z_{\left(v_1, w\right), v_2} = \frac{1}{2}$ und genau eines der $z_{\left(v_1, w'\right), v_2}, z_{\left(v_1, w''\right), v_2}$ ist $\frac{1}{2}$, das andere $0$. Da die restlichen $z_{\left(v_1, u\right), v_2}$ mit $u \neq w, w', w''$ den Wert $0$ annehmen, ist alles gezeigt.
\end{bew}

\begin{Kor}
$P^{K_n, spt, symm, 0}$ ist für $n\geq 4$ \emph{kein} ganzzahliges Polytop.
\end{Kor}

\section{Nicht-ganzzahlige Ecken von $P^{K_n, spt, symm, xy}$}

\subsection{Ergebnisse von polymake-Computerberechnungen}

\label{sec:ergebnis_computerberechnungen}

Durch Computerberechnungen unter Verwendung des Programms polymake \cite{polymake} haben wir folgende 106 Ecken von $P^{K_4, spt, symm, xy}$ gefunden (Spalten in Reihenfolge $x_{\left\{1, 2\right\}}$, $x_{\left\{1, 3\right\}}$, $x_{\left\{1, 4\right\}}$, $x_{\left\{2, 3\right\}}$, $x_{\left\{2, 4\right\}}$, $x_{\left\{3, 4\right\}}$, $y_{\left\{1, 2\right\}, 3}$, $y_{\left\{1, 2\right\}, 4}$, $y_{\left\{1, 3\right\}, 2}$, $y_{\left\{1, 3\right\}, 4}$, $y_{\left\{1, 4\right\}, 2}$, $y_{\left\{1, 4\right\}, 3}$, $y_{\left\{2, 3\right\}, 1}$, $y_{\left\{2, 3\right\}, 4}$, $y_{\left\{2, 4\right\}, 1}$, $y_{\left\{2, 4\right\}, 3}$, $y_{\left\{3, 4\right\}, 1}$, $y_{\left\{3, 4\right\}, 2}$):
\begin{verbatim}
1 0 0 0 1 1 0 0 1 1 2 0 0 2 0 0 0 0
1 0 0 1 0 1 0 0 2 0 1 1 0 0 0 2 0 0
1 0 1 1 0 0 0 0 2 0 0 0 0 0 2 0 1 1
0 1 1 0 1 0 0 2 0 0 0 0 1 1 0 0 2 0
0 0 1 1 1 0 0 2 1 1 0 0 0 0 0 0 0 2
0 0 1 1 0 1 1 1 0 2 0 0 0 0 0 2 0 0
1 0 1 0 0 1 0 0 0 2 0 0 1 1 2 0 0 0
1 1 0 0 1 0 0 0 0 0 2 0 2 0 0 0 1 1
1 1 0 0 0 1 0 0 0 0 0 2 2 0 1 1 0 0
0 1 0 0 1 1 1 1 0 0 0 2 0 2 0 0 0 0
0 1 0 1 1 0 2 0 0 0 1 1 0 0 0 0 0 2
0 1 1 1 0 0 2 0 0 0 0 0 0 0 1 1 2 0

1 0 0 1 1 0 0 0 2 0 2 0 0 0 0 0 0 2
0 0 1 0 1 1 0 2 0 2 0 0 0 2 0 0 0 0
1 1 1 0 0 0 0 0 0 0 0 0 2 0 2 0 2 0
0 1 0 1 0 1 2 0 0 0 0 2 0 0 0 2 0 0

1/2 0 1 1/2 0 1 0 1 0 2 0 0 1 0 1/2 3/2 0 0
1 0 1/2 1 0 1/2 0 0 2 0 0 1 0 0 3/2 1/2 0 1
1 0 1/2 1 0 1/2 0 0 2 0 1 0 0 0 1/2 3/2 1 0
1/2 0 1 1/2 0 1 1 0 0 2 0 0 0 1 3/2 1/2 0 0
1/2 0 1/2 1 0 1 0 1 3/2 1/2 0 1 0 0 0 2 0 0
1 0 1 1/2 0 1/2 0 0 3/2 1/2 0 0 0 1 2 0 1 0
1 0 1 1/2 0 1/2 0 0 1/2 3/2 0 0 1 0 2 0 0 1
1/2 0 1/2 1 0 1 1 0 1/2 3/2 1 0 0 0 0 2 0 0
1/2 1/2 0 0 1 1 1 0 0 1 3/2 1/2 0 2 0 0 0 0
1/2 1/2 0 0 1 1 0 1 1 0 1/2 3/2 0 2 0 0 0 0
1 1 0 0 1/2 1/2 0 0 0 0 1/2 3/2 2 0 1 0 0 1
1 1 0 0 1/2 1/2 0 0 0 0 3/2 1/2 2 0 0 1 1 0
1/2 1 0 0 1/2 1 1 0 0 0 0 2 1/2 3/2 1 0 0 0
1/2 1 0 0 1/2 1 0 1 0 0 0 2 3/2 1/2 0 1 0 0
1 1/2 0 0 1 1/2 0 0 0 1 2 0 3/2 1/2 0 0 0 1
1 1/2 0 0 1 1/2 0 0 1 0 2 0 1/2 3/2 0 0 1 0
0 1/2 1/2 1 1 0 3/2 1/2 0 1 1 0 0 0 0 0 0 2
0 1/2 1/2 1 1 0 1/2 3/2 1 0 0 1 0 0 0 0 0 2
0 1 1 1/2 1/2 0 3/2 1/2 0 0 0 0 0 1 1 0 2 0
0 1 1 1/2 1/2 0 1/2 3/2 0 0 0 0 1 0 0 1 2 0
0 1 1/2 1 1/2 0 2 0 0 0 1 0 0 0 0 1 3/2 1/2
0 1 1/2 1 1/2 0 2 0 0 0 0 1 0 0 1 0 1/2 3/2
0 1/2 1 1/2 1 0 0 2 1 0 0 0 0 1 0 0 3/2 1/2
0 1/2 1 1/2 1 0 0 2 0 1 0 0 1 0 0 0 1/2 3/2

1/2 0 1 1 0 1/2 0 1 3/2 1/2 0 0 0 0 1/2 3/2 1 0
1 0 1/2 1/2 0 1 0 0 3/2 1/2 0 1 0 1 3/2 1/2 0 0
1/2 0 1 1 0 1/2 1 0 1/2 3/2 0 0 0 0 3/2 1/2 0 1
1/2 1 0 0 1 1/2 1 0 0 0 3/2 1/2 1/2 3/2 0 0 1 0
1 1/2 0 0 1/2 1 0 0 1 0 1/2 3/2 1/2 3/2 1 0 0 0
1/2 1 0 0 1 1/2 0 1 0 0 1/2 3/2 3/2 1/2 0 0 0 1
1 1/2 0 0 1/2 1 0 0 0 1 3/2 1/2 3/2 1/2 0 1 0 0
0 1/2 1 1 1/2 0 3/2 1/2 0 1 0 0 0 0 1 0 1/2 3/2
0 1 1/2 1/2 1 0 1/2 3/2 0 0 0 1 1 0 0 0 1/2 3/2
0 1 1/2 1/2 1 0 3/2 1/2 0 0 1 0 0 1 0 0 3/2 1/2
0 1/2 1 1 1/2 0 1/2 3/2 1 0 0 0 0 0 0 1 3/2 1/2
1 0 1/2 1/2 0 1 0 0 1/2 3/2 1 0 1 0 1/2 3/2 0 0

1/4 1 1/4 1/4 1 1/4 3/2 0 0 0 3/2 0 0 3/2 0 0 3/2 0
1/4 1/4 1 1 1/4 1/4 3/2 0 0 3/2 0 0 0 0 3/2 0 0 3/2
1/4 1/4 1 1 1/4 1/4 0 3/2 3/2 0 0 0 0 0 0 3/2 3/2 0
1/4 1 1/4 1/4 1 1/4 0 3/2 0 0 0 3/2 3/2 0 0 0 0 3/2
1 1/4 1/4 1/4 1/4 1 0 0 3/2 0 0 3/2 0 3/2 3/2 0 0 0
1 1/4 1/4 1/4 1/4 1 0 0 0 3/2 3/2 0 3/2 0 0 3/2 0 0

1 0 2/7 4/7 1/7 1 0 0 5/7 9/7 10/7 0 6/7 0 0 12/7 0 0
1 0 4/7 2/7 1/7 1 0 0 9/7 5/7 0 6/7 0 10/7 12/7 0 0 0
2/7 0 1 1 1/7 4/7 0 10/7 9/7 5/7 0 0 0 0 0 12/7 6/7 0
4/7 0 1 1 1/7 2/7 6/7 0 5/7 9/7 0 0 0 0 12/7 0 0 10/7
1 4/7 1/7 0 2/7 1 0 0 6/7 0 0 12/7 5/7 9/7 10/7 0 0 0
1 2/7 1/7 0 4/7 1 0 0 0 10/7 12/7 0 9/7 5/7 0 6/7 0 0
2/7 1 1/7 0 1 4/7 0 10/7 0 0 0 12/7 9/7 5/7 0 0 0 6/7
4/7 1 1/7 0 1 2/7 6/7 0 0 0 12/7 0 5/7 9/7 0 0 10/7 0
1/7 1 4/7 2/7 1 0 0 12/7 0 0 0 6/7 10/7 0 0 0 5/7 9/7
1/7 2/7 1 1 4/7 0 0 12/7 10/7 0 0 0 0 0 0 6/7 9/7 5/7
1/7 4/7 1 1 2/7 0 12/7 0 0 6/7 0 0 0 0 10/7 0 5/7 9/7
1/7 1 2/7 4/7 1 0 12/7 0 0 0 10/7 0 0 6/7 0 0 9/7 5/7
2/7 1/7 1 1 0 4/7 10/7 0 0 12/7 0 0 0 0 9/7 5/7 0 6/7
4/7 1/7 1 1 0 2/7 0 6/7 12/7 0 0 0 0 0 5/7 9/7 10/7 0
1 1/7 2/7 4/7 0 1 0 0 12/7 0 0 10/7 0 6/7 9/7 5/7 0 0
1 1/7 4/7 2/7 0 1 0 0 0 12/7 6/7 0 10/7 0 5/7 9/7 0 0
2/7 1 0 1/7 1 4/7 10/7 0 0 0 9/7 5/7 0 12/7 0 0 6/7 0
1 2/7 0 1/7 4/7 1 0 0 10/7 0 5/7 9/7 0 12/7 6/7 0 0 0
4/7 1 0 1/7 1 2/7 0 6/7 0 0 5/7 9/7 12/7 0 0 0 0 10/7
1 4/7 0 1/7 2/7 1 0 0 0 6/7 9/7 5/7 12/7 0 0 10/7 0 0
0 1 4/7 2/7 1 1/7 9/7 5/7 0 0 6/7 0 0 10/7 0 0 12/7 0
0 2/7 1 1 4/7 1/7 9/7 5/7 0 10/7 0 0 0 0 6/7 0 0 12/7
0 1 2/7 4/7 1 1/7 5/7 9/7 0 0 0 10/7 6/7 0 0 0 0 12/7
0 4/7 1 1 2/7 1/7 5/7 9/7 6/7 0 0 0 0 0 0 10/7 12/7 0

1 0 2/7 1 1/7 4/7 0 0 2 0 10/7 0 0 0 0 12/7 6/7 0
1 0 4/7 1 1/7 2/7 0 0 2 0 0 6/7 0 0 12/7 0 0 10/7
2/7 0 1 4/7 1/7 1 0 10/7 0 2 0 0 6/7 0 0 12/7 0 0
4/7 0 1 2/7 1/7 1 6/7 0 0 2 0 0 0 10/7 12/7 0 0 0
1 1 1/7 0 2/7 4/7 0 0 0 0 0 12/7 2 0 10/7 0 0 6/7
1 1 1/7 0 4/7 2/7 0 0 0 0 12/7 0 2 0 0 6/7 10/7 0
2/7 4/7 1/7 0 1 1 0 10/7 6/7 0 0 12/7 0 2 0 0 0 0
4/7 2/7 1/7 0 1 1 6/7 0 0 10/7 12/7 0 0 2 0 0 0 0
1/7 1 1 2/7 4/7 0 0 12/7 0 0 0 0 10/7 0 0 6/7 2 0
1/7 2/7 4/7 1 1 0 0 12/7 10/7 0 0 6/7 0 0 0 0 0 2
1/7 4/7 2/7 1 1 0 12/7 0 0 6/7 10/7 0 0 0 0 0 0 2
1/7 1 1 4/7 2/7 0 12/7 0 0 0 0 0 0 6/7 10/7 0 2 0
1 1/7 1 2/7 0 4/7 0 0 0 12/7 0 0 10/7 0 2 0 0 6/7
1 1/7 1 4/7 0 2/7 0 0 12/7 0 0 0 0 6/7 2 0 10/7 0
4/7 1/7 2/7 1 0 1 0 6/7 12/7 0 0 10/7 0 0 0 2 0 0
2/7 1/7 4/7 1 0 1 10/7 0 0 12/7 6/7 0 0 0 0 2 0 0
1 4/7 0 1/7 1 2/7 0 0 0 6/7 2 0 12/7 0 0 0 0 10/7
1 2/7 0 1/7 1 4/7 0 0 10/7 0 2 0 0 12/7 0 0 6/7 0
4/7 1 0 1/7 2/7 1 0 6/7 0 0 0 2 12/7 0 0 10/7 0 0
2/7 1 0 1/7 4/7 1 10/7 0 0 0 0 2 0 12/7 6/7 0 0 0
0 2/7 1 4/7 1 1/7 0 2 0 10/7 0 0 6/7 0 0 0 0 12/7
0 4/7 1 2/7 1 1/7 0 2 6/7 0 0 0 0 10/7 0 0 12/7 0
0 1 4/7 1 2/7 1/7 2 0 0 0 6/7 0 0 0 0 10/7 12/7 0
0 1 2/7 1 4/7 1/7 2 0 0 0 0 10/7 0 0 6/7 0 0 12/7
\end{verbatim}

Wir wollen diese Eckenserien Schritt für Schritt durchgehen. Die ersten beiden Eckenserien werden durch die aufspannenden Bäume induziert -- hierzu ist nichts weiter anzumerken.

Die folgenden Bemerkungen werden jeweils die Konstruktionen hinter den fünf verbleibenden Eckenserien darstellen. Sei dazu stets $\left(v_1, v_2, v_3, v_4\right) \in \left[4\right]^{\underline{4}}$.

\begin{Bem}
Setze
\begin{itemize}
\item $x_{\left\{v_1, v_2\right\}} = x_{\left\{v_2, v_3\right\}} = \frac{1}{2}$
\item $x_{\left\{v_3, v_4\right\}} = x_{\left\{v_1, v_4\right\}} = 1$
\item $x_{\left\{v_1, v_3\right\}} = x_{\left\{v_2, v_4\right\}} = 0$
\end{itemize}
und
\begin{itemize}
\item $y_{\left\{v_1, v_2\right\}, v_4} = 1$
\item $y_{\left\{v_1, v_3\right\}, v_4} = 2$
\item $y_{\left\{v_2, v_3\right\}, v_1} = 1$
\item $y_{\left\{v_2, v_4\right\}, v_1} = \frac{1}{2}$
\item $y_{\left\{v_2, v_4\right\}, v_3} = \frac{3}{2}$
\item Alle sonstigen $y$-Komponenten $0$
\end{itemize}

Da der Stabilistor der $S_4$-Operation auf dieser Ecke nur aus dem neutralen Element besteht, liefert dies nach Bahnformel (Lemma \ref{le:bahnformel}) 24 nicht-ganzzahlige Ecken von $P^{K_4, spt, symm, xy}$.
\end{Bem}

\begin{Bem}
Setze
\begin{itemize}
\item $x_{\left\{v_1, v_2\right\}} = x_{\left\{v_3, v_4\right\}} = \frac{1}{2}$
\item $x_{\left\{v_1, v_4\right\}} = x_{\left\{v_2, v_3\right\}} = 1$
\item $x_{\left\{v_1, v_3\right\}} = x_{\left\{v_2, v_4\right\}} = 0$
\end{itemize}
und
\begin{itemize}
\item $y_{\left\{v_1, v_2\right\}, v_4} = y_{\left\{v_3, v_4\right\}, v_1} = 1$
\item $y_{\left\{v_1, v_3\right\}, v_2} = y_{\left\{v_2, v_4\right\}, v_3} = \frac{3}{2}$
\item $y_{\left\{v_1, v_3\right\}, v_4} = y_{\left\{v_2, v_4\right\}, v_1} = \frac{1}{2}$
\item Alle sonstigen $y$-Komponenten $0$
\end{itemize}

Da der Stabilistor der $S_4$-Operation auf dieser Ecke durch $\left(v_1, v_2, v_3, v_4\right) \mapsto \left(v_4, v_3, v_2, v_1\right)$ erzeugt wird (Untergruppe der Ordnung 2), liefert dies nach Bahnformel (Lemma \ref{le:bahnformel}) 12 nicht-ganzzahlige Ecken von $P^{K_4, spt, symm, xy}$.
\end{Bem}

\begin{Bem}
Setze
\begin{itemize}
\item $x_{\left\{v_1, v_2\right\}} = x_{\left\{v_2, v_3\right\}} = x_{\left\{v_3, v_4\right\}} = x_{\left\{v_1, v_4\right\}} = \frac{1}{4}$
\item $x_{\left\{v_1, v_3\right\}} = x_{\left\{v_2, v_4\right\}} = 1$
\end{itemize}
und
\begin{itemize}
\item  $y_{\left\{v_1, v_2\right\}, v_3} = y_{\left\{v_2, v_3\right\}, v_4} = y_{\left\{v_3, v_4\right\}, v_1} = y_{\left\{v_4, v_1\right\}, v_2} = \frac{3}{2}$
\item Alle sonstigen $y$-Komponenten $0$
\end{itemize}

Da der Stabilistor der $S_4$-Operation auf dieser Ecke durch $\left(v_1, v_2, v_3, v_4\right) \mapsto \left(v_2, v_3, v_4, v_1\right)$ erzeugt wird (Untergruppe der Ordnung 4), liefert dies nach Bahnformel (Lemma \ref{le:bahnformel}) 6 nicht-ganzzahlige Ecken von $P^{K_4, spt, symm, xy}$.
\end{Bem}

\begin{Bem}
Setze
\begin{itemize}
\item $x_{\left\{v_1, v_2\right\}} = x_{\left\{v_3, v_4\right\}} = 1$
\item $x_{\left\{v_1, v_4\right\}} = \frac{2}{7}$
\item $x_{\left\{v_2, v_3\right\}} = \frac{4}{7}$
\item $x_{\left\{v_2, v_4\right\}} = \frac{1}{7}$
\item $x_{\left\{v_1, v_3\right\}} = 0$
\end{itemize}
und
\begin{itemize}
\item $y_{\left\{v_1, v_3\right\}, v_2} = \frac{5}{7}$
\item $y_{\left\{v_1, v_3\right\}, v_4} = \frac{9}{7}$
\item $y_{\left\{v_1, v_4\right\}, v_2} = \frac{10}{7}$
\item $y_{\left\{v_2, v_3\right\}, v_1} = \frac{6}{7}$
\item $y_{\left\{v_2, v_4\right\}, v_3} = \frac{12}{7}$
\item Alle sonstigen $y$-Komponenten $0$
\end{itemize}

Da der Stabilistor der $S_4$-Operation auf dieser Ecke nur aus dem neutralen Element besteht, liefert dies nach Bahnformel (Lemma \ref{le:bahnformel}) 24 nicht-ganzzahlige Ecken von $P^{K_4, spt, symm, xy}$.
\end{Bem}

\begin{Bem}
Setze
\begin{itemize}
\item $x_{\left\{v_1, v_2\right\}} = x_{\left\{v_2, v_3\right\}} = 1$
\item $x_{\left\{v_1, v_4\right\}} = \frac{2}{7}$
\item $x_{\left\{v_2, v_4\right\}} = \frac{1}{7}$
\item $x_{\left\{v_3, v_4\right\}} = \frac{4}{7}$
\item $x_{\left\{v_1, v_3\right\}} = 0$
\end{itemize}
und
\begin{itemize}
\item $y_{\left\{v_1, v_3\right\}, v_2} = 2$
\item $y_{\left\{v_1, v_4\right\}, v_2} = \frac{10}{7}$
\item $y_{\left\{v_2, v_4\right\}, v_3} = \frac{12}{7}$
\item $y_{\left\{v_3, v_4\right\}, v_2} = \frac{6}{7}$
\item Alle sonstigen $y$-Komponenten $0$
\end{itemize}

Da der Stabilistor der $S_4$-Operation auf dieser Ecke nur aus dem neutralen Element besteht, liefert dies nach Bahnformel (Lemma \ref{le:bahnformel}) 24 nicht-ganzzahlige Ecken von $P^{K_4, spt, symm, xy}$.
\end{Bem}

Dies sind 16 den aufspannenden Bäumen zugeordnete Ecken und 90 nicht-ganz\-zahlige Ecken -- insgesamt 106 Stück.

\subsection{Existenz nicht-ganzzahliger Ecken von $P^{K_4, spt, symm, xy}$}

Nun wollen wir für eine  "`Eckenserie"'\footnote{Wir verwenden hier den Begriff an dieser Stelle in Anführungszeichen, weil dies bislang nur ein Ergebnis aus Computerberechnungen darstellt.} aus Abschnitt \ref{sec:ergebnis_computerberechnungen} zeigen, dass sie tatsächlich aus Ecken von $P^{K_4, spt, symm, xy}$ besteht.

\begin{Sa}
\label{sa:nichtganzz_ecke_P^xy}
Sei $\left(v_1, v_2, v_3, v_4\right) \in \left[4\right]^{\underline{4}}$.

$
\left(\begin{array}{c}
x \\
z
\end{array}\right)
$ mit
\begin{itemize}
\item $x_{\left\{v_1, v_2\right\}} = x_{\left\{v_2, v_3\right\}} = x_{\left\{v_3, v_4\right\}} = x_{\left\{v_1, v_4\right\}} = \frac{1}{4}$
\item $x_{\left\{v_1, v_3\right\}} = x_{\left\{v_2, v_4\right\}} = 1$
\end{itemize}
und
\begin{itemize}
\item  $y_{\left\{v_1, v_2\right\}, v_3} = y_{\left\{v_2, v_3\right\}, v_4} = y_{\left\{v_3, v_4\right\}, v_1} = y_{\left\{v_4, v_1\right\}, v_2} = \frac{3}{2}$
\item alle sonstigen $y$-Komponenten $0$
\end{itemize}
bildet eine Ecke von $P^{K_4, spt, symm, xy}$.
\end{Sa}
\begin{bew}
Dass der Punkt zulässig ist, lässt sich leicht nachrechnen. Daher müssen wir nur die Eindeutigkeit der Lösung des linearen Gleichungssystems
{
\allowdisplaybreaks
\begin{align*}
x_{\left\{v_1, v_2\right\}} + x_{\left\{v_1, v_3\right\}} + x_{\left\{v_2, v_3\right\}} - y_{\left\{v_1, v_2\right\}, v_3}
% - y_{\left\{v_1, v_3\right\}, v_2} - y_{\left\{v_2, v_3\right\}, v_1} 
 = & 0 \\
x_{\left\{v_1, v_2\right\}} + x_{\left\{v_1, v_4\right\}} + x_{\left\{v_2, v_4\right\}}
% - y_{\left\{v_1, v_2\right\}, v_4}
 - y_{\left\{v_1, v_4\right\}, v_2}
% - y_{\left\{v_2, v_4\right\}, v_1}
 = & 0 \\
x_{\left\{v_1, v_3\right\}} + x_{\left\{v_1, v_4\right\}} + x_{\left\{v_3, v_4\right\}}
% - y_{\left\{v_1, v_3\right\}, v_4} - y_{\left\{v_1, v_4\right\}, v_3}
 - y_{\left\{v_3, v_4\right\}, v_1} = & 0 \\
x_{\left\{v_2, v_3\right\}} + x_{\left\{v_2, v_4\right\}} + x_{\left\{v_3, v_4\right\}} - y_{\left\{v_2, v_3\right\}, v_4}
% - y_{\left\{v_2, v_4\right\}, v_3} - y_{\left\{v_3, v_4\right\}, v_2}
 = & 0 \\
2 x_{\left\{v_1, v_2\right\}} + y_{\left\{v_1, v_2\right\}, v_3} 
%+ y_{\left\{v_1, v_2\right\}, v_4} 
 = & 2 \\
2 x_{\left\{v_1, v_3\right\}} 
% + y_{\left\{v_1, v_3\right\}, v_2} + y_{\left\{v_1, v_3\right\}, v_4} 
 = & 2 \\
2 x_{\left\{v_2, v_3\right\}}
% + y_{\left\{v_2, v_3\right\}, v_1}
 + y_{\left\{v_2, v_3\right\}, v_4} = & 2 \\
2 x_{\left\{v_1, v_4\right\}} + y_{\left\{v_1, v_4\right\}, v_2}
% + y_{\left\{v_1, v_4\right\}, v_3}
 = & 2 \\
2 x_{\left\{v_2, v_4\right\}}
% + y_{\left\{v_2, v_4\right\}, v_1} + y_{\left\{v_2, v_4\right\}, v_3}
 = & 2 \\
2 x_{\left\{v_3, v_4\right\}} + y_{\left\{v_3, v_4\right\}, v_1}
% + y_{\left\{v_3, v_4\right\}, v_2}
 = & 2
\end{align*}
}
zeigen.

Wir sehen sofort, dass $x_{\left\{v_1, v_3\right\}} = x_{\left\{v_2, v_4\right\}}  = 1$ gelten muss. Dies eingesetzt vereinfacht sich das Gleichungssystem zu
{
\allowdisplaybreaks
\begin{align}
x_{\left\{v_1, v_2\right\}} + x_{\left\{v_2, v_3\right\}} - y_{\left\{v_1, v_2\right\}, v_3}
 = & -1 \label{eq:vtx_xy_123_1} \\
x_{\left\{v_1, v_2\right\}} + x_{\left\{v_1, v_4\right\}}
 - y_{\left\{v_1, v_4\right\}, v_2}
 = & -1 \\
x_{\left\{v_1, v_4\right\}} + x_{\left\{v_3, v_4\right\}}
 - y_{\left\{v_3, v_4\right\}, v_1} = & -1 \\
x_{\left\{v_2, v_3\right\}} + x_{\left\{v_3, v_4\right\}} - y_{\left\{v_2, v_3\right\}, v_4}
 = & -1 \label{eq:vtx_xy_234_1} \\
2 x_{\left\{v_1, v_2\right\}} + y_{\left\{v_1, v_2\right\}, v_3} 
 = & 2 \label{eq:vtx_xy_123} \\
2 x_{\left\{v_2, v_3\right\}}
 + y_{\left\{v_2, v_3\right\}, v_4} = & 2 \\
2 x_{\left\{v_1, v_4\right\}} + y_{\left\{v_1, v_4\right\}, v_2}
 = & 2 \\
2 x_{\left\{v_3, v_4\right\}} + y_{\left\{v_3, v_4\right\}, v_1}
 = & 2. \label{eq:vtx_xy_341}
\end{align}
}

Nun stellen wir die Gleichungen \eqref{eq:vtx_xy_123} bis \eqref{eq:vtx_xy_341} nach den $y$-Variablen um und setzen sie in die Gleichungen \eqref{eq:vtx_xy_123_1} bis \eqref{eq:vtx_xy_234_1} ein und erhalten
{
\allowdisplaybreaks
\begin{align*}
3 x_{\left\{v_1, v_2\right\}} + x_{\left\{v_2, v_3\right\}} = & 1 \\
x_{\left\{v_1, v_2\right\}} + 3 x_{\left\{v_1, v_4\right\}} = & 1 \\
x_{\left\{v_1, v_4\right\}} + 3 x_{\left\{v_3, v_4\right\}} = & 1 \\
3 x_{\left\{v_2, v_3\right\}} + x_{\left\{v_3, v_4\right\}} = & 1.
\end{align*}
}

Für dieses lineare Gleichungssystem rechnet man leicht nach, dass die Determinante der Koeffizientenmatrix von $0$ verschieden ist.
\end{bew}

\subsection{Nicht-ganzzahlige Ecken von $P^{K_n, spt, symm, xy}$ für $n \geq 4$}

Nun wollen wir zeigen, dass $P^{K_n, spt, symm, xy}$ für $n \geq 4$ kein ganzzahliges Polytop ist.

\begin{Sa}
\label{sa:P_p^symm, xy_nichtganzz}
Sei $T=\left\{\left[n\right], E\right)$ ein 4-W-Baum mit $\mathcal{W} = \left\{W\right\}$ und $W = \left\{w_1, \ldots, w_4\right\}$. Für alle $w \in W$ sei $\mathfrak{i}$ derart definiert, dass $w_i = w_{\mathfrak{i}\left(w\right)}$ für $i \in \left[4\right]$ sei.

Sei $\left(
\begin{array}{c}
x^4 \\
y^4
\end{array}
\right) \in \mathbb{R}^{\left[4\right] \choose 2} \times \mathbb{R}^{EV_4}$ eine Ecke von $P^{K_4, spt, symm, xy}$  mit
\begin{displaymath}
x^4_{\left\{1, 3\right\}} = x^4_{\left\{2, 4\right\}} = 1.
\end{displaymath}

Setze $\left(
\begin{array}{c}
x \\
y
\end{array}\right) \in \mathbb{R}^{n \choose 2} \times \mathbb{R}^{EV_n}$ als
{
\allowdisplaybreaks
\begin{align*}
x_{\left\{u_1, u_2\right\}} := & \begin{cases}
1 & \left\{u_1, u_2\right\} \in E \backslash {W \choose 2} \\
x^4_{\left\{\mathfrak{i}\left(u_1\right), \mathfrak{i}\left(u_2\right)\right\}} & \left\{u_1, u_2\right\} \in {W \choose 2} \\
0 & \textnormal{sonst}
\end{cases}
\shortintertext{und}
y_{\left\{t_1, t_2\right\}, u} := & \begin{cases}
y^{4}_{\left\{\mathfrak{i}\left(t_1\right), \mathfrak{i}\left(t_2\right)\right\}, \mathfrak{i}\left(u\right)} & t_1, t_2, u \in W \\
0 & t_1, t_2 \in W \wedge  u \notin W \\
\left|w_W^E\left(t_1, t_2\right) \cap \left\{\left\{t_1, u\right\}, \left\{t_2, u\right\}\right\}\right| & \left\{t_1, u\right\}, \left\{t_2, u\right\}, \left\{t_1, t_2\right\} \notin {W \choose 2} \\
x^{4}_{\left\{\mathfrak{i}\left(t_1\right), \mathfrak{i}\left(u\right)\right\}} + \left|w_W^E\left(t_1, t_2\right) \cap \left\{\left\{u, t_2\right\}\right\}\right| & t_1, u \in W \wedge t_2 \notin W \wedge u \in w_W^V\left(t_1, t_2\right) \\
x^{4}_{\left\{\mathfrak{i}\left(u\right), \mathfrak{i}\left(t_2\right)\right\}} + \left|w_W^E\left(t_1, t_2\right) \cap \left\{\left\{t_1, u\right\}\right\}\right| & t_2, u \in W \wedge t_1 \notin W \wedge u \in w_W^V\left(t_1, t_2\right).
\end{cases}
\end{align*}
}
Für die verbleibenden Fälle von Komponenten von $y$ sei $v_i \notin W$ und mit $w_i$ in $\left(\left[n\right], E \backslash {W \choose 2}\right)$ in einer Zusammenhangskomponente und sei $j \in \left\{i +_{\left[4\right]} 1, i +_{\left[4\right]} 3\right\}$. Setze damit
\begin{align*}
y_{\left\{v_i, w_{i+_{\left[4\right]} 2}\right\}, w_j} & = 0 \\
y_{\left\{v_i, w_j\right\}, w_{i+_{\left[4\right]} 2}} & = x^{4}_{\left\{i+_{\left[4\right]} 2, j\right\}} \\
y_{\left\{v_i, w_j\right\}, w_{j +_{\left[4\right]} 2}} & = 1 - x^{4}_{\left\{i+_{\left[4\right]} 2, j\right\}} - x^{4}_{\left\{i, j\right\}}.
\end{align*}

Dann ist $\left(
\begin{array}{c}
x \\
y
\end{array}\right)$ eine Ecke von $P^{K_n, spt, symm, xy}$.
\end{Sa}
\begin{bew}
Da offenbar $x, y \geq 0$ sind (der einzige Fall, wo dies nicht offensichtlich ist, wird weiter unten betrachtet), sind nur die Gleichungen 
\begin{align}
\sum_{i=1}^3 \left(x_{\left\{u_{i}, u_{i \stackrel{\left[3\right]}{+} 1}\right\}} -  y_{\left\{u_{i}, u_{i \stackrel{\left[3\right]}{+} 1}\right\}, u_{i \stackrel{\left[3\right]}{+} 2}}\right) & = 0\ \forall \left\{u_1, u_2, u_3\right\}\in {[n] \choose 3} \label{eq:toshow_nichtganzz_xy1} \\
2 x_{\left\{u_1, u_2\right\}} + \sum_{u\in[n]\backslash\left\{u_1, u_2\right\}} y_{\left\{u_1, u_2\right\}, u} & = 2\ \forall \left\{u_1, u_2\right\}\in {[n] \choose 2} \label{eq:toshow_nichtganzz_xy2}
\end{align}
zu untersuchen, um zu zeigen, dass der Punkt zulässig ist und aus dem Setzen der Nullkomponenten die restlichen Koordinaten eindeutig bestimmt sind.

\paragraph{Schritt 1}
Seien $\left(u_1, u_2, u\right) \in W^{\underline{3}}$. Dann sind  $x_{\left\{u_1, u_2\right\}}$ und $y_{\left\{u_1, u_2\right\}, u}$ eindeutig mit den im Satz dargelegten Werten bestimmt und \eqref{eq:toshow_nichtganzz_xy1} gilt für alle $\left\{u_1, u_2, u_3\right\}\in {W \choose 3}$, sowie \eqref{eq:toshow_nichtganzz_xy2} ist für alle $\left\{u_1, u_2\right\}\in {W \choose 2}$ erfüllt.

Dies folgt daraus, dass $\left(
\begin{array}{c}
x^{4} \\
y^{4}
\end{array}
\right)$ eine Ecke von $P^{K_4, spt, symm, xy}$ ist.

\paragraph{Schritt 2}
$x_{\left\{u_1, u_2\right\}} = 1$ für $\left\{u_1, u_2\right\} \in E \backslash {W \choose 2}$ und \eqref{eq:toshow_nichtganzz_xy2} gilt für $\left\{u_1, u_2\right\} \in E \backslash {W \choose 2}$ (also falls $u_1, u_2$ eine kombinatorische Distanz von 1 besitzen und diese eine Kante, die sie miteinander verbindet, nicht in $W \choose 2$ liegt).

Betrachte dazu \eqref{eq:toshow_nichtganzz_xy2} für $\left\{u_1, u_2\right\}$. Es gilt für alle $u \in \left[n\right] \backslash \left\{u_1, u_2\right\}$: $y_{\left\{u_1, u_2\right\}, u} = 0$. Hieraus folgen die Aussagen.

\paragraph{Schritt 3}
$y_{\left\{u_1, u_2\right\}, u} = 2$ für $u_1, u_2, u$ mit $\left\{u_1, u\right\}, \left\{u, u_2\right\} \in E \backslash {W \choose 2}$ und \eqref{eq:toshow_nichtganzz_xy2} gilt für $\left\{u_1, u_2\right\}$, die in $\left(\left[n\right], E \backslash  {W \choose 2}\right)$ eine kombinatorische Distanz von 2 haben.

Beachte dazu diese Gleichung für $\left\{u_1, u_2\right\}$. Da für alle $v \in \left[n\right] \backslash \left\{u_1, u_2, u\right\}$ gilt: $y_{\left\{u_1, u_2\right\}, v}  = 0$ sind und ebenso $x_{\left\{u_1, u_2\right\}} = 0$ ist, folgt die Aussage.

\paragraph{Schritt 4}
$y_{\left\{u_1, u_2\right\}, u} = 1$, wenn $\left|\left\{\left\{u_1, u\right\}, \left\{u, u_2\right\}\right\} \cap w_W^E\left(u_1, u_2\right)\right| = 1$ und $\left\{u_1, u\right\}, \left\{u, u_2\right\} \notin {W \choose 2}$. Gleichung \eqref{eq:toshow_nichtganzz_xy1} gilt für $\left\{u_1, u_2, u\right\}$ mit genannten Voraussetzungen und
Gleichung \eqref{eq:toshow_nichtganzz_xy2} gilt für $\left\{u_1, u_2\right\}$, deren Abstand im Sinne der Länge eines $W$-Weges mindestens 3 ist und im $u_1$-$u_2$-Weg weder als erste noch als letzte Kanten eine aus ${W \choose 2}$ enthalten.

Sei ohne Einschränkung $u$ zu $u_1$ adjazent. Der zu $u_2$ adjazente Knoten des $u_1$-$u_2$-Weges sei $\widetilde{u}$.

Betrachte Gleichung \eqref{eq:toshow_nichtganzz_xy2}. Diese reduziert sich zu
\begin{displaymath}
y_{\left\{u_1, u_2\right\}, u} + y_{\left\{u_1, u_2\right\}, \widetilde{u}} = 2.
\end{displaymath}

Betrachte nun Gleichung \eqref{eq:toshow_nichtganzz_xy1} für $\left\{u_1, u_2, u\right\}$. Diese reduziert sich zu
\begin{displaymath}
\underbrace{x_{\left\{u_1, u\right\}}}_{=1} - y_{\left\{u_1, u_2\right\}, u} = 0,
\end{displaymath}
woraus die Aussagen folgen.

\paragraph{Schritt 5} Sei $t_1, u \in W \wedge t_2 \notin W \wedge u \in w_W^V\left(t_1, t_2\right)$ und $\left\{u, t_2\right\} \notin {W \choose 2}$ (oder alternativ die Rollen von $t_1$ und $t_2$ vertauscht). Dann gilt $y_{\left\{t_1, t_2\right\}, u} = x^{4}_{\left\{\mathfrak{i}\left(t_1\right), \mathfrak{i}\left(u\right)\right\}} + \left|w_{\mathcal{W}}^E\left(t_1, t_2\right) \cap \left\{\left\{u, t_2\right\}\right\}\right|$ und für besagte $\left\{t_1, t_2, u\right\}$ gilt \eqref{eq:toshow_nichtganzz_xy1}.

Betrachte dazu die Gleichung \eqref{eq:toshow_nichtganzz_xy1}:
\begin{displaymath}
\underbrace{x_{\left\{t_1, t_2\right\}}}_{= 0} + \underbrace{x_{\left\{t_1, u\right\}}}_{= x^4_{\left\{\mathfrak{i}\left(t_1\right), \mathfrak{i}\left(u\right)\right\}}} + \underbrace{x_{\left\{t_2, u\right\}}}_{= \left|w_W^E\left(t_1, t_2\right) \cap \left\{\left\{u, t_2\right\}\right\}\right|} - y_{\left\{t_1, t_2\right\}, u} - \underbrace{y_{\left\{t_1, u\right\}, t_2}}_{= 0} - \underbrace{z_{\left\{t_2, u\right\}, t_1}}_{= 0} = 0,
\end{displaymath} 
woraus die Aussage offenbar folgt.

\paragraph{Schritt 6} Sei $i \in \left[4\right]$ und $j \in \left\{i +_{\left[4\right]} 1, i +_{\left[4\right]} 3\right\}$. Sei $w_i \in W$ (Benennung wie im Satz) und $v_i \in \left[n\right]$ derart, dass zwischen $w_i$ und $v_i$ ein W-Weg existiert, welcher keine Kante aus ${W \choose 2}$ benutzt.

Dann ist $y_{\left\{v_i, w_j\right\}, w_{i+_{\left[4\right]} 2}} = x^{4}_{\left\{i+_{\left[4\right]} 2, j\right\}}$ und Gleichung \eqref{eq:toshow_nichtganzz_xy1} gilt für $\left\{w_{i+_{\left[4\right]} 2}, v_i, w_j\right\}$.

Wir betrachten \eqref{eq:toshow_nichtganzz_xy1} für $\left\{w_{i+_{\left[4\right]} 2}, v_i, w_j\right\}$:
\begin{displaymath}
\underbrace{x_{\left\{w_{i+_{\left[4\right]} 2}, v_i\right\}}}_{= 0} + \underbrace{x_{\left\{w_{i+_{\left[4\right]} 2}, w_j\right\}}}_{= x^{4}_{\left\{i+_{\left[4\right]} 2, j\right\}}} + \underbrace{x_{\left\{v_i, w_j\right\}}}_{= 0}
- \underbrace{y_{\left\{w_{i+_{\left[4\right]} 2}, v_i\right\}, w_j}}_{= 0} - \underbrace{y_{\left\{w_{i+_{\left[4\right]} 2}, w_j\right\}, v_i}}_{= 0} - y_{\left\{v_i, w_j\right\}, w_{i+_{\left[4\right]} 2}} = 0,
\end{displaymath}
also
\begin{equation}
y_{\left\{v_i, w_j\right\}, w_{i+_{\left[4\right]} 2}} = x^{4}_{\left\{i+_{\left[4\right]} 2, j\right\}}. \label{eq:y_v_i, w_j__w_iplus2}
\end{equation}


\paragraph{Schritt 7} Sei $i \in \left[4\right]$ und $j \in \left\{i +_{\left[4\right]} 1, i +_{\left[4\right]} 3\right\}$. Sei $w_i \in W$ (Benennung wie im Satz) und $v_i \in \left[n\right]$ derart, dass zwischen $w_i$ und $v_i$ ein W-Weg existiert, welcher keine Kante aus ${W \choose 2}$ benutzt.

Dann gilt $y_{\left\{v_i, w_j\right\}, w_{j +_{\left[4\right]} 2}} = 1 - x^{4}_{\left\{i+_{\left[4\right]} 2, j\right\}} - x^{4}_{\left\{i, j\right\}}$ und Gleichung \eqref{eq:toshow_nichtganzz_xy2} gilt für $\left\{v_i, w_j\right\}$.

Wir haben bereits gezeigt, dass für $k \neq i$ gilt
\begin{displaymath}
y_{\left\{v_i, w_k\right\}, w_i} = x^{4}_{\left\{i, k\right\}} + \left|w_W^E\left(w_i, v_i\right) \cap \left\{\left\{w_i, v_i\right\}\right\}\right|
\end{displaymath}
gilt.

Nun unterscheiden wir zwei Fälle
\begin{enumerate}
\item $\left\{w_i, v_i\right\} \in w_W^E\left(w_j, v_i\right)$
\item $\left\{w_i, v_i\right\} \notin w_W^E\left(w_j, v_i\right)$
\end{enumerate}

Zu Fall 1:

Mit Gleichung \eqref{eq:toshow_nichtganzz_xy2} folgt
\begin{eqnarray*}
2 \underbrace{x_{\left\{v_i, w_j\right\}}}_{= 0} + \underbrace{y_{\left\{v_i, w_j\right\}, w_i}}_{= x^{4}_{\left\{i, j\right\}} + \underbrace{\left|w_W^E\left(w_j, v_i\right) \cap \left\{\left\{w_i, v_i\right\}\right\}\right|}_{=1}} + \underbrace{y_{\left\{v_i, w_j\right\}, w_{i +_{\left[4\right]} 2}}}_{\stackrel{\eqref{eq:y_v_i, w_j__w_iplus2}}{=} x^{4}_{\left\{i+_{\left[4\right]} 2, j\right\}}} + y_{\left\{v_i, w_j\right\}, w_{j +_{\left[4\right]} 2}} & & \\
+ \sum_{v \in \left[n\right] \backslash \left(W \mathbin{\dot{\cup}} v_i\right)} \underbrace{y_{\left\{v_i, w_j\right\}, v}}_{= 0} & = & 2,
\end{eqnarray*}
also
\begin{displaymath}
y_{\left\{v_i, w_j\right\}, w_{j +_{\left[4\right]} 2}} = 1 - x^{4}_{\left\{i+_{\left[4\right]} 2, j\right\}} - x^{4}_{\left\{i, j\right\}}.
\end{displaymath}

Zu Fall 2:

In diesem Fall gibt es ein $v_1^* \in V_1$ mit $\left\{v_1, v_1^*\right\} \in w_W^E\left(w_j, v_i\right)$ und wir haben (Gleichung \eqref{eq:toshow_nichtganzz_xy2})
\begin{eqnarray*}
2 & = & 2 \underbrace{x_{\left\{v_i, w_j\right\}}}_{= 0} + \underbrace{y_{\left\{v_i, w_j\right\}, w_i}}_{= x^{4}_{\left\{i, j\right\}} + \underbrace{\left|w_W^E\left(w_j, v_i\right) \cap \left\{\left\{w_i, v_i\right\}\right\}\right|}_{= 0}} + \underbrace{y_{\left\{v_i, w_j\right\}, w_{i +_{\left[4\right]} 2}}}_{\stackrel{\eqref{eq:y_v_i, w_j__w_iplus2}}{=} x^{4}_{\left\{i+_{\left[4\right]} 2, j\right\}}} \\
& & + y_{\left\{v_i, w_j\right\}, w_{j +_{\left[4\right]} 2}} + \underbrace{y_{\left\{v_i, w_j\right\}, v_i^*}}_{= 1} + \sum_{v \in \left[n\right] \backslash \left(W \mathbin{\dot{\cup}} v_i \mathbin{\dot{\cup}} v_i^*\right)} \underbrace{y_{\left\{v_i, w_j\right\}, v}}_{= 0},
\end{eqnarray*}
also ebenfalls
\begin{displaymath}
y_{\left\{v_i, w_j\right\}, w_{j +_{\left[4\right]} 2}} = 1 - x^{4}_{\left\{i+_{\left[4\right]} 2, j\right\}} - x^{4}_{\left\{i, j\right\}}.
\end{displaymath}

\paragraph{Schritt 8} $y_{\left\{v_i, w_j\right\}, w_{j +_{\left[4\right]} 2}} \geq 0$ -- die einzige nichttriviale Nichtnegativitätsrelation ($v_i$ und die $w$-Komponenten seien wie im Satz definiert).

Da
\begin{displaymath}
\sum_{\left\{u_1, u_2\right\} \in {W \choose 2}} x_{\left\{u_1, u_2\right\}} = 3
\end{displaymath}
ist und $x_{\left\{w_1, w_3\right\}} = x_{\left\{w_2, w_4\right\}} = 1$ sind, muss die Summe der restlichen Summanden gleich $1$ sein -- mit $x_{\left\{u_1, u_2\right\}} \geq 0 \ \forall \left\{u_1, u_2\right\} \in {\left[n\right] \choose 2}$ folgt für ein $v_i$, welches mit $w_i$ in $\left(\left[n\right], E \backslash {W \choose 2}\right)$ in einer Zusammenhangskomponente liegt:
\begin{eqnarray*}
y_{\left\{v_i, w_j\right\}, w_{j +_{\left[4\right]} 2}} & = & 1 - \underbrace{\left( x^{4}_{\left\{i+_{\left[4\right]} 2, j\right\}} 
+ x^{4}_{\left\{i, j\right\}}\right)}_{\leq 1} \\
& \geq & 0.
\end{eqnarray*}

Nun überprüfen wir, ob wir alle Gleichungen auf Gültigkeit überprüft haben. Für Gleichung \eqref{eq:toshow_nichtganzz_xy2} ist dies der Fall (Schritte 1, 2, 3, 4, 7) -- mit Ausnahme des Falls $\left\{v_1, v_2\right\}$, dass $v_1 \in  \left[n\right] \backslash W, v_2 \in W$ (oder umgekehrt) liegt und die letzte Kante des $v_1$-$v_2$-W-Weges in $\left\{\left\{w_1, w_3\right\}, \left\{w_2, w_4\right\}\right\}$ liegt. Hier rechnet man die Gültigkeit leicht nach.

Bleibt also Gleichung \eqref{eq:toshow_nichtganzz_xy1} zu überprüfen.

Für $\left\{u_1, u_2, u_3\right\} \in {W \choose 3}$ haben wir dies in Schritt 1 überprüft. Falls alle drei Paare $\left\{u_1, u_2\right\}$, $\left\{u_1, u_3\right\}$, $\left\{u_2, u_3\right\} \notin E$ sind, so sind für alle $i \in \left[3\right]$ $x_{\left\{u_i, u_{i +_{[3]} 1}\right\}} = y_{\left\{u_i, u_{i +_{[3]} 1}\right\}, u_{i +_{[3]} 2}} = 0$ und Gleichung \eqref{eq:toshow_nichtganzz_xy1} ist sicher erfüllt. Daher können wir im Folgenden voraussetzen, dass mindestens eine der Kanten in $E$ liegt. Dies sei ohne Einschränkung $\left\{u_1, u_2\right\}$.

Falls alle drei der potentiellen Kanten $\left\{u_1, u_2\right\}$, $\left\{u_1, u_3\right\}$, $\left\{u_2, u_3\right\}$ in ${\left[n\right] \choose 2} \backslash {W \choose 2}$ liegen, liegt entweder $u_2$ auf einem $u_1$-$u_3$-W-Weg oder $u_1$ auf einem $u_2$-$u_3$-W-Weg. Ohne Einschränkung gelte ersteres. Somit ist
\begin{itemize}
\item $x_{\left\{u_1, u_2\right\}} = 1$
\item $y_{\left\{u_1, u_3\right\}, u_2} = x_{\left\{u_1, u_2\right\}} + x_{\left\{u_2, u_3\right\}}$
\item $y_{\left\{u_1, u_2\right\}, u_3} = y_{\left\{u_2, u_3\right\}, u_1} = 0$
\end{itemize}
Es ist offensichtlich, dass dann Gleichung \eqref{eq:toshow_nichtganzz_xy1} erfüllt ist.

Nun zum Fall, dass zwei der Knoten (ohne Einschränkung $u_1$, $u_2$) in $W$ liegen. Den Fall, dass der dritte Knoten ebenfalls in $W$ liegt, haben wir bereits in Schritt 1 behandelt.

Ebenso wurde der Fall, dass es von $u_3$ zu $u_1$ oder $u_2$ einen W-Weg gibt, welcher keine Kante aus $W \choose 2$ enthält, in Schritt 5 behandelt. Somit steht nur noch der Fall an, dass $u_3$ weder mit $u_1$, noch mit $u_2$ in einem Teilbaum aus $\left(\left[n\right], E \backslash {W \choose 2}\right)$ liegt.

Hier unterscheiden wir zwei Fälle:
\begin{enumerate}
\item $\exists j: u_1 = w_j \wedge u_2 = w_{j +_{\left[4\right]} 2}$
\item $\not\exists j: u_1 = w_j \wedge u_2 = w_{j +_{\left[4\right]} 2}$
\end{enumerate}

Wir beachten, dass Fall 2 bereits in Schritt 6 behandelt wurde. Somit ist nur Fall 1 zu betrachten.

Zu Fall 1:

Wir rechnen nach:
\begin{eqnarray*}
0 & \stackrel{!}{=} & \underbrace{x_{\left\{u_1, u_2\right\}}}_{= x^{4}_{\left\{j, j +_{\left[4\right]} 2\right\}}} + \underbrace{x_{\left\{u_1, u_3\right\}}}_{= 0} + \underbrace{x_{\left\{u_2, u_3\right\}}}_{= 0} - \underbrace{y_{\left\{u_1, u_2\right\}, u_3}}_{= 0} - \underbrace{y_{\left\{u_1, u_3\right\}, u_2}}_{= \underbrace{y_{\left\{w_j, u_3\right\}, w_{j +_{\left[4\right]} 2}}}_{= 1 - x^{4}_{\left\{i+_{\left[4\right]} 2, j\right\}} - x^{4}_{\left\{i, j\right\}}}}  - \underbrace{y_{\left\{u_2, u_3\right\}, u_1}}_{= \underbrace{y_{\left\{w_{j +_{\left[4\right]} 2}, u_3\right\}, w_j}}_{= 1 - x^{4}_{\left\{i+_{\left[4\right]} 2, j +_{\left[4\right]} 2\right\}} - x^{4}_{\left\{i, j +_{\left[4\right]} 2\right\}}}}
\end{eqnarray*}
was also äquivalent zu
\begin{displaymath}
x^{4}_{\left\{j, j +_{\left[4\right]} 2\right\}} + x^{4}_{\left\{i+_{\left[4\right]} 2, j\right\}} + x^{4}_{\left\{i, j\right\}} + x^{4}_{\left\{i+_{\left[4\right]} 2, j +_{\left[4\right]} 2\right\}} + x^{4}_{\left\{i, j +_{\left[4\right]} 2\right\}} = 2
\end{displaymath}
ist.

Dies ist aber sicher der Fall, da die Summe über alle Komponenten von $x^{4}$ gleich 3 ist und $x^{4}_{\left\{i, i +_{\left[4\right]} 2\right\}} = 1$ ist.
\end{bew}

Da wegen Satz \ref{sa:nichtganzz_ecke_P^xy} $P^{K_4, spt, symm, xy}$ nicht-ganzzahlige Ecken besitzt, die die Voraussetzungen für Satz \ref{sa:P_p^symm, xy_nichtganzz} erfüllen, folgt:

\begin{Kor}
$P^{K_n, spt, symm, xy}$ ist für $n\geq 4$ \emph{kein} ganzzahliges Polytop.
\end{Kor}

\chapter{Die Einbettung der aufspannenden Bäume}

Bislang haben wir den charakteristischen Vektor $x:=\chi\left(T\right)$ der Kantenmenge eines aufspannenden Baums $T$ stets um
\begin{displaymath}
z_{\left(v, w\right), u} := \begin{cases}
1 & \left\{v, w\right\}\in T\wedge u\in R_{T\backslash \left\{\left\{v, w\right\}\right\}}\left(w\right)\\
0 & \textnormal{sonst}
\end{cases}
\end{displaymath}
ergänzt (analog für Belegung von $y$ im Falle von $P^{K_n, spt, symm, xy}$), um ihn in eine erweiterte Formulierung dieser Arbeit einzubetten.

In diesem Kapitel wollen wir untersuchen, ob auch eine andere Belegung von $z$ möglich ist, so dass $\left(
\begin{array}{c}
x \\
z
\end{array}
\right)$ im Polytop einer erweiterten Formulierung Spannbaum-Polytops liegt (wieder analog auch für $y$ im Falle von $P^{K_n, spt, symm, xy}$).

\section{Die Einbettung der aufspannenden Bäume in $P^{K_n, spt, symm, 0}$}
\label{sec:einb_standard_symm}

Als erstes werden wir dies für $P^{K_n, spt, symm, 0}$ (vergleiche Lemma \ref{le:spt_erweitert_dual2}) untersuchen.

\begin{Sa}
\label{sa:einbettung_normale_erweiterung}
Sei $T$ die Kantenmenge eines aufspannenden Baums von $K_n$.

Dann lässt sich zu $x:=\chi\left(T\right)$ ein eindeutiges $z$ konstruieren (nämlich 
\begin{equation}
z_{\left(v, w\right), u} := \begin{cases}
1 & \left\{v, w\right\}\in T\wedge u\in R_{T\backslash \left\{\left\{v, w\right\}\right\}}\left(w\right)\\
0 & \textnormal{sonst),}
\end{cases} \label{eq:z_belegung}
\end{equation}
so dass $
\left(
\begin{array}{c}
x \\
z
\end{array}
\right)
\in P^{K_n, spt, symm, 0}$ erfüllt ist.
\end{Sa}
\begin{bew}
Aus Gleichung \eqref{eq:spt_erweitert_primal2_1} folgt mit $z_{\left(v, w\right), u} \geq 0 \ \forall \left(v, w, u\right) \in \left[n\right]^{\underline{3}}$ für ein beliebiges $u \in \left[n\right] \backslash \left\{v, w\right\}$:
\begin{displaymath}
z_{\left(v, w\right), u} = 0 \textnormal{ für } \left\{v, w\right\}\notin T
\end{displaymath}
und 
\begin{displaymath}
z_{\left(v, w\right), u}+z_{\left(w, v\right), u} = 1 \textnormal{ für } \left\{v, w\right\}\in T.
\end{displaymath}

Somit bleibt nur zu zeigen, dass für $\left(v, w\right) \in \left[n\right]^{\underline{2}}$ mit $\left\{v, w\right\}$ in $T$, sowie $u \in \left[n\right] \backslash \left\{v, w\right\}$ \eqref{eq:z_belegung} gilt.

Definiere aus $T$ eine Arboreszenz $T_A$ durch Auswahl einer Wurzel in $T$ und Orientierung aller Bögen weg von dieser Wurzel. Setze
\begin{displaymath}
v\prec w \Leftrightarrow (w, v) \in T_A
\end{displaymath}
(der Grund, warum wir dies so setzen, ist der, dass wir wollen, dass Blätter keinen Vorgänger haben). Definiere $\prec^+$ als den transitiven (aber nicht reflexiven) Abschluss von $\prec$.

\paragraph{Behauptung} Für alle $\left(v, \widetilde{w}, \widetilde{u}\right)\in \left[n\right]^{\underline{3}}$ gilt:
\begin{align}
z_{\left(v, \widetilde{w}\right), \widetilde{u}} := \begin{cases}
1 & \left\{v, \widetilde{w}\right\}\in T\wedge \widetilde{u}\in R_{T\backslash \left\{\left\{v, \widetilde{w}\right\}\right\}}\left(\widetilde{w}\right)\\
0 & \textnormal{sonst}
\end{cases}
\label{eq:z_induktion}
\end{align}

Dies beweisen wir per vollständiger Induktion über $v$ durch Beweis folgender Aussagen:

\paragraph{Induktionsanfang} Für alle $v$, die Blätter sind, gilt  für alle $\left(\widetilde{w}, \widetilde{u}\right)\in \left(\left[n\right]\backslash \left\{v\right\}\right)^{\underline{2}}$ \eqref{eq:z_induktion}. \\

Sei $v$ in $T_A$ ein Blatt (d. h. $v$ hat in $T_A$ nur eingehende Kanten). Dann gibt es genau ein $w^*$ mit $\left\{v, w^*\right\}\in T$.

Sei $\left(\widetilde{w}, \widetilde{u}\right)\in \left(\left[n\right]\backslash \left\{v\right\}\right)^{\underline{2}}$ beliebig.

Im Fall $\widetilde{w}\neq w^*$ ist nichts zu zeigen, da dann $z_{\left(v, \widetilde{w}\right), \widetilde{u}} = 0$ gilt, da sicher $\left\{v, \widetilde{w}\right\}\notin T$ ist.

Sei somit $\widetilde{w} = w^*$ (und somit $\widetilde{u}\neq w^*$).

Mit Gleichung \eqref{eq:spt_erweitert_primal2_2} folgt:
\begin{eqnarray*}
1 & = & x_{\left\{v, \widetilde{u}\right\}} + \sum_{u\in[n]\backslash\left\{v, \widetilde{u}\right\}} z_{\left(v, u\right), \widetilde{u}} \\
& = & z_{\left(v, w^*\right), \widetilde{u}}
\end{eqnarray*}

Da $\widetilde{u}$ nach Konstruktion sicher mit $w^*$ in $T\backslash \left\{\left\{v, w^*\right\}\right\}$ in einer Zusammenhangskomponente liegt, beweist dies die Aussage.

\paragraph{Induktionsschritt} Sei $v\in \left[n\right]$ und \eqref{eq:z_induktion} gelte für alle $\left(v^*, \widetilde{w}, \widetilde{u}\right)$ mit $v^*\prec^+ v$ und $\left(\widetilde{w}, \widetilde{u}\right)\in \left(\left[n\right]\backslash \left\{v\right\}\right)^{\underline{2}}$. Dann gilt \eqref{eq:z_induktion} für $\left(v, \widetilde{w}, \widetilde{u}\right)$ mit $\left(\widetilde{w}, \widetilde{u}\right)\in \left(\left[n\right]\backslash \left\{v\right\}\right)^{\underline{2}}$. \\

Wie schon oben angemerkt, reicht es, den Fall $\left\{v, \widetilde{w}\right\}\in T$ zu betrachten.

\subparagraph{Fall 1:} $\left\{v, \widetilde{u}\right\}\in T$

Betrachte Gleichung \eqref{eq:spt_erweitert_primal2_2}
\begin{eqnarray*}
1 & = & x_{\left\{v, \widetilde{u}\right\}} + \sum_{u\in[n]\backslash\left\{v, \widetilde{u}\right\}} z_{\left(v, u\right), \widetilde{u}} \\
& = & 1 + \sum_{u\in[n]\backslash\left\{v, \widetilde{u}\right\}: \left\{v, u\right\}\in T} z_{\left(v, u\right), \widetilde{u}}
\end{eqnarray*}

Da $\left\{v, \widetilde{u}\right\}\in T$ ist, so folgt daraus:
\begin{displaymath}
\sum_{u\in[n]\backslash\left\{v, \widetilde{u}\right\}: \left\{v, u\right\}\in T} z_{\left(v, u\right), \widetilde{u}} = 0,
\end{displaymath}
also gilt mit $u:=\widetilde{w}$ unter Nutzung von $z\geq 0$:
\begin{displaymath}
z_{\left(v, \widetilde{w}\right), \widetilde{u}} = 0.
\end{displaymath}
Da $\widetilde{u}$ (wegen $\left\{v, \widetilde{u}\right\}\in T$) sicherlich nicht mit $\widetilde{w}$ in $T\backslash \left\{\left\{v, \widetilde{w}\right\}\right\}$ in einer Zusammenhangskomponente liegt, ist dies die zu beweisende Aussage.

\subparagraph{Fall 2:} $\left\{v, \widetilde{u}\right\}\notin T$ \\
Es gilt weiterhin $\left\{v, \widetilde{w}\right\}\in T$.

Wenn wir aus $T$ den Knoten $v$ (und alle seine inzidenten Kanten) entfernen, so zerfällt $T$ in (disjunkte) Bäume $T_{w_1}, \ldots, T_{w_k}$ mit $w_i\in T_{w_i}\wedge w_i\prec v\ \forall 1\leq i \leq k$, sowie (wenn $v$ nicht die Wurzel von $T_A$ ist) einen Baum $T_{w_{k+1}}$ mit $v\prec w^*$.

Sei (um im Folgenden hier keine Fallunterscheidung durchführen zu müssen)
\begin{displaymath}
\widetilde{k} := \begin{cases}
k+1 & \textnormal{für } v \textnormal{ ist nicht Wurzel von } T_A \\
k & \textnormal{für } v \textnormal{ ist Wurzel von } T_A \\
\end{cases}
\end{displaymath}

Es folgt aus Gleichung \eqref{eq:spt_erweitert_primal2_2}:
\begin{eqnarray}
1 & = & x_{\left\{v, \widetilde{u}\right\}} + \sum_{u\in[n]\backslash\left\{v, w\right\}} z_{\left(v, u\right), \widetilde{u}} \nonumber \\
& = & \sum_{i=1}^{\widetilde{k}} z_{\left(v, w_i\right), \widetilde{u}} \label{eq:case2sum}
\end{eqnarray}

\subparagraph{Fall 2a:} $\widetilde{u}\in T_{w_j}$ ($1\leq j \leq k$) und $\widetilde{w} = w_j$

Es gilt nach \eqref{eq:spt_erweitert_primal2_1} und wegen $w_j \prec^+ v$:
\begin{eqnarray}
z_{\left(v, w_j\right), \widetilde{u}} & = & x_{\left\{v, w_j\right\}} - z_{\left(w_j, v\right), \widetilde{u}} \nonumber \\
& \stackrel{\textnormal{Ind.-vor.}}{=} & 1. \label{eq:z_v_w_j_u1}
\end{eqnarray}

\subparagraph{Fall 2b:} $\widetilde{u}\in T_{w_j}$ ($1\leq j \leq k$) und $\widetilde{w} = w_l$ mit $1\leq l \leq k\wedge l\neq j$

Es gilt:
\begin{eqnarray*}
1 & \stackrel{\eqref{eq:case2sum}}{=} & \sum_{i=1}^{\widetilde{k}} z_{\left(v, w_i\right), \widetilde{u}} \\
& = & z_{\left(v, w_j\right), \widetilde{u}} + \sum_{i=1, i\neq j}^{\widetilde{k}} z_{\left(v, w_i\right), \widetilde{u}} \\
& \stackrel{\eqref{eq:z_v_w_j_u1}}{=} & 1 + \sum_{i=1, i\neq j}^{\widetilde{k}} z_{\left(v, w_i\right), \widetilde{u}}.
\end{eqnarray*}

Also ist für $\widetilde{w} = w_l$ wegen $z\geq 0$ $z_{\left(v, w_l\right), \widetilde{u}} = 0$ (was sicherlich korrekt ist, da $\widetilde{u}\notin R_{T\backslash \left\{\left\{v, w_l\right\}\right\}} \left(w_l\right)$ ist).

\subparagraph{Fall 2c:} $\widetilde{u}\in T_{w_{k+1}}$ und $\widetilde{w}=w_j$ mit $j\neq k+1$

Aus Gleichung \eqref{eq:spt_erweitert_primal2_1} folgt unter Beachtung von $w_j \prec^+ v$:
\begin{eqnarray}
z_{\left(v, w_j\right), \widetilde{u}} & = & x_{\left\{v, w_j\right\}} - z_{\left(w_j, v\right), \widetilde{u}} \nonumber \\
& \stackrel{\textnormal{Ind.-vor.}}{=} & 0.  \label{eq:z_v_w_j_u2}
\end{eqnarray}

\subparagraph{Fall 2d:} $\widetilde{u}\in T_{w_{k+1}}$ und $\widetilde{w} = w_{k+1}$

\begin{eqnarray*}
1 & \stackrel{\eqref{eq:case2sum}}{=} & \sum_{i=1}^{k+1} z_{\left(v, w_i\right), \widetilde{u}} \\
& = & z_{\left(v, w_{k+1}\right), \widetilde{u}} + \sum_{i=1}^{k} z_{\left(v, w_i\right), \widetilde{u}} \\
& \stackrel{\eqref{eq:z_v_w_j_u2}}{=} & z_{\left(v, w_{k+1}\right), \widetilde{u}}.
\end{eqnarray*}
\end{bew}

\section{Die Einbettung der aufspannenden Bäume in $P^{K_n, spt, symm, xy}$}

\begin{Sa}
Sei $T$ die Kantenmenge eines aufspannender Baum von $K_n$.

Dann lässt sich zu $x:=\chi\left(T\right)$ ein eindeutiges $y$ konstruieren (nämlich 
\begin{equation}
y_{\left\{v_1, v_2\right\}, u} := \sum_{i=1}^2 \begin{cases}
1 & \left\{v_i, u\right\}\in T\wedge v_{3-i}\in R_{T\backslash \left\{\left\{v_i, u\right\}\right\}}\left(u\right)\\
0 & \textnormal{sonst),}
\end{cases} \label{eq:y_belegung}
\end{equation}
so dass $
\left(
\begin{array}{c}
x \\
y
\end{array}
\right)
\in P^{K_n, spt, symm, xy}$ erfüllt ist.
\end{Sa}
\begin{bew}
Zur Erinnerung seien nochmal die für $P^{K_n, spt, symm, xy}$ gültigen Gleichungen und Ungleichungen wiedergegeben:
\begin{eqnarray}
\sum_{i=1}^3 \left(x_{\left\{v_{i}, v_{i \stackrel{\left[3\right]}{+} 1}\right\}} -  y_{\left\{v_{i}, v_{i \stackrel{\left[3\right]}{+} 1}\right\}, v_{i \stackrel{\left[3\right]}{+} 2}}\right) & = & 0\ \forall \left\{v_1, v_2, v_3\right\}\in {[n] \choose 3} \label{eq:P^xy_1} \\
2 x_{\left\{v_1, v_2\right\}} + \sum_{u\in[n]\backslash\left\{v_1, v_2\right\}} y_{\left\{v_1, v_2\right\}, u} & = & 2\ \forall \left\{v_1, v_2\right\}\in {[n] \choose 2} \label{eq:P^xy_2} \\
x_{\left\{v_1, v_2\right\}} & \geq & 0 \ \forall \left\{v_1, v_2\right\} \in {\left[n\right] \choose 2} \nonumber \\
y_{\left\{v_1, v_2\right\}, u} & \geq & \left.0 \ \forall \left(\left\{v_1, v_2\right\}, u\right) \in EV_n\right\}. \nonumber
\end{eqnarray}

Es sei angemerkt, dass die zu beweisende Aussage für $n=2$ trivialerweise erfüllt ist (da es dann gar keine $y$-Variablen gibt). Somit können wir $n\geq 3$ voraussetzen.

Es folgt aus Gleichung \eqref{eq:P^xy_2}, dass
\begin{displaymath}
\left\{v_1, v_2\right\} \in T \Rightarrow \forall u \in \left[n\right] \backslash \left\{v_1, v_2\right\}: y_{\left\{v_1, v_2\right\}, u} = 0
\end{displaymath}
gilt. Daher sei ohne Einschränkung $\left\{v_1, v_2\right\} \notin T$.

Als nächstes zeigen wir, dass, wenn $x=\chi(T)$ und $\left(\begin{array}{c}
x \\
y
\end{array}\right) \in P^{K_n, spt, symm, xy}$ ist, dann gilt:
\begin{displaymath}
y_{\left\{v_1, v_2\right\}, u} \neq 0 \Rightarrow \left\{v_1, u\right\} \in T \vee \left\{v_2, u\right\} \in T.
\end{displaymath}

Dies folgt aus Gleichung \eqref{eq:P^xy_1}:
\begin{equation}
\underbrace{x_{\left\{v_1, v_2\right\}}}_{\substack{= 0,\textnormal{ da }\left\{v_1, v_2\right\}\notin T \\ \textnormal{ nach Annahme}}} + x_{\left\{v_1, u\right\}} + x_{\left\{v_2, u\right\}} -  y_{\left\{v_1, v_2\right\}, u} - y_{\left\{v_1, u\right\}, v_2} - y_{\left\{v_2, u\right\}, v_1} = 0. \label{eq:xy_neq0}
\end{equation}
Da nach Annahme $y_{\left\{v_1, v_2\right\}, u} > 0$ ist und $y\geq 0$ gilt, folgt somit $ x_{\left\{v_1, u\right\}} + x_{\left\{v_2, u\right\}} > 0$, woraus $\left\{v_1, u\right\} \in T$ oder $\left\{v_2, u\right\} \in T$ folgt.

Außerdem gilt (weil $T$ Baum ist), dass für vorgegebene $\left(v_1, v_2, v_3\right) \in \left[n\right]^{\underline{3}}$ sicher nicht alle drei Kanten $\left\{v_1, v_2\right\}, \left\{v_2, v_3\right\}, \left\{v_3, v_1\right\}$ in $T$ sein können.

Hieraus und aus Gleichung \eqref{eq:xy_neq0} folgt, dass, wenn $\left\{v_1, u\right\}, \left\{v_2, u\right\} \in T$ ist, $y_{\left\{v_1, v_2\right\}, u} = 2$ gilt.

Wir fassen dies in einer Tabelle zusammen: \\

\begin{tabular}{| l | c |}
\hline
Situation & Wert von $y_{\left\{v_1, v_2\right\}, u}$ \\
\hline
$\left\{v_1, v_2\right\} \in T$ & $0$ \\
$\left\{v_1, v_2\right\}, \left\{v_1, u\right\}, \left\{v_2, u\right\} \notin T$ & $0$ \\
$\left\{v_1, v_2\right\} \notin T \wedge \left\{v_1, u\right\}, \left\{v_2, u\right\} \in T$ & $2$ \\
\hline
\end{tabular} \\

Da wir soeben bewiesen haben, dass, wenn $\left\{v_1, v_2\right\} \in T$ oder $\left\{v_1, u\right\} \notin T \wedge \left\{v_2, u\right\} \notin T$ ist, $y_{\left\{v_1, v_2\right\}, u} = 0$ gilt, können wir im Folgenden stets voraussetzen:
\begin{itemize}
\item $\left\{v_1, v_2\right\} \notin T$
\item $\left\{v_1, u\right\} \in T \vee \left\{v_2, u\right\} \in T$ (aber nicht beides) -- ohne Einschränkung gelte Ersteres.
\end{itemize}

Sei $W:=\left(\underbrace{w_0}_{=v_1}, e_1, w_1, \ldots, e_k, \underbrace{w_k}_{=v_2}\right)$ ein $v_1$-$v_2$-Weg in $T$.

Wir werden per vollständiger Induktion über die Weglänge $k$ zeigen, dass $y_{\left\{v_1, v_2\right\}, u}$ einen eindeutigen Wert annimmt.

Der Induktionsanfang ($k=1$ und falls $u\in W$ auch $k=2$) ist gemäß den Vorbetrachtungen erfüllt.

Gelte die Aussage daher für alle $v_1$, $v_2$, $u$, so dass die kombinatorische Distanz zwischen $v_1$ und $v_2$ kleiner oder gleich $k-1$ ist.

\paragraph{Fall 1:} $u\in V\left(W\right)$ \\

Wegen der Annahme $\left\{v_1, u\right\} \in T$ und Fall-Voraussetzung $u\in V\left(W\right)$ ist $w_1 = u$. Außerdem ist aufgrund der allgemeinen Annahmen $\left\{v_2, u\right\} \notin T$.

Dann gilt (Gleichung \eqref{eq:P^xy_1}):
\begin{eqnarray*}
0 & = & \underbrace{x_{\left\{v_1, u\right\}}}_{= 1} + \underbrace{x_{\left\{v_1, v_2\right\}}}_{= 0} + \underbrace{x_{\left\{v_2, u\right\}}}_{= 0} - y_{\left\{v_1, v_2\right\}, u} - \underbrace{y_{\left\{v_1, u\right\}, v_2}}_{= 0\textnormal{, da }\left\{v_1, u\right\}\in T} - y_{\left\{v_2, u\right\}, v_1}
\end{eqnarray*}

Da die kombinatorische Distanz in $T$ zwischen $v_2$ und $u$ gleich $k-1$ ist, gilt nach Induktionsvoraussetzung $y_{\left\{v_2, u\right\}, v_1} = 0$ und somit $y_{\left\{v_1, v_2\right\}, u} = 1$.

\paragraph{Fall 2:} $u\notin V\left(W\right)$ \\

Es gilt weiterhin nach Annahme $\left\{v_1, u\right\} \in T$.

Wir wissen bereit, dass für $k=2$ gilt:
\begin{eqnarray*}
y_{\left\{v_1, v_2\right\}, w_2} & = & 2
\end{eqnarray*}
und für $k\geq 3$ gilt wegen Fall 1:
\begin{eqnarray*}
y_{\left\{v_1, v_2\right\}, w_2} & = & 1 \\
y_{\left\{v_1, v_2\right\}, w_{k-1}} & = & 1.
\end{eqnarray*}

In jedem Fall haben wir
\begin{displaymath}
\sum_{\widehat{u}\in \left\{w_2, w_{k-1}\right\}} y_{\left\{v_1, v_2\right\}, \widehat{u}} = 2.
\end{displaymath}

Somit erhält man (Gleichung \eqref{eq:P^xy_2}):
\begin{eqnarray*}
2 & = & 2 \underbrace{x_{\left\{v_1, v_2\right\}}}_{= 0} + 
\sum_{\widehat{u}\in[n]\backslash\left\{v_1, v_2\right\}} y_{\left\{v_1, v_2\right\}, \widehat{u}} \\
& = & \underbrace{\sum_{\widehat{u}\in \left\{w_2, w_{k-1}\right\}} y_{\left\{v_1, v_2\right\}, \widehat{u}}}_{ = 2} + \sum_{\widehat{u}\in \left[n\right] \backslash \left\{w_2, w_{k-1}\right\}} y_{\left\{v_1, v_2\right\}, \widehat{u}},
\end{eqnarray*}
woraus man unter Nutzung von $y_{\left\{v_1, v_2\right\}, u} \geq 0 \ \forall \left(\left\{v_1, v_2\right\}, u\right) \in EV_n$ erhält: $y_{\left\{v_1, v_2\right\}, u} = 0$.
\end{bew}

\section{Die Einbettung der aufspannenden Bäume in $P^{K_n, spt, asymm, 4}$}

\begin{Sa}
\label{sa:einb_asymm_4}
Es gibt für $n \geq 4$ (mindestens) einen aufspannenden Baum in $K_n$, dessen Einbettung seines charakteristischen Vektors in $P^{K_n, spt, asymm, 4}$ \emph{nicht} eindeutig ist.
\end{Sa}
\begin{bew}
Betrachte für $n \geq 4$ die Kantenmenge
\begin{displaymath}
T := \left\{
\left\{1, 2\right\}, \left\{1, 4\right\}, \left\{3, 4\right\}
\right\} \mathbin{\dot{\cup}} \mathop{\dot{\bigcup}}_{k=4}^{n-1} \left\{\left\{k, k+1\right\}\right\}.
\end{displaymath}
eines aufspannenden Baums von $K_n$.

Die z-Variablen der kanonischen Einbettung sind (für $\left(v_1, v_2, v_3\right) \in \left[n\right]^{\underline{3}^{1, 2 < 3}}$)
\begin{displaymath}
z_{\left(v_1, v_2\right), v_3} = \begin{cases}
1 & \left(v_1, v_2\right) = \left(2, 1\right) \\
1 & \left(v_1, v_2\right) = \left(1, 4\right) \\
1 & \left(v_1, v_2\right) = \left(k, k+1\right)\textnormal{ mit }k \geq 3 \\
0 & \textnormal{sonst}.
\end{cases}
\end{displaymath}

Nun werden wir zeigen, dass die folgende Belegung der z-Variablen ebenfalls eine Einbettung von $T$ in $P^{K_n, spt, asymm, 4}$ darstellt:
\begin{displaymath}
z_{\left(v_1, v_2\right), v_3} = \begin{cases}
1 & \left(v_1, v_2, v_3\right) = \left(1, 2, 3\right) \\
0 & \left(v_1, v_2, v_3\right) = \left(2, 1, 3\right) \\
1 & \left(v_1, v_2\right) = \left(2, 1\right) \wedge v_3 \geq 4 \\
1 & \left(v_1, v_2\right) = \left(1, 4\right) \\
1 & \left(v_1, v_2\right) = \left(k, k+1\right)\textnormal{ mit }k \geq 3 \\
0 & \textnormal{sonst}.
\end{cases}
\end{displaymath}

Dazu überprüfen wir die Gleichungen von $P^{K_n, spt, asymm, 4}$.

Die Gleichungsserie \eqref{eq:spt_erweitert_asymm4_prim_1}:
\begin{displaymath}
x_{\left\{v_1, v_2\right\}} - z_{\left(v_1, v_2\right), w} - z_{\left(v_2, v_1\right), w} = 0\ \forall \left(\left\{v_1, v_2\right\}, w\right)\in EV_n^< 
\end{displaymath}
ist offensichtlich erfüllt, da für alle $\left\{v_1, v_2\right\} \notin T$ $z_{\left(v_1, v_2\right), w} = z_{\left(v_2, v_1\right), w} = 0$ und für $\left\{v_1, v_2\right\} \in T$ genau eines der $z_{\left(v_1, v_2\right), w}, z_{\left(v_2, v_1\right), w}$ den Wert 1 und das andere den Wert 0 annimmt.

Zu den Gleichungen \eqref{eq:spt_erweitert_asymm4_prim_2} bzw. \eqref{eq:spt_erweitert_asymm4_prim_2,5}:
\begin{eqnarray*}
x_{\left\{v_1, v_2\right\}} + \sum_{u\in[v_2-1]\backslash\left\{v_1\right\}} z_{\left(v_1, u\right), v_2} & \leq & 1\ \forall \left(v_1, v_2\right)\in {[n-1]^{\underline{2}}}^< \\
x_{\left\{v, n\right\}} + \sum_{u\in[n-1]\backslash\left\{v\right\}} z_{\left(v, u\right), n} & = & 1\ \forall v \in [n-1].
\end{eqnarray*}

Für $\left\{v_1, v_2\right\} \in T$ ist nichts zu zeigen -- hier ist für alle $\left(v_1, u, v_2\right) \in \left[n\right]^{\underline{3}^{1, 2 < 3}}$ sicher $z_{\left(v_1, u\right), v_2} = 0$.

Demnach können wir $\left\{v_1, v_2\right\} \notin T$ mit $v_1 < v_2$ voraussetzen.

Für $v_1 \geq 3$ (und wegen $\left\{v_1, v_2\right\} \notin T$ somit $v_2 \geq v_1 + 2$) ist von den $z_{\left(v_1, u\right), v_2}$ genau  $z_{\left(v_1, v_1 + 1\right), v_2} = 1$ (die anderen 0). Somit sind die Gleichungen erfüllt.

Für $v_1 = 2$ und $v_2 \neq 3$ gilt für alle zulässigen $u, v_2$  $z_{\left(v_1, u\right), v_2} = 0$ -- auch hier sind die Gleichungen erfüllt.

Im Falle von $\left(v_1, v_2\right) = \left(2, 3\right)$ gibt es nur $z_{\left(2, 1\right), 3}$ zu betrachten -- es nimmt den Wert $0$ an.

Wenn $v_1 = 1$ und $v_2 \neq 2, 4$ ist (beides ist wegen $\left\{v_1, v_2\right\} \notin T$ ausgeschlossen), ist im Falle $v_2 = 3$ nur $z_{\left(1, 2\right), 3}$ zu betrachten (es nimmt den Wert $1$ an). Im Falle $v_2 \geq 5$ ist
\begin{displaymath}
z_{\left(1, u\right), v_2} = \begin{cases}
1 & u=4 \\
0 & \textnormal{sonst}.
\end{cases}
\end{displaymath}
\end{bew}

Aus Lemma \ref{le:proj_Pasymm4_n3} folgt sofort

\begin{Le}
\label{le:einbettung_asymm4_n3}
Die Einbettung des charakteristischen Vektors eines aufspannenden Baums von $K_3$ in $P^{K_3, spt, asymm, 4}$ ist eindeutig.
\end{Le}

Aus der Kombination von Satz \ref{sa:einb_asymm_4} und Lemma \ref{le:einbettung_asymm4_n3} erhalten wir

\begin{Kor}
Die Einbettung des charakteristischen Vektors eines aufspannenden Baums von $K_n$ in $P^{K_n, spt, asymm, 4}$ ist für $n \leq 3$ eindeutig, für $n\geq 4$ dagegen nicht.
\end{Kor}

\chapter{Die Dimension der erweiterten Formulierungen}

In diesem Kapitel wollen wir die Dimensionen der Polyeder der wichtigsten vorgestellten erweiterten Formulierungen des Spannbaum-Polytops berechnen.

\section{Hintergründe aus der linearen Algebra und diskreten Geometrie}

\begin{Sa}
\label{sa:dim1}
Sei
\begin{displaymath}
P := \left\{x \in \mathbb{R}^d: 
\begin{array}{ccc}
A^1 x & = & b^1 \\
A^2 x & \leq & b^2
\end{array}\right\}
\end{displaymath}
nichtleer, wobei $A^i \in \mathbb{R}^{m_i \times d}$ und $b^i \in \mathbb{R}^{m_i}$ für $i \in \left[2\right]$ seien.

Keine der Ungleichungen sei eine versteckte Gleichung, d. h. für alle $i \in \left[m_i\right]$ gelte $\exists x \in P: \left(A^2 x\right)_i < b^2_i$.

Dann gilt:
\begin{displaymath}
\dim P := \dim \aff P = \dim \ker A^1.
\end{displaymath}
\end{Sa}
\begin{bew}
Ohne Einschränkung können wir voraussetzen, dass $m_2 \geq 1$ ist (sonst folgt die Aussage aus Lösungseigenschaften linearer Gleichungssysteme).

Da aus den Eigenschaften der Lösungen von linearen Gleichungssystemen folgt, dass
\begin{displaymath}
\dim \aff P \leq \dim \ker A^1
\end{displaymath}
gilt, ist nur zu zeigen, dass
\begin{displaymath}
\dim \aff P \geq \dim \ker A^1
\end{displaymath}
erfüllt ist.

Sei $k:= \dim \ker A^1$ und sei $q_1, \ldots, q_k$ eine Basis von $\ker A^1$.

Da für alle $i \in \left[m_2\right]$ ein $x \in P$ mit $\left(A^2 x\right)_i < b^2_i$ existiert, seien $r_1, \ldots, r_{m_2} \in P$ solche Punkte, in denen $r_i$ die i-te Gleichung mit echter Ungleichheit erfüllt.

Man überzeugt sich leicht (Konvexität von $P$), dass
\begin{displaymath}
s := \frac{1}{m_2} \sum_{i=1}^{m_2} r_i
\end{displaymath}
in $P$ liegt und alle Ungleichungen mit echter Ungleichheit erfüllt.

Für $\epsilon > 0$ hinreichend klein liegen $s+\epsilon q_i$ für alle $i \in \left[k\right]$ in $P$ -- zusammen mit $s$ bilden diese aufgrund der linearen Unabhängigkeit der $q_i$ $k+1$ affin unabhängige Punkte, woraus die Aussage folgt.
\end{bew}

\begin{Sa}
\label{sa:dim2}
Sei $A\in \mathbb{K}^{m \times n}$ (wobei $\mathbb{K}$ ein beliebiger Körper sei). Dann gilt:
\begin{displaymath}
\dim \ker A = n-m+\dim \ker A^T.
\end{displaymath}
\end{Sa}
\begin{bew}
\begin{eqnarray*}
\dim \ker A & = & n - \dim \bild A \\
& = & n - \rg A \\
& = & n - \rg A^T \\
& = & n - \dim \bild A^T \\
& = & n - \left(m - \dim \ker A^T\right) \\
& = & n - m+\dim \ker A^T.
\end{eqnarray*}
\end{bew}

Somit haben wir das Problem der Dimensionsbestimmung auf das Problem reduziert, die Dimension des Kerns einer transponierten Matrix zu bestimmen.

Wenn wir die Struktur sämtlicher in dieser Arbeit betrachteten erweiterten Formulierungen uns noch einmal anschauen, werden wir feststellen, dass die (echten) Gleichungen, die von den Variablenwerten erfüllt werden müssen, folgende Struktur besitzen: es gibt zwei Variablenserien
\begin{itemize}
\item $x$-Variablen: indiziert durch zwei Indizes (aus ${\left[n\right] \choose 2}$)
\item $z$- bzw. $y$-Variablen (im Folgenden werden wir diese mit $z$ bezeichnen): indiziert durch drei Indizes
\end{itemize}
und zwei Gleichungsserien:
\begin{itemize}
\item eine erste indiziert durch drei Indizes
\item eine zweite indiziert durch zwei Indizes
\end{itemize}

Also gilt folgende Struktur
\begin{displaymath}
\underbrace{\left(
\begin{array}{cc}
A_{11} & A_{12} \\
A_{21} & A_{22}
\end{array}
\right)}_{=: A} \left(
\begin{array}{c}
x \\
z
\end{array}
\right)
= 
\left(
\begin{array}{c}
b_{1} \\
b_{2}
\end{array}
\right).
\end{displaymath}

Für die Spalten von $A^T$ (somit Zeilen von $A$) führen wir "`Dualvariablen"' $q$ und $r$ ein.

Es ist also das Gleichungssystem
\begin{equation}
\left(
\begin{array}{cc}
A_{11}^T & A_{21}^T \\
A_{12}^T & A_{22}^T
\end{array}
\right) \left(
\begin{array}{c}
q \\
r
\end{array}
\right) = 
\left(
\begin{array}{c}
0^{\left[n\right] \choose 2} \\
0
\end{array}
\right) \label{eq:dimensions_lgs}
\end{equation}
zu untersuchen, wobei in diesem die erste Gleichungsserie zu den zu den $x$-Variablen gehörenden Spalten korrespondiert und die zweite zu den $z$-Variablen.

Um dieses zu analysieren, werden wir erst einmal das lineare Gleichungssystem
\begin{displaymath}
\left(
\begin{array}{cc}
A_{12}^T & A_{22}^T
\end{array}
\right) \left(
\begin{array}{c}
q \\
r
\end{array}
\right) = \left(
\begin{array}{c}
0
\end{array}
\right).
\end{displaymath}
betrachten und anschließend die Lösungsstruktur auf das Gleichungssystem \eqref{eq:dimensions_lgs} verallgemeinern.

Dies werden wir im folgenden Abschnitt für einige der erweiterten Formulierungen  durchführen.

\section{Dimensionsanalyse}

\subsection{Untersuchung von $P^{K_n, spt, symm, 0}$}

Es sei daran erinnert, das $P^{K_n, spt, symm, 0}$ gemäß Lemma \ref{le:spt_erweitert_dual2} durch folgende Gleichungen und Ungleichungen beschrieben wird:
\begin{eqnarray*}
P^{K_n, spt, symm, 0} & = & \left\{
\left(\begin{array}{c}
x \\
z
\end{array}\right)\in\mathbb{R}^{[n]\choose 2}\times \mathbb{R}^{[n]^{\underline{3}}}: \right. \\
x_{\left\{v, w\right\}} - z_{\left(v, w\right), u} - z_{\left(w, v\right), u} & = & 0\ \forall \left(\left\{v, w\right\}, u\right)\in EV_n\ \left(q_{\left\{v, w\right\}, u}\right)\\
x_{\left\{v, w\right\}} + \sum_{u\in[n]\backslash\left\{v, w\right\}} z_{\left(v, u\right), w} & = & 1\ \forall \left(v, w\right)\in [n]^{\underline{2}} \ \left(r_{\left(v, w\right)}\right) \\
z_{\left(v, w\right), u} & \geq & \left.0\ \forall \left(v, w, u\right)\in [n]^{\underline{3}} \right\}.
\end{eqnarray*}

Somit ist folgendes homogenes lineares Gleichungssystem zu analysieren:
\begin{eqnarray*}
\sum_{u\in[n]\backslash\left\{v, w\right\}} q_{\left\{v, w\right\}, u} + r_{\left(v, w\right)} + r_{\left(w, v\right)} & = & 0\ \forall \left\{v, w\right\}\in {[n] \choose 2} \ \left(x_{\left\{v, w\right\}}\right) \\
-q_{\left\{v, w\right\}, u} + r_{\left(v, u\right)} & = & 0\ \forall \left(v, w, u\right)\in [n]^{\underline{3}} \ \left(z_{\left(v, w\right), u}\right).
\end{eqnarray*}

\begin{Le}
\label{le:kern}
Betrachte das homogene lineare Gleichungssystem
\begin{eqnarray}
-q_{\left\{v, w\right\}, u} + r_{\left(v, u\right)} & = & 0\ \forall \left(v, w, u\right)\in [n]^{\underline{3}}  \label{eq:le_kern}
\end{eqnarray}
mit
\begin{displaymath}
\left(\begin{array}{c}
q \\
r
\end{array}\right) \in \mathbb{R}^{EV_n}\times \mathbb{R}^{[n]^{\underline{2}}}.
\end{displaymath}

Jede Lösung dieses homogenen linearen Gleichungssystems lässt sich folgendermaßen darstellen:

Sei $\lambda\in\mathbb{R}^n$ vorgegeben. Setze
\begin{eqnarray}
r_{\left(i, j\right)} & := & \lambda_j\ \forall \left(i, j\right)\in\left[n\right]^2  \label{eq:le_kern_r_final} \\
q_{\left\{i, j\right\},k} & := & \lambda_k\ \forall \left(\left\{i, j\right\}, k\right)\in EV_n. \label{eq:le_kern_q_final}
\end{eqnarray}
\end{Le}
\begin{bew}
Man rechnet leicht nach, dass dies tatsächlich Lösungen sind. Daher reicht es zu zeigen, dass sich jede Lösung in dieser Form darstellen lässt.

Sei $\left(\begin{array}{c}
q \\
r
\end{array}\right)$ eine Lösung des homogenen linearen Gleichungssystems.

Setze
\begin{equation}
\lambda_{i} := r_{\left(i \stackrel{[n]}{-} 1, i\right)} \ \forall i\in\left[n\right]. \label{eq:le_kern_r1}
\end{equation}
Dann folgt aus Gleichung \eqref{eq:le_kern} für alle $i, j\in\left[n\right], j\neq i, i \stackrel{[n]}{+} 1$:
\begin{eqnarray*}
q_{\left\{i, j\right\}, i \stackrel{[n]}{+} 1} & = & r_{\left(i, i \stackrel{[n]}{+} 1\right)} \\
& = & \lambda_{i \stackrel{[n]}{+} 1},
\end{eqnarray*}
womit wir abermals unter Nutzung von \eqref{eq:le_kern} für alle $i, j\in\left[n\right], j\neq i \stackrel{[n]}{-} 1, i$ erhalten:
\begin{eqnarray*}
r_{\left(j, i\right)} & = & q_{\left\{i \stackrel{[n]}{-} 1, j\right\}, i} \\
& = & \lambda_i.
\end{eqnarray*}
Zusammen mit Gleichung \eqref{eq:le_kern_r1} ergibt dies:
\begin{displaymath}
r_{\left(j, i\right)} = \lambda_i\ \forall \left(i, j\right)\in\left[n\right]^2.
\end{displaymath}
Dies in Gleichung \eqref{eq:le_kern} eingesetzt ergibt
\begin{displaymath}
q_{\left\{i, j\right\},k} = \lambda_k\ \forall \left(\left\{i, j\right\}, k\right)\in EV_n.
\end{displaymath}
\end{bew}

\begin{Kor}
\label{kor:kern}
Betrachte das homogene lineare Gleichungssystem
\begin{eqnarray}
\sum_{u\in[n]\backslash\left\{v, w\right\}} q_{\left\{v, w\right\}, u} + r_{\left(v, w\right)} + r_{\left(w, v\right)}& = & 0\ \forall \left\{v, w\right\}\in {[n] \choose 2} \label{eq:kern1} \\
-q_{\left\{v, w\right\}, u} + r_{\left(v, u\right)} & = & 0\ \forall \left(v, w, u\right)\in [n]^{\underline{3}} \label{eq:kern2}
\end{eqnarray}
mit
\begin{displaymath}
\left(\begin{array}{c}
q \\
r
\end{array}\right) \in \mathbb{R}^{EV_n}\times \mathbb{R}^{[n]^{\underline{2}}}.
\end{displaymath}

Jede Lösung dieses homogenen linearen Gleichungssystems lässt sich folgendermaßen darstellen:

Sei $\lambda\in\mathbb{R}^n$ mit $\sum_{i=1}^n \lambda_i = 0$ vorgegeben. Setze
\begin{eqnarray*}
r_{\left(i, j\right)} & := & \lambda_j\ \forall \left(i, j\right)\in\left[n\right]^2 \\
q_{\left\{i, j\right\},k} & := & \lambda_k\ \forall \left(\left\{i, j\right\}, k\right)\in EV_n.
\end{eqnarray*}
\end{Kor}
\begin{bew}
Aus Lemma \ref{le:kern} erhalten wir sofort die Lösungsstruktur der Gleichungsserie \eqref{eq:kern2}. Somit müssen wir nur noch untersuchen, welche dieser Lösungen auch Gleichung \eqref{eq:kern1} erfüllen.

Wenn man die Gleichungen \eqref{eq:le_kern_r_final} und \eqref{eq:le_kern_q_final} in eine beliebige Gleichung aus \eqref{eq:kern1} einsetzt, erhält man für beliebige $\left\{v, w\right\}\in {[n] \choose 2}$:

\begin{eqnarray*}
0 & = & \sum_{u\in[n]\backslash\left\{v, w\right\}} q_{\left\{v, w\right\}, u} + r_{\left(v, w\right)} + r_{\left(w, 
v\right)} \\
& = & \left(\sum_{u\in[n]\backslash\left\{v, w\right\}} \lambda_u\right) + \lambda_v + \lambda_w \\
& = & \sum_{u\in[n]} \lambda_u,
\end{eqnarray*}
was die zu zeigende notwendige Bedingung darstellt.
\end{bew}

\subsection{Untersuchung von $P^{K_n, spt, symm, xy}$}

Wir wollen in Erinnerung rufen, dass gemäß Definition \ref{def:P_p^symm_xy} $P^{K_n, spt, symm, xy}$ folgendermaßen definiert wurde:
\begin{eqnarray*}
P^{K_n, spt, symm, xy} & := & \left\{ \left(\begin{array}{c}
x \\
y
\end{array}\right)\in\mathbb{R}^{[n]\choose 2}\times \mathbb{R}^{EV_n}: \right.
\nonumber \\
\sum_{i=1}^3 \left(x_{\left\{v_{i}, v_{i \stackrel{\left[3\right]}{+} 1}\right\}} -  y_{\left\{v_{i}, v_{i \stackrel{\left[3\right]}{+} 1}\right\}, v_{i \stackrel{\left[3\right]}{+} 2}}\right) & = & 0\ \forall \left\{v_1, v_2, v_3\right\}\in {[n] \choose 3}\ \left(q_{\left\{v_1, v_2, v_3\right\}}\right) \\
2 x_{\left\{v_1, v_2\right\}} + \sum_{u\in[n]\backslash\left\{v_1, v_2\right\}} y_{\left\{v_1, v_2\right\}, u} & = & 2\ \forall \left\{v_1, v_2\right\}\in {[n] \choose 2}\ \left(r_{\left\{v_1, v_2\right\}}\right) \\
x_{\left\{v_1, v_2\right\}} & \geq & 0 \ \forall \left\{v_1, v_2\right\} \in {\left[n\right] \choose 2} \nonumber \\
y_{\left\{v_1, v_2\right\}, u} & \geq & \left.0 \ \forall \left(\left\{v_1, v_2\right\}, u\right) \in EV_n\right\}. \nonumber
\end{eqnarray*}

Somit ist folgendes homogenes lineares Gleichungssystem zu betrachten:
\begin{eqnarray*}
\sum_{u\in[n]\backslash\left\{v, w\right\}} q_{\left\{v_1, v_2, u\right\}} + 2 r_{\left\{v_1, v_2\right\}} & = & 0 \ \forall \left\{v_1, v_2\right\}\in {[n] \choose 2} \ \left(x_{\left\{v_1, v_2\right\}}\right) \\
-q_{\left\{v_1, v_2, u\right\}} + r_{\left\{v_1, v_2\right\}} & = &  0\ \forall \left(\left\{v_1, v_2\right\}, u\right)\in EV_n  \ \left(y_{\left\{v_1, v_2\right\}, u}\right).
\end{eqnarray*}

\begin{Le}
\label{le:loesung_transponiert_nsymmxy}
Betrachte das homogene lineare Gleichungssystem
\begin{eqnarray*}
-q_{\left\{v_1, v_2, u\right\}} + r_{\left\{v_1, v_2\right\}} & = &  0\ \forall \left(\left\{v_1, v_2\right\}, u\right)\in EV_n
\end{eqnarray*}
Dieses besitzt genau folgende Lösung für ein vorgegebenes $\lambda\in \mathbb{R}$:
\begin{eqnarray*}
q_{\left\{v_1, v_2, v_3\right\}} & = & \lambda \ \forall \left\{v_1, v_2, v_3\right\} \in {\left[n\right] \choose 2} \\
r_{\left\{v_1, v_2\right\}} & = & \lambda \ \forall \left\{v_1, v_2\right\}\in {\left[n\right] \choose 2}.
\end{eqnarray*}
\end{Le}
\begin{bew}
Sei $\left(v_1^*, v_2^*\right) \in \left[n\right]^{\underline{2}}$ fest und $\lambda := r_{\left\{v_1^*, v_2^*\right\}}$.

Daraus folgt für alle $u \in \left[n\right] \backslash \left\{v_1, v_2\right\}$:
\begin{displaymath}
q_{\left\{v_1^*, v_2^*, u\right\}} = \lambda.
\end{displaymath}

Somit gilt für alle $\left\{v_1, v_2\right\} \in {\left[n\right] \choose 2}$ mit $\left\{v_1, v_2\right\} \cap \left\{v_1^*, v_2^*\right\} \neq \emptyset$
\begin{displaymath}
r_{\left\{v_1, v_2\right\}} = \lambda,
\end{displaymath}
woraus für $\left\{v_1, v_2, v_3\right\} \in {\left[n\right] \choose 3}$ mit $\left\{v_1, v_2, v_3\right\} \cap \left\{v_1^*, v_2^*\right\} \neq \emptyset$ folgt:
\begin{displaymath}
q_{\left\{v_1, v_2, v_3\right\}} = \lambda.
\end{displaymath}

Hieraus erhält man, dass für alle $\left\{v_1, v_2\right\} \in {\left[n\right] \choose 2}$
\begin{displaymath}
r_{\left\{v_1, v_2\right\}} = \lambda
\end{displaymath}
erfüllt ist -- somit gilt auch für alle $\left\{v_1, v_2, v_3\right\} \in {\left[n\right] \choose 3}$
\begin{displaymath}
q_{\left\{v_1, v_2, v_3\right\}} = \lambda.
\end{displaymath}
\end{bew}

\begin{Kor}
\label{kor:loesung_transponiert_nsymmxy}
Betrachte das homogene lineare Gleichungssystem
\begin{eqnarray}
\sum_{u\in[n]\backslash\left\{v, w\right\}} q_{\left\{v, w, u\right\}} + 2 r_{\left\{v, w\right\}} & = & 0 \ \forall \left\{v, w\right\}\in {[n] \choose 2} \label{eq:loesung_transponiert_nsymmxy_1} \\
-q_{\left\{v, w, u\right\}} + r_{\left\{v, u\right\}} & = &  0\ \forall \left(\left\{v, u\right\}, w\right)\in EV_n. \nonumber
\end{eqnarray}
Dieses besitzt als Lösung genau die Nullösung.
\end{Kor}
\begin{bew}
Um die Eindeutigkeit einzusehen, setze die Lösung aus Lemma \ref{le:loesung_transponiert_nsymmxy} in Gleichung \eqref{eq:loesung_transponiert_nsymmxy_1} ein.
\end{bew}

\subsection{Untersuchung von $P^{K_n, spt, asymm, 4}$}

\label{sec:untersuchung_Pnasymm4}

$P^{K_n, spt, asymm, 4}$ wird nach Satz \ref{sa:asymmetrisch_erweitert4} durch folgende Gleichungen/Ungleichungen beschrieben:
\begin{eqnarray}
P^{K_n, spt, asymm, 4} & := & \left\{
\left(\begin{array}{c}
x \\
z
\end{array}\right)\in \mathbb{R}^{[n]\choose 2} \times \mathbb{R}^{\left[n\right]^{\underline{3}^{1, 2<3}}}: \nonumber\right. \nonumber \\
x_{\left\{v, w\right\}} - z_{\left(v, w\right), u} - z_{\left(w, v\right), u} & = & 0\ \forall \left(\left\{v, w\right\}, u\right)\in EV_n^< \ \left(q_{\left\{v, w\right\}, u}\right) \nonumber \\
x_{\left\{v, w\right\}} + \sum_{u\in[w-1]\backslash\left\{v\right\}} z_{\left(v, u\right), w} & \leq & 1\ \forall \left(v, w\right)\in {[n-1]^{\underline{2}}}^<  \label{eq:to_check_strict} \\
x_{\left\{v, n\right\}} + \sum_{u\in[n-1]\backslash\left\{v\right\}} z_{\left(v, u\right), n} & = & 1\ \forall v\in [n-1] \ \left(r_{\left(v, n\right)}\right) \nonumber \\
x_{\left\{v, n\right\}} & \geq & 0 \ \forall v\in \left[n-1\right] \nonumber \\
z_{\left(v, w\right), u} & \geq & \left. 0 \ \forall \left(v, w, u\right)\in \left[n\right]^{\underline{3}^{1, 2<3}} \right\}. \nonumber
\end{eqnarray}

Während es bisher klar war, dass keine Ungleichung eine implizite Gleichheit impliziert, müssen wir hier in Bezug auf Ungleichung \eqref{eq:to_check_strict} genauer hinschauen.

Sei $n \geq 3$ (sonst gibt es kein $\left(v, w\right)\in {[n-1]^{\underline{2}}}^<$ und somit keine Gleichung vom Typ \eqref{eq:to_check_strict}) und $\left(v, w\right)\in {[n-1]^{\underline{2}}}^<$.

Definiere folgenden Baum
\begin{displaymath}
T := \left(\left[n\right], \left\{ \left\{v, n\right\} \right\} \cup \bigcup_{u \in \left[n\right] \backslash \left\{v, w\right\}} \left\{\left\{w, u\right\}\right\} \right),
\end{displaymath}
$x$ als seinen charakteristischen Vektor und $z$ als den kanonischen $z$-Vektor bezüglich $T$ bzw. $x$.

Man rechnet leicht nach, dass für alle $u\in[w-1]\backslash\left\{v\right\}$ gilt: $z_{\left(v, u\right), w} = 0$. Somit gilt echte Ungleichheit in der durch $\left(v, w\right)$ indizierten Gleichung \eqref{eq:to_check_strict}.

Nachdem dies betrachtet wurde, gehen wir zum homogenen linearen Gleichungssystem über:
\begin{eqnarray*}
\sum_{u > w}^n q_{\left\{v, w\right\}, u} & = & 0\ \forall \left(v, w\right)\in {[n-1]^{\underline{2}}}^<\ \left(x_{\left\{v, w\right\}}\right) \\
r_{\left(v, n\right)} & = & 0\ \forall v \in [n-1]\ \left(x_{\left\{v, n\right\}}\right) \\
-q_{\left\{v, w\right\}, u} & = & 0\ \forall \left(v, w, u\right)\in [n-1]^{\underline{3}^{1,2<3}} \ \left(z_{\left(v, w\right), u}\right) \\
-q_{\left\{v, w\right\}, n} + r_{\left(v, n\right)} & = & 0\ \forall \left(v, w, u\right)\in [n-1]^{\underline{2}} \ \left(z_{\left(v, w\right), n}\right)
\end{eqnarray*}
mit
\begin{displaymath}
\left(\begin{array}{c}
q \\
r
\end{array}\right) \in 
\mathbb{R}^{EV_n^{<}}\times {\mathbb{R}^{\underline{2}}}^<.
\end{displaymath}

Es ist evident, dass dieses lineare Gleichungssystem nur die Nulllösung besitzt. Dies notieren wir als Lemma:

\begin{Le}
\label{le:linearisiert_asymm4_vereinfacht}
Das homogene lineare Gleichungssystem
\begin{eqnarray*}
\sum_{u > w}^n q_{\left\{v, w\right\}, u} & = & 0\ \forall \left(v, w\right)\in {[n-1]^{\underline{2}}}^<\ \left(x_{\left\{v, w\right\}}\right) \\
r_{\left(v, n\right)} & = & 0\ \forall v \in [n-1]\ \left(x_{\left\{v, n\right\}}\right) \\
-q_{\left\{v, w\right\}, u} & = & 0\ \forall \left(v, w, u\right)\in [n-1]^{\underline{3}^{1,2<3}} \ \left(z_{\left(v, w\right), u}\right) \\
-q_{\left\{v, w\right\}, n} + r_{\left(v, n\right)} & = & 0\ \forall \left(v, w, u\right)\in [n-1]^{\underline{2}} \ \left(z_{\left(v, w\right), n}\right)
\end{eqnarray*}
mit
\begin{displaymath}
\left(\begin{array}{c}
q \\
r
\end{array}\right) \in 
\mathbb{R}^{EV_n^{<}}\times \mathbb{R}^{\bigcup_{u \in \left[n-1\right]} \left\{\left(u, n\right)\right\}}.
\end{displaymath}
besitzt als Lösung genau die Nulllösung.
\end{Le}

\section{Berechnungen der Dimensionen}

Nun nehmen wir alle Resultate zusammen und berechnen die Dimensionen der Polytope der erweiterten Formulierungen.

\begin{Sa}
\label{sa:dimension_formulierungen}
\begin{eqnarray*}
\dim P^{K_n, spt, symm, 0} & = & \frac{\left(n-1\right)^2 \left(n-2\right)}{2} \\
\dim P^{K_n, spt, symm, xy} & = & \frac{n \left(n-1\right)\left(n-2\right)}{3} \\
\dim P^{K_n, spt, asymm, 4} & = & \frac{\left(n+3\right)\left(n-1\right)\left(n-2\right)}{6}
\end{eqnarray*}
\end{Sa}
\begin{bew}
Dass die Ungleichungen im Allgemeinen keine Gleichheit induzieren, ist offensichtlich bzw. haben wir für $P^{K_n, spt, asymm, 4}$ in Abschnitt \ref{sec:untersuchung_Pnasymm4} gezeigt.

Somit können wir die Sätze \ref{sa:dim1} und \ref{sa:dim2} anwenden.

Für die Anzahl an Variablen erhalten wir ${n \choose 2} = \frac{n \left(n-1\right)}{2}$ $x$-Variablen und für die Anzahl zusätzlicher Variablen gilt:
\begin{itemize}
\item $n \left(n-1\right) \left(n-2\right)$ für $P^{K_n, spt, symm, 0}$ (bezüglich Beschreibung $\mathcal{I}^{n, symm, 0}_2$; vgl. Bemerkung \ref{bem:spt_erweitert_dual2})
\item $\frac{1}{2} n \left(n-1\right) \left(n-2\right)$ für $P^{K_n, spt, symm, xy}$ (vgl. Bemerkung \ref{bem:gleichungen_Pxy})
\item $\frac{1}{3} n \left(n-1\right) \left(n-2\right)$ für $P^{K_n, spt, asymm, 4}$ (vgl. Bemerkung \ref{bem:asymm_erw4_eigenschaften}).
\end{itemize}

Als Anzahl an Gleichungen haben wir
\begin{itemize}
\item $\frac{1}{2} n \left(n-1\right) \left(n-2\right) + n \left(n-1\right)$ für $P^{K_n, spt, symm, 0}$ (bezüglich Beschreibung $\mathcal{I}^{n, symm, 0}_2$; vgl. Bemerkung \ref{bem:spt_erweitert_dual2})
\item $\frac{1}{6} n \left(n-1\right) \left(n-2\right) + \frac{1}{2} n \left(n-1\right)$ für $P^{K_n, spt, symm, xy}$ (vgl. Bemerkung \ref{bem:gleichungen_Pxy})
\item $\frac{1}{6} n \left(n-1\right) \left(n-2\right) + \left(n-1\right)$ für $P^{K_n, spt, asymm, 4}$ (vgl. Bemerkung \ref{bem:asymm_erw4_eigenschaften}).
\end{itemize}

Für die Dimension des Kerns von $A^T$ (mit $A$ Matrix der Gleichungen) erhalten wir
\begin{itemize}
\item $n-1$ (vgl. Korollar \ref{kor:kern}) für $P^{K_n, spt, symm, 0}$ (bezüglich Beschreibung $\mathcal{I}^{n, symm, 0}_2$)
\item $0$ für $P^{K_n, spt, symm, xy}$ (vgl. Korollar \ref{kor:loesung_transponiert_nsymmxy}) und $P^{K_n, spt, asymm, 4}$ (vgl. Lemma \ref{le:linearisiert_asymm4_vereinfacht})
\end{itemize}

Somit gilt: 
\begin{eqnarray*}
\dim P^{K_n, spt, symm, 0} & = & \frac{n \left(n-1\right)}{2}+n \left(n-1\right)(n-2) \\
& & -\left(\frac{n \left(n-1\right)\left(n-2\right)}{2}+n \left(n-1\right)\right)+\left(n-1\right) \\
& = & \frac{\left(n-1\right)^2 \left(n-2\right)}{2}.
\end{eqnarray*}

Für $P^{K_n, spt, symm, xy}$ ergibt sich
\begin{eqnarray*}
\dim P^{K_n, spt, symm, xy} & = & \frac{n \left(n-1\right)}{2}+\frac{1}{6} n \left(n-1\right)(n-2) -\left(\frac{n \left(n-1\right)\left(n-2\right)}{2}+\frac{n \left(n-1\right)}{2}\right) \\
& = & \frac{n \left(n-1\right)\left(n-2\right)}{3}.
\end{eqnarray*}

Für $P^{K_n, spt, asymm, 4}$ ergibt sich
\begin{eqnarray*}
\dim P^{K_n, spt, asymm, 4} & = & \frac{n \left(n-1\right)}{2}+\frac{1}{3} n \left(n-1\right)(n-2)  -\left(\frac{n \left(n-1\right)\left(n-2\right)}{6}+\left(n-1\right)\right) \\
& = & \frac{\left(n+3\right)\left(n-1\right)\left(n-2\right)}{6}.
\end{eqnarray*}
\end{bew}

\begin{Def}
\label{def:xz_n}
Sei
\begin{displaymath}
XZ_n := \left\{ \left(\begin{array}{c}
x\left(T\right) \\
z\left(T\right)
\end{array}\right)\in\mathbb{R}^{[n]\choose 2}\times \mathbb{R}^{[n]^{\underline{3}}}: 
\left(\left[n\right], T\right)\textnormal{ aufspannender Baum in }K_n
\right\}
\end{displaymath}
mit
\begin{eqnarray*}
x\left(T\right) & := & \chi\left(T\right) \\
z\left(T\right)_{\left(v, w\right), u} & := & \begin{cases}
1 & \left\{v, w\right\}\in T\wedge u\in R_{T\backslash \left\{\left\{v, w\right\}\right\}}\left(w\right)\\
0 & \textnormal{sonst.}
\end{cases}
\end{eqnarray*}
\end{Def}

Man kann sich die Frage stellen, ob $\dim P^{K_n, symm, 0} = \dim \aff XZ_n$ für alle $n$ ist.

Die Antwort auf diese Frage ist offen -- jedoch wurden Computerexperimente durchgeführt, nach deren Ergebnis für $n \leq 5$ diese Aussage erfüllt ist.

\part{Ausblick}

\chapter{Offene Fragen}

Es stellen sich natürlich offene Fragen:
\begin{itemize}
\item Gibt es eine asymptotisch kleinere symmetrische erweiterte Formulierung des Spannbaum-Polytops über dem Graphen $K_n$ als $P^{K_n, spt, symm, xy}$?
\item Gibt es eine asymptotisch kleinere asymmetrische erweiterte Formulierung des Spannbaum-Polytops über dem Graphen $K_n$ als $P^{K_n, spt, asymm, 4}$?
\item Analoge Fragen kann man natürlich auch für das Wald- und das Konnektor-Polytop stellen.
\item In Kapitel \ref{sec:konnektor_wald_erw} haben wir in Korollar \ref{kor:erw_form_wald_konnektor} deutlich freie Bedingungen an Polytope gezeigt, welche das Entwickeln erweiterter Formulierungen für das Konnektor-Polytop und das Wald-Polytop erlauben. Für die konkreten Konstruktionen haben wir jedoch lediglich (für das Konnektor-Polytop) $x \geq 0$ weggelasen. Können wir diesen Freiraum nutzen, um bessere erweiterte Formulierungen des Wald- und Konnektor-Polytops zu entwickeln?
\item Wenn wir ein lineares Programm dadurch erzeugen, dass wir $\left\langle c, x\right\rangle$ ($c \in \mathbb{R}^{\left[n\right] \choose 2}$) über einer der erweiterten Formulierungen dieser Arbeit maximieren -- welche mathematische Bedeutung hat dann das dazu duale Programm?
\item Wie kann man das Framework der W-Bäume erweitern, um weitere Typen fraktionaler Ecken der verschiedenen symmetrischen erweiterten Formulierungen des Spannbaum-Polytops zu erfassen?
\item Gilt $\dim P^{K_n, symm, 0} = \dim \aff XZ_n$ (vgl. Definition \ref{def:xz_n})?
\end{itemize}

\appendix

\backmatter

\bibliography{spanning_trees}

\end{document}
